\renewcommand{\thechapter}{1}

\chapter{Introduction}

\section{Bose-Einstein condensation}

\subsection{Phase transition of a non-interacting Bose gas}
Bose gases are characterized by the Bose-Einstein distribution giving the number of atoms $n(E_j)$ occupying each energy eigenstate $E_j$ as
\begin{equation}
n(E_j) = \frac{1}{e^{(E_j-\mu)/k_{\rm B}T}-1},
\end{equation}
where  $k_{\rm B}$ is the Boltzmann constant, $T$ is the temperature in Kelvin, $\mu$ is the chemical potential. Assuming the total atom number $N$ is fixed, the chemical potential $\mu(T,N)$ ensures that the total occupation of all $\sum_j n(E_j)=N$. 

The Bose distribution leads to Bose-Einstein condensation, the collapse of a macoscopic fraction of the atoms into the ground state. This comes as a direct consequence of the Bose distribution's characteristic $-1$ in the denominator. Consider the occupation number $n(E_j)$---it must remain positive, as a negative occupation number is unphysical. That means that the quantity $e^{(E_j-\mu)/k_{\rm B}T}$ must remain greater than $1$, or $(E_j-\mu)/k_{\rm B}T<0$ for all $E_j$. Therefore, $\mu\leq E_0$, where $E_0$ is the ground state energy. 

Then, for a given temperature $T$, there is a maximum occupation number for each excited state given by $n(E_j) = \frac{1}{e^{E_j/k_{\rm B}T}-1}$. The only energy state whose occupation number is unbounded is the ground state, as $n(E_0)$ tends toward infinity as $\mu$ tends towards $0$. Therefore, as the temperature decreases, the maximum occupation of each excited state decreases until they can no longer support all of the atoms. The remaining atoms then have no choice but to collapse into the lowest energy level and Bose condense. 

We will show this quantitatively for the case of a 3-D harmonically trapped gas of atoms, relavant to the experiments described in this thesis. It is convenient to define the fugacity $\zeta=e^{\mu/k_{\rm B}T}$, and re-write the Bose-Einstein distribution as 
eigenstate $E_j$ as
\begin{equation}
n(E_j) = \frac{\zeta}{e^{E_j/k_{\rm B}T}-\zeta}.
\end{equation}
The harmonic oscillator potentail can be written as 
\begin{equation}
V(r) = \frac{1}{2} m (\omega_x^2 x^2 + \omega_y^2 y^2 + \omega_z^2 z^2),
\end{equation}
where $\omega_x$, $\omega_y$ and $\omega_z$ are the angular trapping frequencies along ${\bf e}_x$, ${\bf e}_y$, and ${\bf e}_z$.  The eigenenergies with this potential are
\begin{equation}
E(j_x,j_y,j_z) = (\frac{1}{2} + j_x)\hbar\omega_x +(\frac{1}{2} + j_y)\hbar\omega_y+(\frac{1}{2} + j_z)\hbar\omega_z.
\end{equation}

In order to find $\mu$, we must find $\sum_{j_x,j_y,j_z}n(E(j_x,j_y,j_z))$ and set it equal to $N$. This task is greatly simplified by going to the continuum limit and finding the density of states. To do this, we neglect the zero-point energy (setting $E_0=0$, the effects of the zero-point energy are discussed in \cite{Pethick} section 2.5)  and assume there is on average one state per volume element $\hbar^3 \omega_x \omega_y \omega_z$. Then, the total number of states with energy less than or equal to some value $\epsilon$ is given by the volume of a prism made between points $(x,y,z)=(0,0,0),(\epsilon,0,0),(0,\epsilon,0)$ and $(0,0,\epsilon)$ in units of the volume element:
\begin{equation}
G(\epsilon) = \frac{\epsilon^3}{6\hbar^3\omega_x \omega_y \omega_z}.
\end{equation}
The density of states is given by 
\begin{equation}
g(\epsilon) = \frac{d}{d\epsilon} G(\epsilon) = \frac{\epsilon^2}{3\hbar^3\omega_x \omega_y \omega_z}. 
\end{equation}

Note that the occupation of the ground state is not included in this continuum picture. We can therefore use it only to calculate the total number of atoms in all of the excites states:
\begin{equation}
N_{\rm ex} = \int_0^{\infty} d\epsilon g(\epsilon) n(\epsilon) = \int_0^{\infty} d\epsilon \frac{\epsilon^2}{3\hbar\omega_x \omega_y \omega_z} \frac{\zeta}{e^{\epsilon/k_{\rm B}T}-\zeta} = \frac{(k_{\rm B}T)^3}{\hbar^3\omega_x \omega_y \omega_z}{\rm Li}_3(\zeta),
\label{eqn:excitedPopulation}
\end{equation}
where ${\rm Li}_3(\zeta)$ is the polylogarithm function\footnote{This calculation was done with Wolfram Alpha, not Russian algebra skills}. 
We define the mean trapping frequency $\bar{\omega} = (\omega_x \omega_y \omega_z)^{1/3}$ and the harmonic oscillator energy as $\hbar\bar{\omega}$, with the thermal energy in harmonic oscillator units $\tau = k_{\rm B}T/\hbar\bar{\omega}$, giving
\begin{equation}
N_{\rm ex} = \tau^3 {\rm Li}_3(\zeta).
\end{equation}

Finding the occupation number of the ground state from the Bose-Einstein distribution
\begin{equation}
N_0 = \frac{\zeta}{1-\zeta},
\label{eqn:groundPopulation}
\end{equation}
we can then find the chemical potential, or equivalently the fugacity $\zeta$, to satisfy
\begin{equation}
N = N_0 + N_{\rm ex}.
\end{equation}
This is a transcendental equation that can only be solved numerically. We present an example of the solution in Figure \ref{fig:BoseDistribution}. Here, we have calculated the fractional population in different harmonic oscillator energy levels for three different temperatures, using trapping frequencies are $\omega_x=\omega_y=\omega_z=2\pi 50$ Hz, and atom number $N=10^6$. For enerigies above the ground state (dots in the figure), we binned 50 energy levels together, such that each dot represents the total fractional population in 50 adjacent levels. This was obtained by integrating eqn. \ref{eqn:excitedPopulation} from $\epsilon - 25\hbar\bar{\omega}$ to $\epsilon + 25\hbar\bar{\omega}$. The stars represent the fractional population in just the ground state, calculated from eqn. \ref{eqn:groundPopulation}. Note that at temperature $T=255$ nK (red), the ground state population is consistent with a continuous extrapolation from the excited state populations and is almost zero. At lower temperatures, $T=180$ nK (blue) the ground state population is in excess of any reasonable extrapolation from the excited state fractions, and at $T=80$ nK (green) almost all the atoms are in the ground state. 

\begin{figure}
	\includegraphics{"BEC_DFG figures/condensation".pdf}
\caption{Occupation of energy states of a 3-D harmonic oscillator. The trapping frequencies are $\omega_x=\omega_y=\omega_z=2\pi 50$ Hz, and the atom number is $N=10^6$. Dots represent the total fractional population in 50 ajacent energy levels, including degeneracies. The stars represent the fractional population in just the ground state.  }
\label{fig:BoseDistribution}
\end{figure}

The onset of Bose-Einstein condensation occurs at a critical temperature $T_c$. This temperature is defined as the temperature at which the occupation number of excited states is equal to the atom number, ie when the atoms have occupied all available excited states and any remaining atoms will have to pile into the ground state. Since the maximal occupation of the excited states will occur at $\mu=0$, the occupation of the excited state is bounded from above by $N_{\rm ex}(\mu=0)$, and the critical temperature is defined by 
\begin{equation}
N=N_{\rm ex}(\mu=0, T=T_c)=\frac{(k_{\rm B}T_c)^3}{\hbar^3 \omega_x \omega_y \omega_z}Li_3(\zeta=1).
\end{equation}
Using $Li_3(1)\approx1.202$, we obtain for a given atom number and trapping frequencies
\begin{equation}
T_c = \frac{1.202 N}{k_{\rm B}^3}\hbar^3 \omega_x \omega_y \omega_z.
\label{eqn:tc}
\end{equation}
For the parameters in Figure \ref{fig:BoseDistribution}, $T_c = 225$ nK. 

For temperatures below the critical temperature, the condensation fraction $f_c$---the fraction of atoms in the ground state---is directly related to the ratio of the temperature to the critical temperature:
\begin{equation}
f_c=1-\frac{N}{N_{\rm ex}}=1-\frac{(k_{\rm B}T)^3}{\hbar^3 \omega_x \omega_y \omega_z}Li_3(\zeta=1)=1-\left(\frac{T}{T_c}\right)^3,
\end{equation}
where in the last step we have plugged in the definition of the critical temperature eqn. \ref{eqn:tc}.


\subsection{Interacting Bose gas}

In the previous section, we assumed there was no interaction between the atoms other than that enforced by statistics. In this section, we will relax this assumption somewhat and describe the condensed atomic state through its characteristic Gross-Pitaevskii equation. 

Since condensation occurs at very low temperatures, and thus very low kinetic energies, we can assume that any scattering processes between the atoms are $\it{s}$-wave and can be described simply by a scattering length $a$. For $^{87}$Rb, relevant to experiments described in this thesis, the scattering length between two atoms in the $F=2$ hyperfine state is $a=95.44(7) a_0$ \cite{Egorov2013}, where $a_0=5.29x10^{-11}$ m is the Bohr radius. The short-range interaction between two particles can be approximated as a contact interaction with an effective strength $U_0$ as (see \cite{Pethick} section 5.2.1):
\begin{equation}
U(r_1,r_2) = U_0 \delta(r_1-r_2) = \frac{4\pi\hbar^2 a}{m} \delta(r_1-r_2),
\end{equation}
where $m$ is the atomic mass and $\delta$ is the Dirac delta function. The full Hamiltonian of the many-body system is then
\begin{equation}
H=\sum_i \frac{p_i}{2m} + V(r_i) + U_0\sum_{i<j}\delta{r_i-r_j},
\end{equation}
where $i$ labels the particles, $p_i$ is the momentum, $r_i$ is the position, and $V$ is the external potential. 

Add in interactions: full Hamiltonain

Mean-field approximation

Thomas-Fermi radii

Position space - pictures from uwave lock?

Momentum space 

Pictures of BEC condensing (steal from talk?)


\section{Degenerate Fermi Gas}


\section{RbK apparatus}

Bird's eye view picture

Brief description of BEC making procedure

Differences from Lauren's thesis:

    Describe Hsin-I's new imaging path
    Describe extra lens for beam shaping the dipole trap

Brief description of DFG making procedure (of old)

Differences from Lauren's thesis:
	    Describe 2D MOT'