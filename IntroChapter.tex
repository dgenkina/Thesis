\renewcommand{\thechapter}{2}

\chapter{Introduction}

\section{Bose-Einstein condensation}

\subsection{Phase transition of a non-interacting Bose gas}
Bose gases are characterized by the Bose-Einstein distribution giving the number of atoms $n(E_j)$ occupying each energy eigenstate $E_j$ as
\begin{equation}
n(E_j) = \frac{1}{e^{(E_j-\mu)/k_{\rm B}T}-1},
\end{equation}
where  $k_{\rm B}$ is the Boltzmann constant, $T$ is the temperature in Kelvin, $\mu$ is the chemical potential. Assuming the total atom number $N$ is fixed, the chemical potential $\mu(T,N)$ ensures that the total occupation of all $\sum_j n(E_j)=N$. 

The Bose distribution leads to Bose-Einstein condensation, the collapse of a macoscopic fraction of the atoms into the ground state. This comes as a direct consequence of the Bose distribution's characteristic $-1$ in the denominator. Consider the occupation number $n(E_j)$---it must remain positive, as a negative occupation number is unphysical. That means that the quantity $e^{(E_j-\mu)/k_{\rm B}T}$ must remain greater than $1$, or $(E_j-\mu)/k_{\rm B}T<0$ for all $E_j$. Therefore, $\mu\leq E_0$, where $E_0$ is the ground state energy. 

Then, for a given temperature $T$, there is a maximum occupation number for each excited state given by $n(E_j) = \frac{1}{e^{E_j/k_{\rm B}T}-1}$. The only energy state whose occupation number is unbounded is the ground state, as $n(E_0)$ tends toward infinity as $\mu$ tends towards $0$. Therefore, as the temperature decreases, the maximum occupation of each excited state decreases until they can no longer support all of the atoms. The remaining atoms then have no choice but to collapse into the lowest energy level and Bose condense. 

We will show this quantitatively for the case of a 3-D harmonically trapped gas of atoms, relavant to the experiments described in this thesis. It is convenient to define the fugacity $\zeta=e^{\mu/k_{\rm B}T}$, and re-write the Bose-Einstein distribution as 
eigenstate $E_j$ as
\begin{equation}
n(E_j) = \frac{\zeta}{e^{E_j/k_{\rm B}T}-\zeta}.
\end{equation}
The harmonic oscillator potentail can be written as 
\begin{equation}
V(r) = \frac{1}{2} m (\omega_x^2 x^2 + \omega_y^2 y^2 + \omega_z^2 z^2),
\end{equation}
where $\omega_x$, $\omega_y$ and $\omega_z$ are the angular trapping frequencies along ${\bf e}_x$, ${\bf e}_y$, and ${\bf e}_z$.  The eigenenergies with this potential are
\begin{equation}
E(j_x,j_y,j_z) = (\frac{1}{2} + j_x)\hbar\omega_x +(\frac{1}{2} + j_y)\hbar\omega_y+(\frac{1}{2} + j_z)\hbar\omega_z.
\end{equation}

In order to find $\mu$, we must find $\sum_{j_x,j_y,j_z}n(E(j_x,j_y,j_z))$ and set it equal to $N$. This task is greatly simplified by going to the continuum limit and finding the density of states. To do this, we neglect the zero-point energy (setting $E_0=0$, the effects of the zero-point energy are discussed in \cite{Pethick} section 2.5)  and assume there is on average one state per volume element $\hbar^3 \omega_x \omega_y \omega_z$. Then, the total number of states with energy less than or equal to some value $\epsilon$ is given by the volume of a prism made between points $(x,y,z)=(0,0,0),(\epsilon,0,0),(0,\epsilon,0)$ and $(0,0,\epsilon)$ in units of the volume element:
\begin{equation}
G(\epsilon) = \frac{\epsilon^3}{6\hbar^3\omega_x \omega_y \omega_z}.
\end{equation}
The density of states is given by 
\begin{equation}
g(\epsilon) = \frac{d}{d\epsilon} G(\epsilon) = \frac{\epsilon^2}{3\hbar^3\omega_x \omega_y \omega_z}. 
\end{equation}

Note that the occupation of the ground state is not included in this continuum picture. We can therefore use it only to calculate the total number of atoms in all of the excites states:
\begin{equation}
N_{\rm ex} = \int_0^{\infty} d\epsilon g(\epsilon) n(\epsilon) = \int_0^{\infty} d\epsilon \frac{\epsilon^2}{3\hbar\omega_x \omega_y \omega_z} \frac{\zeta}{e^{\epsilon/k_{\rm B}T}-\zeta} = \frac{(k_{\rm B}T)^3}{\hbar^3\omega_x \omega_y \omega_z}{\rm Li}_3(\zeta),
\label{eqn:excitedPopulation}
\end{equation}
where ${\rm Li}_3(\zeta)$ is the polylogarithm function\footnote{This calculation was done with Wolfram Alpha, not Russian algebra skills}. 
We define the mean trapping frequency $\bar{\omega} = (\omega_x \omega_y \omega_z)^{1/3}$ and the harmonic oscillator energy as $\hbar\bar{\omega}$, with the thermal energy in harmonic oscillator units $\tau = k_{\rm B}T/\hbar\bar{\omega}$, giving
\begin{equation}
N_{\rm ex} = \tau^3 {\rm Li}_3(\zeta).
\end{equation}

Finding the occupation number of the ground state from the Bose-Einstein distribution
\begin{equation}
N_0 = \frac{\zeta}{1-\zeta},
\label{eqn:groundPopulation}
\end{equation}
we can then find the chemical potential, or equivalently the fugacity $\zeta$, to satisfy
\begin{equation}
N = N_0 + N_{\rm ex}.
\end{equation}
This is a transcendental equation that can only be solved numerically. We present an example of the solution in Figure \ref{fig:BoseDistribution}. Here, we have calculated the fractional population in different harmonic oscillator energy levels for three different temperatures, using trapping frequencies are $\omega_x=\omega_y=\omega_z=2\pi 50$ Hz, and atom number $N=10^6$. For enerigies above the ground state (dots in the figure), we binned 50 energy levels together, such that each dot represents the total fractional population in 50 adjacent levels. This was obtained by integrating eqn. \ref{eqn:excitedPopulation} from $\epsilon - 25\hbar\bar{\omega}$ to $\epsilon + 25\hbar\bar{\omega}$. The stars represent the fractional population in just the ground state, calculated from eqn. \ref{eqn:groundPopulation}. Note that at temperature $T=255$ nK (red), the ground state population is consistent with a continuous extrapolation from the excited state populations and is almost zero. At lower temperatures, $T=180$ nK (blue) the ground state population is in excess of any reasonable extrapolation from the excited state fractions, and at $T=80$ nK (green) almost all the atoms are in the ground state. 

\begin{figure}
	\includegraphics{"BEC_DFG figures/condensation".pdf}
\caption{Occupation of energy states of a 3-D harmonic oscillator. The trapping frequencies are $\omega_x=\omega_y=\omega_z=2\pi 50$ Hz, and the atom number is $N=10^6$. Dots represent the total fractional population in 50 ajacent energy levels, including degeneracies. The stars represent the fractional population in just the ground state.  }
\label{fig:BoseDistribution}
\end{figure}

The onset of Bose-Einstein condensation occurs at a critical temperature $T_c$. This temperature is defined as the temperature at which the occupation number of excited states is equal to the atom number, ie when the atoms have occupied all available excited states and any remaining atoms will have to pile into the ground state. Since the maximal occupation of the excited states will occur at $\mu=0$, the occupation of the excited state is bounded from above by $N_{\rm ex}(\mu=0)$, and the critical temperature is defined by 
\begin{equation}
N=N_{\rm ex}(\mu=0, T=T_c)=\frac{(k_{\rm B}T_c)^3}{\hbar^3 \omega_x \omega_y \omega_z}Li_3(\zeta=1).
\end{equation}
Using $Li_3(1)\approx1.202$, we obtain for a given atom number and trapping frequencies
\begin{equation}
T_c = \frac{1.202 N}{k_{\rm B}^3}\hbar^3 \omega_x \omega_y \omega_z.
\label{eqn:tc}
\end{equation}
For the parameters in Figure \ref{fig:BoseDistribution}, $T_c = 225$ nK. 

For temperatures below the critical temperature, the condensation fraction $f_c$---the fraction of atoms in the ground state---is directly related to the ratio of the temperature to the critical temperature:
\begin{equation}
f_c=1-\frac{N}{N_{\rm ex}}=1-\frac{(k_{\rm B}T)^3}{\hbar^3 \omega_x \omega_y \omega_z}Li_3(\zeta=1)=1-\left(\frac{T}{T_c}\right)^3,
\end{equation}
where in the last step we have plugged in the definition of the critical temperature eqn. \ref{eqn:tc}.

\begin{figure}
	\includegraphics{"BEC_DFG figures/CondensingAtoms".png}
\caption{Time-of-flight images of atoms. (a) Above the critical temperature - the atoms are thermally distirbuted. (b) Below the critical temperature - about half of the atoms are condensed in the central peak. (c) Far below the critical temperature - almost all atoms are condensed in the central peak.}
\label{fig:CondensingAtoms}
\end{figure}

Figure \ref{fig:CondensingAtoms} shows the progression towards condensation as the temperature of a cloud of \Rb{} is decreased below $T_c$. The images are obtained via a time-of-flight measurement (see section \ref{sec:timeOfFlight}), where the atoms are allowed to expand freely, mapping the initial momentum to final position, imaged via absorption imaging (see section \ref{sec:absorptionImaging}). The $x$ and $y$ axes represent momentum along $x$ and $y$, while the z axis represents the number of atoms. The $z$ axis momentum is integrated over.  Figure \ref{fig:CondensingAtoms}a shows a cloud above the condensation temperature - the momentum distribution is gaussian, given by the Maxwell-Boltzmann distribution. In  fig. \ref{fig:CondensingAtoms}b, the temperature has been decreased below $T_c$, and about half the atoms have collapsed into the ground state, producing a large peak in atom number around zero momentum. In  fig. \ref{fig:CondensingAtoms}c, the temperature has been decreased even further and almost all the atoms populate the central peak - the distribution is no longer gaussian but a sharp peak around zero momentum. 


\subsection{Interacting Bose gas}

In the previous section, we assumed there was no interaction between the atoms other than that enforced by statistics. In this section, we will relax this assumption somewhat and describe the condensed atomic state through its characteristic Gross-Pitaevskii equation. 

Since condensation occurs at very low temperatures, and thus very low kinetic energies, we can assume that any scattering processes between the atoms are $\it{s}$-wave and can be described simply by a scattering length $a$. For $^{87}$Rb, relevant to experiments described in this thesis, the scattering length between two atoms in the $F=2$ hyperfine state is $a=95.44(7) a_0$ \cite{Egorov2013}, where $a_0=5.29x10^{-11}$ m is the Bohr radius. The short-range interaction between two particles can be approximated as a contact interaction with an effective strength $U_0$ as (see \cite{Pethick} section 5.2.1):
\begin{equation}
U(r_1,r_2) = U_0 \delta(r_1-r_2) = \frac{4\pi\hbar^2 a}{m} \delta(r_1-r_2),
\end{equation}
where $m$ is the atomic mass and $\delta$ is the Dirac delta function. The full Hamiltonian of the many-body system is then
\begin{equation}
H=\sum_i \frac{p_i}{2m} + V(r_i) + U_0\sum_{i<j}\delta{r_i-r_j},
\end{equation}
where $i$ labels the particles, $p_i$ is the momentum, $r_i$ is the position, and $V$ is the external potential.

We make the mean field approximation by assuming that no interactions between two atoms take them out of the ground state, and hence all atoms can be assumed to be in the same single particle wavefunction, making the overall wavefunction
\begin{equation}
\Psi(r_1,r_2,...r_N)=\prod_i^N \phi(r_i),
\end{equation}
where $\phi$ is the single particle wavefunction. It is convenient to define the wavefunction of the condensed state, $\psi(r) = \sqrt{N}\phi(r)$, making the normalization $N=\int dr |\psi(r)|^2$.

The energy of this wavefunction under the Hamiltonian above is given by
\begin{equation}
E=\int dr\left[ \frac{\hbar^2}{2m}|\nabla\psi(r)|^2 + V(r)|\psi(r)|^2 + \frac{1}{2}U_0|\psi(r)|^4\right]
\end{equation}
Given $N$ particles, there are $N(N-1)/2$ unique pairs of particles that can have a pairwise interaction, approximately equal to $N^2/2$ for large $N$. The $N^2$ is absorbed into the definition of $\psi$, but the factor of $1/2$ remains on the final interaction term. The task of finding the condensed eigenstate reduces to minimizing this energy under the normalization constraint $N=\int dr |\psi(r)|^2$. This can be done by using the method of Lagrange multipliers to minimize $E-\mu N$. Then, we can minimize this quantity by finding the point where the derivative with respect to $\psi$ and $\psi^*$ is zero. Taking the derivative with respect to $\psi^*$ we obtain 
\begin{equation}
-\frac{\hbar^2}{2m} \nabla^2 \psi(r) + V(r)\psi(r) + U_0 |\psi(r)|^2\psi(r) = \mu \psi(r),
\end{equation}
which is the Gross-Pitaevskii equation. This is a non-linear equation that generally needs to be solved numerically.

There is another approximation that can be made in cases where the atomic density is high enough that the interaction energy is significantly larger than the kinetic energy. Then, the kinetic term in the Hamiltonian can be neglected. This is called the Thomas-Fermi approximation. Then, the wavefunction is given simply by
\begin{equation}
|\psi(r)|^2 = \frac{\mu - V(r)}{U_0}.
\end{equation}
In this approximation, the probability density simply takes the form of the inverse of the potential. In the case of a harmonically trapped BEC, it is shaped like an inverted parabola. The Thomas-Fermi radius, ie the extent of the particle wavefuntion, is the point where the probability density goes to zero: $\mu - V(r_0) = 0$. For a harmonic trap, along any direction, this is given by $r_0^2 = 2\mu/m\omega^2$. 

\begin{figure}
	\includegraphics{"BEC_DFG figures/InSitu".pdf}
\caption{In situ measurement of a fraction of bose condensed atoms. (a) Absorption image taken of $\approx1\%$ of the cloud. The $x$ and $y$ axes represent $x$ and $y$ position, while color represents the atom number. (b) The blue line repesents atom number as a function of position along hte $x$ axis, integrated over the $y$ axis. The black dashed line represents the best fit of a Gaussian function to the atomic distribution. The dashed red line represents the best fit of a Thomas-Fermi profile to the atomic distribution.}
\label{fig:InSitu}
\end{figure}

Figure \ref{fig:InSitu}a shows an absorption image of a small fraction of a BEC in situ (see section \ref{sec:timeOfFlight}), meaning as they are in the trap - without expanding in time-of-flight. Therefore, the $x$ and $y$ axis represent position, while color represents the atom number. Figure \ref{fig:InSitu}b shows the atom number integrated over the y-axis in blue. The red dashed lines represent a best fit line to a Thomas-Fermi distribution, here an inverse parabola. The black dashed lines represent a best fit of a Gaussian to the atomic distribution. The Thomas-Fermi distirbution matches the atomic distribution more closely in the center where the density is high, but the Gaussian distribution does a better job at the tails of the distribution. This is due to the presence of some thermal atoms, which remain Maxwell-Boltzmann distributed. 


\section{Degenerate Fermi Gas}

Feshbach resonances 
[COPIED FROM PAPER BEGIN]
Feshbach resonances are widely used for tuning the interaction strength in ultracold atomic gases. They have been particularly instrumental in the study of interactions and interaction-dependent processes in cold Fermi gases. In contrast to atomic Bose-Einstein condensates (BECs), where even weak interactions play a crucial role, for example giving rise to their characteristic Thomas-Fermi density profiles \cite{KetterleBEC},  interractions must compete with the Fermi energy before becoming relevant. Practically speaking, the density of Fermi clouds is typically $\sim$1000 times less than that of BECs \footnote{This is not the case for recently realized erbium and dysprosium DFGs \cite{Aikawa14,Lu12}, where strong dipolar interactions are present.}, making it necessary to enhance the strength of interactions in order to observe significant interaction effects\cite{KetterleDFG}. The tunability of interactions provided by Feshbach resonances has allowed for creation of molecular Bose-Einstein condensates from Fermi gases \cite{Greiner03,Zwierlein03, Jochim03} as well as observation of the phase transition from the Bardeen-Cooper-Schrieffer (BCS) superconduting regime to the BEC regime at sufficiently low temperatures \cite{Bartenstein04, Bourdel04, Zwierlein04, Regal04}.
\par A Feshbach resonance occurs when a diatomic molecular state energetically approaches the two-atom continuum \cite{Chin10, Timmermans99}. In experiment, the relative energy of the free atomic states in two hyperfine sublevels and the molecular state is defined by a bias magnetic field. Consequently, the Feshbach resonance can be accessed by changing the bias field. In the simple case where there are no inelastic two-body channels, such as for the \K{} resonance discussed in this work, the effect of the resonance on the scattering length between two free atoms is \cite{Chin10}
\begin{equation}
a(B)=a_{\rm{bg}}\left(1-\frac{\Delta}{B-B_0}\right),
\label{feshbachEq}
\end{equation}
where $a_{\rm{bg}}$ is the background scattering length, $\Delta$ is the width of the resonance, and $B_0$ is the field value at which the resonance occurs. The scattering length diverges at the resonance.
[COPIED FROM PAPER END]

\section{RbK apparatus}

Bird's eye view picture

Brief description of BEC making procedure

Differences from Lauren's thesis:

    Describe Hsin-I's new imaging path
    Describe extra lens for beam shaping the dipole trap

Brief description of DFG making procedure (of old)

Differences from Lauren's thesis:
	    Describe 2D MOT'