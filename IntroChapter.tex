\renewcommand{\thechapter}{2}

\chapter{Introduction}

\section{Bose-Einstein condensation}

\subsection{Phase transition of a non-interacting Bose gas}
Bose gases are characterized by the Bose-Einstein distribution giving the number of atoms $n(E_j)$ occupying each energy eigenstate $E_j$ as
\begin{equation}
n(E_j) = \frac{1}{e^{(E_j-\mu)/k_{\rm B}T}-1},
\end{equation}
where  $k_{\rm B}$ is the Boltzmann constant, $T$ is the temperature in Kelvin, $\mu$ is the chemical potential. Assuming the total atom number $N$ is fixed, the chemical potential $\mu(T,N)$ ensures that the total occupation of all $\sum_j n(E_j)=N$. 

The Bose distribution leads to Bose-Einstein condensation, the collapse of a macoscopic fraction of the atoms into the ground state. This comes as a direct consequence of the Bose distribution's characteristic $-1$ in the denominator. Consider the occupation number $n(E_j)$---it must remain positive, as a negative occupation number is unphysical. That means that the quantity $e^{(E_j-\mu)/k_{\rm B}T}$ must remain greater than $1$, or $(E_j-\mu)/k_{\rm B}T<0$ for all $E_j$. Therefore, $\mu\leq E_0$, where $E_0$ is the ground state energy. 

Then, for a given temperature $T$, there is a maximum occupation number for each excited state given by $n(E_j) = \frac{1}{e^{E_j/k_{\rm B}T}-1}$. The only energy state whose occupation number is unbounded is the ground state, as $n(E_0)$ tends toward infinity as $\mu$ tends towards $0$. Therefore, as the temperature decreases, the maximum occupation of each excited state decreases until they can no longer support all of the atoms. The remaining atoms then have no choice but to collapse into the lowest energy level and Bose condense. 

We will show this quantitatively for the case of a 3-D harmonically trapped gas of atoms, relavant to the experiments described in this thesis. It is convenient to define the fugacity $\zeta=e^{\mu/k_{\rm B}T}$, and re-write the Bose-Einstein distribution as 
eigenstate $E_j$ as
\begin{equation}
n(E_j) = \frac{\zeta}{e^{E_j/k_{\rm B}T}-\zeta}.
\end{equation}
The harmonic oscillator potentail can be written as 
\begin{equation}
V(r) = \frac{1}{2} m (\omega_x^2 x^2 + \omega_y^2 y^2 + \omega_z^2 z^2),
\end{equation}
where $\omega_x$, $\omega_y$ and $\omega_z$ are the angular trapping frequencies along ${\bf e}_x$, ${\bf e}_y$, and ${\bf e}_z$.  The eigenenergies with this potential are
\begin{equation}
E(j_x,j_y,j_z) = (\frac{1}{2} + j_x)\hbar\omega_x +(\frac{1}{2} + j_y)\hbar\omega_y+(\frac{1}{2} + j_z)\hbar\omega_z.
\end{equation}

In order to find $\mu$, we must find $\sum_{j_x,j_y,j_z}n(E(j_x,j_y,j_z))$ and set it equal to $N$. This task is greatly simplified by going to the continuum limit and finding the density of states. To do this, we neglect the zero-point energy (setting $E_0=0$, the effects of the zero-point energy are discussed in \cite{Pethick} section 2.5)  and assume there is on average one state per volume element $\hbar^3 \omega_x \omega_y \omega_z$. Then, the total number of states with energy less than or equal to some value $\epsilon$ is given by the volume of a prism made between points $(x,y,z)=(0,0,0),(\epsilon,0,0),(0,\epsilon,0)$ and $(0,0,\epsilon)$ in units of the volume element:
\begin{equation}
G(\epsilon) = \frac{\epsilon^3}{6\hbar^3\omega_x \omega_y \omega_z}.
\label{eqn:numberOfStates}
\end{equation}
The density of states is given by 
\begin{equation}
g(\epsilon) = \frac{d}{d\epsilon} G(\epsilon) = \frac{\epsilon^2}{3\hbar^3\omega_x \omega_y \omega_z}. 
\label{eqn:densityOfStates}
\end{equation}

Note that the occupation of the ground state is not included in this continuum picture. We can therefore use it only to calculate the total number of atoms in all of the excites states:
\begin{equation}
N_{\rm ex} = \int_0^{\infty} d\epsilon g(\epsilon) n(\epsilon) = \int_0^{\infty} d\epsilon \frac{\epsilon^2}{3\hbar\omega_x \omega_y \omega_z} \frac{\zeta}{e^{\epsilon/k_{\rm B}T}-\zeta} = \frac{(k_{\rm B}T)^3}{\hbar^3\omega_x \omega_y \omega_z}{\rm Li}_3(\zeta),
\label{eqn:excitedPopulation}
\end{equation}
where ${\rm Li}_3(\zeta)$ is the polylogarithm function\footnote{This calculation was done with Wolfram Alpha, not Russian algebra skills}. 
We define the mean trapping frequency $\bar{\omega} = (\omega_x \omega_y \omega_z)^{1/3}$ and the harmonic oscillator energy as $\hbar\bar{\omega}$, with the thermal energy in harmonic oscillator units $\tau = k_{\rm B}T/\hbar\bar{\omega}$, giving
\begin{equation}
N_{\rm ex} = \tau^3 {\rm Li}_3(\zeta).
\end{equation}

Finding the occupation number of the ground state from the Bose-Einstein distribution
\begin{equation}
N_0 = \frac{\zeta}{1-\zeta},
\label{eqn:groundPopulation}
\end{equation}
we can then find the chemical potential, or equivalently the fugacity $\zeta$, to satisfy
\begin{equation}
N = N_0 + N_{\rm ex}.
\end{equation}
This is a transcendental equation that can only be solved numerically. We present an example of the solution in Figure \ref{fig:BoseDistribution}. Here, we have calculated the fractional population in different harmonic oscillator energy levels for three different temperatures, using trapping frequencies are $\omega_x=\omega_y=\omega_z=2\pi 50$ Hz, and atom number $N=10^6$. For enerigies above the ground state (dots in the figure), we binned 50 energy levels together, such that each dot represents the total fractional population in 50 adjacent levels. This was obtained by integrating eqn. \ref{eqn:excitedPopulation} from $\epsilon - 25\hbar\bar{\omega}$ to $\epsilon + 25\hbar\bar{\omega}$. The stars represent the fractional population in just the ground state, calculated from eqn. \ref{eqn:groundPopulation}. Note that at temperature $T=255$ nK (red), the ground state population is consistent with a continuous extrapolation from the excited state populations and is almost zero. At lower temperatures, $T=180$ nK (blue) the ground state population is in excess of any reasonable extrapolation from the excited state fractions, and at $T=80$ nK (green) almost all the atoms are in the ground state. 

\begin{figure}
	\includegraphics{"BEC_DFG figures/condensation".pdf}
\caption{Occupation of energy states of a 3-D harmonic oscillator. The trapping frequencies are $\omega_x=\omega_y=\omega_z=2\pi 50$ Hz, and the atom number is $N=10^6$. Dots represent the total fractional population in 50 ajacent energy levels, including degeneracies. The stars represent the fractional population in just the ground state.  }
\label{fig:BoseDistribution}
\end{figure}

The onset of Bose-Einstein condensation occurs at a critical temperature $T_c$. This temperature is defined as the temperature at which the occupation number of excited states is equal to the atom number, ie when the atoms have occupied all available excited states and any remaining atoms will have to pile into the ground state. Since the maximal occupation of the excited states will occur at $\mu=0$, the occupation of the excited state is bounded from above by $N_{\rm ex}(\mu=0)$, and the critical temperature is defined by 
\begin{equation}
N=N_{\rm ex}(\mu=0, T=T_c)=\frac{(k_{\rm B}T_c)^3}{\hbar^3 \omega_x \omega_y \omega_z}Li_3(\zeta=1).
\end{equation}
Using $Li_3(1)\approx1.202$, we obtain for a given atom number and trapping frequencies
\begin{equation}
T_c = \frac{1.202 N}{k_{\rm B}^3}\hbar^3 \omega_x \omega_y \omega_z.
\label{eqn:tc}
\end{equation}
For the parameters in Figure \ref{fig:BoseDistribution}, $T_c = 225$ nK. 

For temperatures below the critical temperature, the condensation fraction $f_c$---the fraction of atoms in the ground state---is directly related to the ratio of the temperature to the critical temperature:
\begin{equation}
f_c=1-\frac{N}{N_{\rm ex}}=1-\frac{(k_{\rm B}T)^3}{\hbar^3 \omega_x \omega_y \omega_z}Li_3(\zeta=1)=1-\left(\frac{T}{T_c}\right)^3,
\end{equation}
where in the last step we have plugged in the definition of the critical temperature eqn. \ref{eqn:tc}.

\begin{figure}
	\includegraphics{"BEC_DFG figures/CondensingAtoms".png}
\caption{Time-of-flight images of atoms. (a) Above the critical temperature - the atoms are thermally distirbuted. (b) Below the critical temperature - about half of the atoms are condensed in the central peak. (c) Far below the critical temperature - almost all atoms are condensed in the central peak.}
\label{fig:CondensingAtoms}
\end{figure}

Figure \ref{fig:CondensingAtoms} shows the progression towards condensation as the temperature of a cloud of \Rb{} is decreased below $T_c$. The images are obtained via a time-of-flight measurement (see section \ref{sec:timeOfFlight}), where the atoms are allowed to expand freely, mapping the initial momentum to final position, imaged via absorption imaging (see section \ref{sec:absorptionImaging}). The $x$ and $y$ axes represent momentum along $x$ and $y$, while the z axis represents the number of atoms. The $z$ axis momentum is integrated over.  Figure \ref{fig:CondensingAtoms}a shows a cloud above the condensation temperature - the momentum distribution is gaussian, given by the Maxwell-Boltzmann distribution. In  fig. \ref{fig:CondensingAtoms}b, the temperature has been decreased below $T_c$, and about half the atoms have collapsed into the ground state, producing a large peak in atom number around zero momentum. In  fig. \ref{fig:CondensingAtoms}c, the temperature has been decreased even further and almost all the atoms populate the central peak - the distribution is no longer gaussian but a sharp peak around zero momentum. 


\subsection{Interacting Bose gas}

In the previous section, we assumed there was no interaction between the atoms other than that enforced by statistics. In this section, we will relax this assumption somewhat and describe the condensed atomic state through its characteristic Gross-Pitaevskii equation. 

Since condensation occurs at very low temperatures, and thus very low kinetic energies, we can assume that any scattering processes between the atoms are $\it{s}$-wave and can be described simply by a scattering length $a$. For $^{87}$Rb, relevant to experiments described in this thesis, the scattering length between two atoms in the $F=2$ hyperfine state is $a=95.44(7) a_0$ \cite{Egorov2013}, where $a_0=5.29x10^{-11}$ m is the Bohr radius. The short-range interaction between two particles can be approximated as a contact interaction with an effective strength $U_0$ as (see \cite{Pethick} section 5.2.1):
\begin{equation}
U(r_1,r_2) = U_0 \delta(r_1-r_2) = \frac{4\pi\hbar^2 a}{m} \delta(r_1-r_2),
\end{equation}
where $m$ is the atomic mass and $\delta$ is the Dirac delta function. The full Hamiltonian of the many-body system is then
\begin{equation}
H=\sum_i \frac{p_i}{2m} + V(r_i) + U_0\sum_{i<j}\delta{r_i-r_j},
\end{equation}
where $i$ labels the particles, $p_i$ is the momentum, $r_i$ is the position, and $V$ is the external potential.

We make the mean field approximation by assuming that no interactions between two atoms take them out of the ground state, and hence all atoms can be assumed to be in the same single particle wavefunction, making the overall wavefunction
\begin{equation}
\Psi(r_1,r_2,...r_N)=\prod_i^N \phi(r_i),
\end{equation}
where $\phi$ is the single particle wavefunction. It is convenient to define the wavefunction of the condensed state, $\psi(r) = \sqrt{N}\phi(r)$, making the normalization $N=\int dr |\psi(r)|^2$.

The energy of this wavefunction under the Hamiltonian above is given by
\begin{equation}
E=\int dr\left[ \frac{\hbar^2}{2m}|\nabla\psi(r)|^2 + V(r)|\psi(r)|^2 + \frac{1}{2}U_0|\psi(r)|^4\right]
\end{equation}
Given $N$ particles, there are $N(N-1)/2$ unique pairs of particles that can have a pairwise interaction, approximately equal to $N^2/2$ for large $N$. The $N^2$ is absorbed into the definition of $\psi$, but the factor of $1/2$ remains on the final interaction term. The task of finding the condensed eigenstate reduces to minimizing this energy under the normalization constraint $N=\int dr |\psi(r)|^2$. This can be done by using the method of Lagrange multipliers to minimize $E-\mu N$. Then, we can minimize this quantity by finding the point where the derivative with respect to $\psi$ and $\psi^*$ is zero. Taking the derivative with respect to $\psi^*$ we obtain 
\begin{equation}
-\frac{\hbar^2}{2m} \nabla^2 \psi(r) + V(r)\psi(r) + U_0 |\psi(r)|^2\psi(r) = \mu \psi(r),
\end{equation}
which is the Gross-Pitaevskii equation. This is a non-linear equation that generally needs to be solved numerically.

There is another approximation that can be made in cases where the atomic density is high enough that the interaction energy is significantly larger than the kinetic energy. Then, the kinetic term in the Hamiltonian can be neglected. This is called the Thomas-Fermi approximation. Then, the wavefunction is given simply by
\begin{equation}
|\psi(r)|^2 = \frac{\mu - V(r)}{U_0}.
\end{equation}
In this approximation, the probability density simply takes the form of the inverse of the potential. In the case of a harmonically trapped BEC, it is shaped like an inverted parabola. The Thomas-Fermi radius, ie the extent of the particle wavefuntion, is the point where the probability density goes to zero: $\mu - V(r_0) = 0$. For a harmonic trap, along any direction, this is given by $r_0^2 = 2\mu/m\omega^2$. 

\begin{figure}
	\includegraphics{"BEC_DFG figures/InSitu".pdf}
\caption{In situ measurement of a fraction of bose condensed atoms. (a) Absorption image taken of $\approx1\%$ of the cloud. The $x$ and $y$ axes represent $x$ and $y$ position, while color represents the atom number. (b) The blue line repesents atom number as a function of position along hte $x$ axis, integrated over the $y$ axis. The black dashed line represents the best fit of a Gaussian function to the atomic distribution. The dashed red line represents the best fit of a Thomas-Fermi profile to the atomic distribution.}
\label{fig:InSitu}
\end{figure}

Figure \ref{fig:InSitu}a shows an absorption image of a small fraction of a BEC in situ (see section \ref{sec:timeOfFlight}), meaning as they are in the trap - without expanding in time-of-flight. Therefore, the $x$ and $y$ axis represent position, while color represents the atom number. Figure \ref{fig:InSitu}b shows the atom number integrated over the y-axis in blue. The red dashed lines represent a best fit line to a Thomas-Fermi distribution, here an inverse parabola. The black dashed lines represent a best fit of a Gaussian to the atomic distribution. The Thomas-Fermi distirbution matches the atomic distribution more closely in the center where the density is high, but the Gaussian distribution does a better job at the tails of the distribution. This is due to the presence of some thermal atoms, which remain Maxwell-Boltzmann distributed. 


\section{Degenerate Fermi Gas}
\subsection{Fermi statistcs and the onset of degeneracy}
The occupation of different energy levels by Fermions is given by the Fermi-Dirac distribution:
\begin{equation}
n(\epsilon)= \frac{1}{e^{(\epsilon-\mu)/k_BT}+1}.
\end{equation}
The difference from the Bose-Einstein distribution is simply the sign of the $1$ in the denominator. This has important implications, however. First, since $e^x$ varies between $0$ and $\inf$, the occupation $n(\epsilon)$ varies between $1$ and $0$ - a consequence of the Pauli exclusion principle. Second, as the temperature $T$ tends towards $0$, there become two distinct cases: $\epsilon-\mu>0$ and $\epsilon-\mu<0$. If $\epsilon-\mu>0$, $e^{(\epsilon-\mu)/k_BT}$ tends towards $\inf$, and $n(\epsilon)$ tends towards $0$. If  $\epsilon-\mu<0$, $e^{(\epsilon-\mu)/k_BT}$ tends towards $0$, and $n(\epsilon)$ tends towards $1$. Therefore, at $T=0$, the energy states below the chemical potential $\mu$ are maximally occupied (with probability $1$) and the energy states above the chemical potential are unoccupied. 

We can use this to determine the chemical potential at $T=0$ by constraining the total atom number:
\begin{equation}
N=\sum_j n(E_j) = \sum_{\epsilon_j<\mu} 1.
\end{equation}
Again, we take the common example of the 3-D harmonic trap. Then the task reduces to simply finding the number of energy levels at or below a certain energy $\mu$. This is given by eqn. \ref{eqn:numberOfStates}. From this, we find the chemical potential at zero energy, which is known as the Fermi energy $E_F$, as
\begin{equation}
E_F = (6 N)^{1/3}\hbar\bar{\omega},
\end{equation}
where $\bar{\omega}=(\omega_x\omega_y\omega_z)^{1/3}$ is the geometric mean of the three trapping frequencies. From the Fermi energy, we can define the associated Fermi temperature $T_F$ as
\begin{equation}
T_F = \frac{(6 N)^{1/3}\hbar\bar{\omega}}{k_B},
\end{equation}
and the Fermi momentum $\hbar k_F$ as
\begin{equation}
\hbar k_F = \sqrt{2 m E_F},
\end{equation}
where $m$ is the mass of the Fermion. 

For higher temeperatures, we can solve for the chemical potential, or the fugacity $\zeta$, by integrating the Fermi-Dirac distribution weighted by the density of states (eqn. \ref{eqn:densityOfStates}) to obtain
\begin{equation}
N = \int_0^{\infty} \frac{\epsilon^2}{2\hbar^3 \bar{\omega}^3}\frac{\zeta}{e^{\epsilon/k_B T}+\zeta} = -\frac{(k_BT)^3}{\hbar^3 \bar{\omega}^3}Li_3(-\zeta),
\end{equation}
where $Li_3$ is again the polylogarithm function. Again, this is a transcendental equation that can be solved numerically. However, in contrast to the BEC case, we do not have to consider the ground state occupation separately, as it is bounded by $1$ like every other state. 

\begin{figure}
	\includegraphics{"BEC_DFG figures/FermiDist".pdf}
\caption{Occupation number as a function of energy for a Fermi gas of $N=10^6$ atoms in a 3-D harmonic oscillator with frequencies $\omega_x=\omega_y=\omega_z=2\pi 50$ Hz. The Fermi temperature for these parameters is $T_F=436$ nK.}
\label{fig:FermiDist}
\end{figure}

We show an example of the occupation distribution for different temperatures in Figure \ref{fig:FermiDist}. Here, we have used the same parameter values as for the BEC case: $N=10^6$ and $\omega_x=\omega_y=\omega_z=2\pi 50$ Hz. The Fermi temperature for these parameters is $T_F=436$ nK.  For illustrative purposes, we plot $n(\epsilon)$, unweighted by the density of states $g(\epsilon)$. At zero temperature (red line in the figure), only states below the Fermi energy are occupied. At higher temperatures, the distribution is smoothed out (green and orange lines) until at the Fermi temperature there is almost no significance to the Fermi energy.

In contrast with Bose-Einstein condensation, the transition to a Degenerate Fermi Gas (DFG) is not a phase transition, and there is no absolute measure of the onset of degeneracy. Instead, a Fermi gas can be considered degenerate when the occupation function $n(\epsilon)$ differs significantly from that of a thermal gas. This occurs when the temperature is of order $0.2 T_F$. 

\subsection{Interactions and Feshbach resonances}\label{sec:Feshbach}
Although the magnitgude of the contact interaction $U_0$ for DFGs is not intrinsially different from that of BECs. There are, however, two key differences. First, the Pauli exclusion principle forbids \swave{} interactions between atoms of the same spin. Higher partial wave interactions are 'frozen out' at low temperatures, when the impact parameter of the collision becomes larger than the effective cross section of interactions (see \cite{KetterleDFG}, sec. 2.1.2). Therefore, in order to observe interactions, and indeed to cool the gas to degeneracy, another species needs to be present so that intraspecies \swave{} interactions can occur. This can be a different atomic species or a different spin state of the same atom. 

Second, the densities of standard DFGs ($\approx 10^{12}$ atoms/cm$^3$) are much lower than that of BECs ($\approx 10^{14}$ atoms/cm$^3$). Since the likelyhood of two-body collisions is proportional to the atomic density $\rho^2$, this leads to a much smaller effect of interactions in DFGs.

A widely used technique for enhancing interaction effects in DFGs is Feshbach resonances. A Feshbach resonance occurs between two spiecies (either atomic species or spin species of the same atom) when the open channel, ie the two particles independently in their external potential, energetically approaches a closed channel, ie a bound molecular state of the two species, shown schematically in Figure \ref{fig:FeshbachSchematic}a. 

Generally, the atoms in an open channel are energetically sensitive to a bachground magnetic field $B$ via the hyperfine interaction $H_B=\mu\cdot B$, where $\mu$ is the magnetic dipole moment. Tuning the magnetic field should therefore tune the energy of the open channel. The molecular bound state may also have an overall magnetic moment, but it is generally not identical to that of the two atoms in the open channel, and therefore varies differently with the background field. Figure \ref{fig:FeshbachSchematic}b shows an example where the bound state has zero magnetic moment. Here, the energy of both the closed and open channel as a function of backround magnetic field $B$ is plotted in the vicinity of a Feshbach resonance. The resonance occures at a field $B_0$ where the energies of the two channels coincide.

\begin{figure}
	\includegraphics{"BEC_DFG figures/FeshbachSchematic".png}
\caption{Schematic of a Feshbach resonance. (a) Pictoral representation of energy as a function of interatomic distance for an open channel (red) and closed channel (blue). (b) Energy as a function of background magnetic field $B$ for the closed (blue) and open (red) channels. The energies coincide at the Feshbach resonance point $B_0$. Energy of Figure taken from \cite{KetterleDFG}.}
\label{fig:FeshbachSchematic}
\end{figure}

Assuming there is at least infinitesmal coupling between the closed and open channels, as the energies of the two channels approach each other the perturbative correction term to the energy grows and the interaction between the atoms is effected. This is most easily seen in the \swave{} case through changes the scattering length $a$. In the case where there are no inelastic two-body channels, such as for the \K{} resonance discussed in this thesis, the interatomic scattering length as a function of background field is given by\cite{Chin10}
\begin{equation}
a(B)=a_{\rm{bg}}\left(1-\frac{\Delta}{B-B_0}\right),
\label{eqn:feshbach}
\end{equation}
where $a_{\rm{bg}}$ is the background scattering length, $\Delta$ is the width of the resonance, and $B_0$ is the field value at which the resonance occurs. The scattering length diverges at the resonance.

The tunability of interactions provided by Feshbach resonances has allowed for creation of molecular Bose-Einstein condensates from Fermi gases \cite{Greiner03,Zwierlein03, Jochim03} as well as observation of the phase transition from the Bardeen-Cooper-Schrieffer (BCS) superconduting regime to the BEC regime at sufficiently low temperatures \cite{Bartenstein04, Bourdel04, Zwierlein04, Regal04}.

\section{RbK apparatus}

The rubidium-potassium (RbK) appartus at NIST Gaithersburg has been previously detailed in \cite{Lin2009,KarinaThesis, LaurenThesis}. In this thesis, we will give a brief overview of the apparatus and how it is used to produce BECs of \Rb{} and DFGs of \K{}, and only give detailed documentation for those parts of the apparatus that differ from previous works. 

A photograph of the main experiment is shown in Figure \ref{fig:ExperimentPicture}. This is mounted on an optical table, with the science chamber elevated above the surface of the table. The atoms start at the ovens (off to the right, not in the photograph) and travel down the Zeeman slower until they are trapped in the science chamber. The optical dipole trap laser, as well as the 1-D optical lattice laser, are located on the optical table and coupled into optical fibers, which are output on the main floor of breadboard before being sent towards the atoms. All other lasers are located on other optical table and brought over to the experiment table via optical fibers. 

\begin{figure}
	\includegraphics{"BEC_DFG figures/ExperimentPicture".png}
\caption{Photograph of RbK apparatus at NIST Gaithersburg. The main science chamber is at the center, hidden behind optics and coils. The Zeeman slower connects the atomic ovens (not shown) to the chamber. There are several levels of breadboards on which optics are mounted, labelled here as basement (surface of optical table), main floor, balcony and attic.}
\label{fig:ExperimentPicture}
\end{figure}

\subsection{Laser beams}\label{sec:laserBeams}
Figure \ref{fig:BirdsEyeApparatus} details the beam paths of the light going through the atoms. Figure \ref{fig:BirdsEyeApparatus}a shows a side view of the apparatus. The up and down going MOTcooling beams are shown in red, reaching the atoms when the flipper mirrors $M_{top}$ and $M_{bottom}$ are flipped in. The down going probe beam, used for imaging along the $x-y$ axis both in situ and in time-of-flight, is shown in solid blue. The probe beam is split via a polarizing beam splitter cube to allow for both in situ and time-of-flight imagnig of the same cloud, shown in the inset in fig.  \ref{fig:BirdsEyeApparatus}b and described in greater detail in sec. \ref{sec:BECchanges}.  The dashed blue line represents the upward going probe beam introduced for alignment purposes, described in greater detail in sec. \ref{sec:BECchanges}. 
\begin{figure}
	\includegraphics{"BEC_DFG figures/BirdsEyeApparatus".pdf}
\caption{Schematic of RbK apparatus. (a) Side view of apparatus. Only beams propagating along the \ez{} direction through the atoms is pictured. (b) Top view of apparatus. Only beams propagating along the $x-y$ plane are shown. Schematic is not to scale and the angles are approximate}
\label{fig:BirdsEyeApparatus}
\end{figure}

Figure \ref{fig:BirdsEyeApparatus}b show's a bird's eye view of the apparatus, with optics on the main floor breadboard. The slower cooling (solid dark blue) and slower repump (dashed dark blue) are coming in from the left to slow the atoms as they are moving through the Zeeman slower. The remaining four MOT cooling beams, coming from four opposing directions, are shown in red.  They reach the atoms when their flipper mirrrors, $M1-4$, are flipped in. All six flipper mirrors are computer controlled by the same digital channel, so they can be flipped in and out together. Only the beams going in through mirrors $M1$ and $M2$ are accompanied by MOT repump light, dashed red lines. 

The optical dipole trap beams (solid green) come from the same 1064 nm laser (IPG YDL-30-LP), and are split via an acousto-optic modulator into two orders, which enter from opposite directions and intersect each other at approximately a $90$ degree angle, providing confinement along all three axes. There are three available Raman beams (solid magenta): Raman A, entering past the flipped-out $M2$ mirror, Raman B, at $90$ degrees to Raman A entering past the $M1$ mirror, and Raman C, counter-propagating with Raman A and entering past the $M4$ mirror. For experiments described in Chapters \ref{SynDimChapter,BlochOscChapter}, we used the Raman A and C beams. There is a 1-D optical lattice beam (dashed green), also 1064 nm (IPG YAR-10K-1064-LP-SF, seeded by a pick off from a [FILL IN PART NUMBER]), sent in past the $M2$ mirror and retroreflected on the opposite end of the chamber to form a standing wave pattern. This was also used for experiments in Chapters \ref{SynDimChapter,BlochOscChapter}. There is also another imaging beam, imaging the atoms along the $x-z$ plane, going to a Flea3 camera. 

When \K{} atoms are in use, the slower cooling, slower repump, MOT cooling, MOT repump and imaging beams are all a combination of frequencies for both \Rb{} and \K{}, fiber coupled before they were sent to the main experiment table. [DETAIL PART NUMBERS AND FREQUENCIES] In addition, a green plug beam (solid dark green in \ref{fig:BirdsEyeApparatus}b) is used (see section \ref{sec:DFGsequence}). For \K{} experiments detailed in Chapter \ref{SwaveScatteringChapter}, we used a near resonant retroreflected optical lattice beam, shown in dashed dark magenta entering past the $M4$ flipper mirror, coming out past the $M2$ mirror before getting retro-reflected.

\subsection{Magnetic coils}\label{sec:magneticCoils}
\begin{figure}
	\includegraphics{"BEC_DFG figures/ApparatusCoils".pdf}
\caption{Schematic of magnetic coils on the RbK apparatus. The black wire frame represents the main experiment chamber, with the Zeeman slower off to the right. The Zeeman slower and reverse coils are wound around the Zeeman slower in varying spatial frequency (magenta). The quad (orange), gradient cancellation $dB_z/dz$ (bright green) and biasZ (brown) are all pairs of identical coils on the top and bottom of the apparatus. Bias X-Y coils (red) are a pair of identical coils around the axes of the $M1$ and $M3$ mirrors, and the bias X+Y (dark green) are a pair of identical coils around the axes of the $M2$ and $M4$ mirrors. The rf coils (blue) are a pair of circular coils on top of the experimental chamber, spaced enough to allow the top MOT beam through. The gradient cancellation coils $dB_{x+y}/dz$ (cyan) are four square coils on top and bottom of the experiment along the X+Y axis. }
\label{fig:ApparatusCoils}
\end{figure}

Figure \ref{fig:ApparatusCoils} is a schematic depiction of all the coils used to produce magnetic fields on the RbK apparatus. The quad coils (orange in the figure) are a large pair of coils used to produce a quadropole field for the MOT. The top and bottom coils are connected through four IGBT switches, forming an h-bridge. This allows switching between two configurations: anti-Helmholtz and Helmholtz. In anti-Helmholtz configuration, the top and bottom coils conduct current in opposite directions, producing a quadrupole field gradient at the center. This is the configuration used for the MOT, as well as for producing a Stern-Gerlach gradient for spin resolved imaging. In Helmholtz configuration, the two coils conduct current in the same direction, producing a strong bias field along the \ez{} direction. This was used to get close to the Feschbach resonance in the experiment detailed in Chapter \ref{SwaveScatteringChapter}. 

There are three pairs of bias coils, used to cancel constant background fields or provide field offsets along the three axes. All three are in Helmholtz configuration. The biasZ coils (brown) are on top and bottom of the experiment and provide a constant $B_z$ field at the center. The biasX+Y coils (dark green) are vertical on two opposite sides of the apparatus along the \ex{}+\ey{} directions, and the biasX-Y (red) are on the other two opposing sides along the \ex{}-\ey{} directions. There are also two sets of gradient cancelletion coils available, although they are not subject to feedback loops or computer control. The first is another pair of coils on top and bottom of the apparatus (bright green), connected in anti-Helmholtz configuration to produce a small gradient $dB_z/dz$. The second is four square coils mounted above and below each biasX+Y coil (cyan). Both pairs of coils (a pair here is two of the square coils one above the other) are wound in Helmholtz configuration, and the two pairs are in series, providing a small gradient $dB_{x+y}/dz$ at the atoms.  


\subsection{Procedure for making a BEC}\label{sec:BECsequence}

In the first step of the BEC making procedure, the atoms starting at the oven are cooled via a Zeeman slower and captured in a Magneto-Optical trap in the sicence chamber. During this step, the Zeeman slower is on, with both the coils and the slower cooling and repump lights on. These beams (dark blue in Figure \ref{fig:BirdsEyeApparatus}b)address the $F=2$ to $F=3$ transition of \Rb{}. At the same time, the flipper mirrors $M1-4$,$M_{bottom}$,$M_{top}$ are flipped in and the MOT cooling and repump beams (red in Figure \ref{fig:BirdsEyeApparatus}) are on. The quad coils are on in anti-Helmholtz configuration with 25 A of current running through them, producing a field gradient of $\frac{dB_z}{dz}\approx13$ Gauss/cm. This step can be set to take anywhere from $\approx0.7$ s to $\approx5$ s depending on how many atoms are necessary.  

Next is the optical molasses step. For this step, the Zeeman coils and slower lights are turned off. The quad coil current is also switched off, leaving just the MOT cooling light and only leakage MOT repump light. [LOOK UP MOLASSES DETUNINGS EXACTLY]. This is the sub-doppler cooling stage. Since the repump light is all but off in this step, the atoms are also depumped into the $F=1$ manifold. Then, the atoms are optically pumped into the $\ket{F=1,m_F=-1}$ state to make them trappable by the quadropole field. This is done by turning on the slower repump beam (dashed dark blue in fig. \ref{fig:BirdsEyeApparatus}b) [LOOK UP TIME]. Then, the XZ imaging beam (blue in fig. \ref{fig:BirdsEyeApparatus}b) is briefly turned on to get rid of any remaining $F=2$ atoms. 

Next, we compress the atoms and perform forced Rf evaporation. To compress, the quad coils arefirst turned on to 130 A. After holding for 20 ms, we sweep the current linearly to 250 A in 200 ms. The forced rf evaporation is then performed by turning on the rf coupling field and sweeping the frequency from 20 MHz to 4 MHz in 4 s [CHECK TIME] to couple the highest energy atoms from $\ket{F=1,m_F=-1}$ to $\ket{F=1,m_F=0}$ and allow them to escape the trap. The slow ramp is designed to allow the system to rethermalize through collisions as the hottest atoms are ejected. During rf evaporation, the crossed optical dipole trap

Then, the atoms are loaded into the crossed optical dipole trap, where further evaporation occurs.

% The lack of repump caused the atoms to be optically
%pumped into the F = 1 ground state. Whe optically pumped the atoms into the
%|F = 1,mF = −1i for 1 ms state by illuminating them with the slower repumping
%light. We then turn on the xz imaging probe to blow away any remaining F = 2
%atoms.
%We then rapidly turned on a quadrupole magnetic trap with a current of
%130 A in our large coils in an anti-Helmholtz configuration. After a 20 ms hold,
%we linearly ramped the current from 130 A to 250 A in 0.2 s to compress the
%atoms and increase their collision rate. After compression, we turn on our optical
%dipole trap. For all experiments in this thesis, the beam propagating along
%ex has the smallest beam waist (See Tab. 4.4 for specific beam waist information)
%in ez, the direction of gravity. We turned on the dipole trap with most of
%the power in this beam and at the highest power possible. We turned on our
%rf magnetic field and linearly ramped the frequency from 22 MHz to 4 MHz in
%4 s. This flipped the spin of the highest energy atoms and then allowed the
%remaining atoms to thermalize at a lower temperature. We decompressed the
%quadrupole trap by ramping the current in the coils from 250 A to 60 A in an
%exponential ramp with time constant τ = 1.5 s over 3 s. While decompressing,
%we lowered the atoms into the dipole trap by ramping a smaller bias coil in
%Helmholtz configuration providing field in ez from 10 A to 8 A. This lowers the
%quadrupole zero toward the dipole trap.
%We completed the evaporation to BEC in the optical dipole trap. The
%quadrupole is still on, just barely compensating for gravity. We evaporated in
%the optical dipole trap in either 3 s for typical BECs or 5 s for an elongated
%BEC. For the overall power in the dipole trap, we either used a linear ramp
%or sometimes we used an “O’Hara” ramp [103] to ramp from the initial power
%command to the final power command. The O’Hara ramp ramps the dipole potential
%more rapidly at first, then slows down. Also during this evaporation we
%linearly ramped the amplitude command to acousto-optic modulator (AOM)
%controlling the power balance in the two beams of our dipole trap. This further
%lowered the power in the beam along ex while increasing the power in the ey
%beam. After evaporation in the dipole trap, we ramped the quad current to 0 A
%with an exponential ramp with time constant τ = 0.8 s over 5 s. Ramping off the
%quadrupole continued the evaporation by decreasing the optical trap depth due
%gravity. This sequence creates a |F = 1,mF = −1i 87Rb BEC in an optical dipole
%trap.

\subsection{Changes to apparatus for Rubidium}\label{sec:BECchanges}

Differences from Lauren's thesis:

    Put master laser setup in
    PBS cube on XY imaging system
    Describe Hsin-I's new imaging path
    Describe extra lens for beam shaping the dipole trap

\subsection{Procedure for making a DFG}\label{sec:DFGsequence}
Brief description of DFG making procedure (of old)
%Our Zeeman slower slowed both 87Rb and 40K atomic beams. Because 40K is
%lighter than 87Rb, the atoms will “bloom” or acquire a transverse velocity at the
%output of the slower. To increase the number of 40K atoms ultimately captured
%in the MOT, we cooled the atoms in an orthogonal direction to the slower immediately
%after the oven, but before the slower. We loaded the 40K MOT first by
%turning on the slower cooling, slower repump, transverse cooling, MOT cooling,
%and MOT repump light for 40K, but not 87Rb. After 7 s of 40K MOT loading,
%we also loaded 87Rb for a 1.5 s of MOT loading. The current in the large, anti-
%Helmholtz was 25 A, similar to BEC creation.
%We cooled both species in optical molasses for 2 ms. For 87Rb, we did an
%abbreviated ramp of the MOT cooling light detuning from 33 MHz to 53 MHz
%away from resonance. For 40K, we lowered the intensity of the MOT cooling
%light without modifying its detuning. We optically pumped both species into
%their maximally stretched magnetically trappable states, |F = 9/2,mF = 9/2i for
%40K and |F = 2,mF = 2i for 87Rb. 40K had a dedicated optical pumping beam. We
%used the slower rempump and MOT repump light, as well. For 87Rb, we pulsed
%on the slower light and both the slower and MOT repump light.
%Both species were then loaded into a quadrupole magnetic trap with a current
%of 130 A in our large coils in anti-Helmholtz configuration. The currents in
%our 3 pairs of Helmholtz bias coils are listed in Tab. 4.2 for both the creation of a
%DFG and a BEC. Similar to the BEC sequence, we compressed the magnetic trap
%by linearly ramping the current in in the large coils from 130 A to 160 A. After
%compression, we cooled our atoms evaporatively via forced rf evaporation,
%sweeping the rf frequency from 18 MHz to 2 MHz in 10 s. During evaporation,
%the magnetic trap was plugged by a λ = 532 nm beam, tightly focused to ≈ 30 μm
%and ≈ 5W in power, providing a repulsive potential around the quadrupole
%zero to prevent Majorana losses. Since the 40K atoms were spin polarized and
%therefore only interacted by increasingly suppressed p-wave interactions, they
%re-thermalized largely due to sympathetic cooling with 87Rb atoms.
%After the 10 s evaporation in the plugged quadrupole trap, we turned on
%our crossed beam optical dipole trap at full power (∼ 6W) predominantly in
%the more tightly focused beam. Both the green plug beam and the optical
%dipole trap remain on as we decompressed the quadrupole trap by exponentially
%ramping the current from 160 A to 25.5 A with time constant τ = 1.5 s
%over 3 s. Next, we evaporated in the optical dipole trap by ramping the relative
%power balance in the crossed beams to their final balance in 3 s. This reduced
%the overall dipole trap depth because more power was in a beam with a larger
%beam waist. We also ramped off the power in the green plug beam during this
%3 s as well. We simultaneously evaporated from the dipole trap by lowering
%the overall optical power with an O’Hara ramp and using an exponential ramp
%to turn off the quadrupole field with time constant τ = 1.5 s over 2 s. At this
%juncture, we had |F = 2,mF = 2i 87Rb atoms and |F = 9/2,mF = 9/2i 40K atoms in
%our crossed beam optical dipole trap.
%We then used adiabatic rapid passage (ARP) to transfer the 87Rb atoms from
%the |F = 2,mF = 2i state to the |F = 1,mF = +1i ground state via 6.8556 GHz microwave
%coupling (20.02 MHz from the zero field resonance) followed by ramping
%the current in the bias ez coils 1 A in 50 ms. This state was chosen to minimize
%spin changing collisions with 40K atoms during further evaporation [22]. After
%the ARP, we briefly applied an on-resonant probe laser, ejecting any remaining
%87Rb atoms in the F = 2 manifold from the trap. We evaporated further in the
%optical trap again using an O’Hara ramp for 1 s. The optical potential from the
%dipole trap no longer holds the 87Rb against gravity. We used a second ARP to
%transfer the 40K atoms into the |F = 9/2,mF = −9/2i state by illuminating the
%atoms with a 3.3 MHz rf field and sweeping the current in the bias ez coils 5 A in
%150 ms. At the end of this sequence, we had a DFG of about 75, 000 40K atoms in
%the |F = 9/2,mF = −9/2i state with temperature ≈ 0.4TF.

\subsection{Current status of Potassium apparatus}
Differences from Lauren's thesis:
	    Describe 2D MOT'