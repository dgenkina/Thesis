\renewcommand{\thechapter}{4}

\chapter{Absorption Imaging with Recoil Induced Detuning}\label{AbsorptionImagingChapter}

In this Chapter, we describe the simulations we performed in order to interpret absorption images in a non-standard regime: at long imaging times, where the recoil induced detuning needed to be taken into account. This simulation was necessary to interpret data collected for our s-wave scattering experiment, described in Chapter \ref{SwaveScatteringChapter}. In this Chapter, we first describe the recoil-induced detuning effect  and derive the equations to be solved. Then, we attempt to solve these equations perturbatively, and show that this treatment is insufficient in the regime of interest. We then perform two versions of the numberical simulations: one where the atoms are assumed to remain stationary relative to each other during imaging, and one where they are free to move. We show that although the atoms do move significantly during the imaging time, this does not have a strong effect on the final observed intensity. Finally, we use our simulated results to calibrate the saturation intensity in our camera units, and find the parameters for optimal signal-to-noise (SNR) ratio imaging. This work was previously reproted in \cite{Genkina2015}.


\section{Recoil-induced detuning}

After absorbing a number of photons $N$, an atom will obtain a recoil velocity of $N v_{\rm{r}}$. Via the Doppler effect, this will result in a detuning $\delta=N k_{\rm{r}} v_{\rm{r}}$. This detuning will increase as more atoms are absorbed, and therefore depend on time, making the absorbed intensity also time dependent. We can generalize Eq. \ref{eq:dIdz} to include a time dependence on the detuning term and therefore also the intensity: 
\begin{equation}
\frac{d}{dz}\frac{I(z,t)}{I_{\rm{sat}}}=-\rho(z)\sigma_0\frac{I(z,t)/I_{\rm{sat}}}{1+4\delta(z,t)^2 /\Gamma^2+I(z,t)/I_{\rm{sat}}}. \label{eq3}
\end{equation}
The number of photons absorbed per atom will depend on the intensity lost, up until the current time, at that location. The detuning will therefore be proportional to the total number of photons lost up until time $t$ at that location, proportional to the absorbed intensity divided by the single photon energy $\hbar\omega_L$, divided by the number of atoms that participated in the aborption $\rho(z)$ times the detuning $ k_{\rm{r}} v_{\rm{r}}$:
\begin{equation}
\delta(t,z)=\frac{ k_{\rm{r}} v_{\rm{r}}}{\hbar\omega_L\rho(z)}\int_0^t\frac{dI(z,\tau)}{dz}\mathrm{d}\tau.
\label{eq4}
\end{equation}
These equations are interdependent, and cannot be in general solved analytically. 

Figure \ref{fig:expos}a shows the velocity and detuning as a function of position in space for three different imaging times, calculated numberically. All calculations in this chapter were done for a cloud of $^{40}\mathrm{K}$ atoms, as that is relevant to our experiment described in the next chapter. The resonant wavelength is $\lambda_L = 770.11$ nm, the natural linewidth of the transition is $\Gamma =  6.035$ MHz, the resulting saturation intensity and recoil velocity are $I_{\rm sat} = 17.5$ $ \rm{W/m^2}$ and $v_{\rm r}=0.01297$ m/s.  

\section{Perturbative treatment}
We can treat these equations perturbatively in time. To first order, we can set the detuning in Eq. \ref{eq3} to $\delta=0$, assume $I(z)$ is time independent, and plug that into Eq. \ref{eq4} to obtain 
\begin{align}
\delta(t,z)&=\frac{ k_{\rm{r}} v_{\rm{r}}}{\hbar\omega_L\rho(z)}\int_0^t-\rho(z)\sigma_0\frac{I(z)}{1+I(z)/I_{\rm{sat}}}\mathrm{d}\tau \\
&=\frac{ k_{\rm{r}} v_{\rm{r}}\sigma_0}{\hbar\omega_L}\frac{I(z)}{1+I/I_{\rm{sat}}}t.
\label{eq:recursive}
\end{align} 
This can then be recursively plugged into Eq. \ref{eq3} to obtain
\begin{equation}
\frac{d}{dz}\frac{I(z,t)}{I_{\rm{sat}}}=-\rho(z)\sigma_0\frac{I(z,t)/I_{\rm{sat}}}{1+4\left(\frac{ k_{\rm{r}} v_{\rm{r}}\sigma_0}{\hbar\omega_L\Gamma}\frac{I(z)}{1+I/I_{\rm{sat}}}\right)^2 t^2 +I(z,t)/I_{\rm{sat}}}.
\end{equation}
Integrating both sides of the above equation, we obtain a perturbative equation to second order in time \cite{LJLthesis}:
\begin{equation}
\sigma_0 n = ln(I_0/I_f) + \frac{I_0-I_f}{I_{\rm{sat}}} + \frac{ (k_{\rm{r}} v_{\rm{r}} t)^2}{3}\left(\frac{I_{\rm{sat}}}{I_f+I_{\rm{sat}}}-\frac{I_{\rm{sat}}}{I_0+I_{\rm{sat}}} +\rm{ ln}\left(\frac{I_f+I_{\rm{sat}}}{I_0+I_{\rm{sat}}}\right)\right).
\label{eq:perturb}
\end{equation}
In Fig. \ref{fig:expos}b, we examine for what imaging times the above perturbative equation, as well as the modelt that completely ignores recoil induced detuning, is valid. We do this by performing numerical simulations to extract a value for the final intensity $I_f$ and using Eq. \ref{eq2} and Eq. \ref{eq:perturb} to extract values $\sigma_0 n$ that would be deduced from experimnet. We find that within the recoil time, both analytic expressions start to differ from the true atomic column density by over $5\%$, and the perturbative model of Eq. \ref{eq:perturb} quickly divereges thereafter. 

In the following sections, we describe two versions of numerical simulations that we have performed in order to appropriately extract atomic column densities from experimental data.  
%
%\par The average atom velocity parallel the light field after scattering $N$ photons is $N v_{\rm{r}}$ and it is Doppler shifted  $\delta= k_{\rm{r}} N v_{\rm{r}}$ from resonance. For any probe intensity, there is an imaging time that we call the recoil time $t_{\rm{r}}$ after which  $\delta>\Gamma/2$ and, even if the probe beam is initially on resonance with the atomic transition, we cannot neglect the detuning's  effect on the scattering rate. Furthermore, this detuning varies both with imaging time $t$ and with distance along the propagation direction $z$ (Fig. \ref{fig:expos}(a)). Thus, the laser's spatially varying intensity profile in the atomic cloud also depends on time:
%\begin{equation}
%\frac{d\tilde{I}(t,z)}{dz}=-\sigma_0 \rho(t,z) \frac{\tilde{I}(t,z)}{1+4\tilde{\delta}(t,z)^2 +\tilde{I}(t,z)}. \label{eq3}
%\end{equation}
%Assuming that the atoms do not move significantly during the imaging time (we will remove this assumption shortly), the dimensionless detuning is
%\begin{equation}
%\tilde{\delta}(t,z)=\frac{ k_{\rm{r}} v_{\rm{r}}}{2\sigma_0 \rho(t,z)}\int_0^t \frac{d\tilde{I}(z,\tau)}{dz}\,\mathrm{d}\tau; \label{eq4}
%\end{equation}
%the relationship between the atomic density and the observed intensities is no longer straightforward.
\begin{figure}
	\subfigure[]{\includegraphics[scale=1.0]{"Chapter2 Figures/figure1".pdf}}
	\subfigure[]{\includegraphics[scale=1.0]{"Chapter2 Figures/figure2".pdf}}
\caption{(a) Dependence of velocity and detuning on position simulated for \K{} at three different imaging times and a probe intensity $I_0=0.8 I_{\rm sat}$. (b) Column densities deduced from optical depths obtained from recoil detuning corrected simulation of on-resonant imaging of $^{40}K$ atoms at probe intensity $I_0=0.8 I_{\rm sat}$. The blue line is the input column density $\sigma_0 n=1.6$. The green line is the high probe intensity corrected column density given by Eq. (\ref{eq2}). The red line is the column density as expanded to second order in time, Eq. (\ref{eq:perturb}).}
\label{fig:expos}
\end{figure}
%\par By considering this equation perturbatively in imaging time we obtain corrections to second order \cite{LJLthesis}
%
%\begin{eqnarray}
%&\sigma_0 n^{\rm{(2)}}  =  c_0+c_1t+c_2t^2, \mbox{ where } \\
% &c_0= \sigma_0 n^{\rm{(1)}}, c_1=0, c_2=\frac{( k_{\rm{r}} v_{\rm{r}})^2}{3}\left[\frac{1}{\tilde{I}_f+1}-\frac{1}{\tilde{I}_0+1}+\mathrm{ln}\left(\frac{\tilde{I}_f+1}{\tilde{I}_0+1}\right)\right]
%\label{eq:OD2}
%\end{eqnarray}
%As shown in  Fig. \ref{fig:expos}(b), the perturbative treatment is accurate to times up to the recoil time after which it begins to diverge. To adequately correct for the recoil induced detuning of the atoms, we numerically simulated the imaging process to obtain $\tilde{I}_f$ as a function of imaging time, atomic density, and probe intensity.
%%\begin{figure}
%	\includegraphics*{figure2.pdf}
%\caption{Using time dependent $I_f$ values obtained from recoil detuning corrected simulation of on-resonant imaging of $^{40}K$ atoms at probe intensity $I_0=0.8\, I_{\rm{sat}}$, this graph shows the optical depths obtained by each model. The `true' optical depth is $\sigma_0 n=1.6$. $OD_1$ is the high probe intensity corrected optical depth given by Eq. (\ref{eq2}). $OD_2$ is the high probe intensity corrected and expanded to second order in time optical depth, Eq. (\ref{eq:OD2}) \cite{LJLthesis}. The recoil time is the time it takes for the cloud, on average, to become detuned by a linewidth $\Gamma$. Both models start to differ from the true value before a recoil time.  }
%\label{fig:ODcorrections}
%\end{figure}
%\par In the following, we describe two versions of this simulation. First, we took a simplistic approach where the spatial distribution of atoms did not change appreciably during the imaging time: $vt\ll I_0/(\hbar \omega_{\rm{L}} \gamma_{\rm{sc}} \rho)$. We verified this approach in known limits and then cross-checked the validity of the static assumption. For realistic input parameters, we found this assumption to be invalid. We then developed a quasi-classical approach and allowed the atoms to move during the imaging time. While the atomic trajectories were wildly different than in the static atom approximation, we found that the predicted $OD$s only differed on the 0.5$\%$ level.


\section{Stationary atom model}

In order to numerically simulate the imaging process, we assume a Gaussian distribution of atoms along the propagation direction, $\rho(z) = n/sqrt{2\pi}w e^(-z^2/2w^2)$. The dependence of the result on the choice of cloud  width $w$ is discussed in the next chapter. We divide the cloud into small bins along $z$. For the initial version of the simulation, the atoms were assumed to stay within the same bins for the entire duration of the imaging pulse, i.e. the cloud shape remained constant. We then used Eqs. \ref{eq3}-\ref{eq4} to numerically propagate the probe intensity and detuning as a function of both time and space. The algorithm used is detailed by Alg. \ref{algorithm1}.
%
%To solve Eqs. (\ref{eq3})-(\ref{eq4}), we divided the cloud into spatial bins.  In this approximation, the number of atoms in each bin was time-independent.  The algorithm used is shown in Alg. (\ref{algorithm1}), in which we took a Gaussian profile for our initial density distribution. We call the optical depth simulated by this algorithm the simulated optical depth $OD^{\rm{sim1}}$.

\begin{algorithm}
\caption{Stationary atom model}
\label{algorithm1}
\begin{algorithmic}
\STATE $I[n=0,t]=I_0$ \COMMENT{$n$ is the bin index, $t$ is the time index, $I$ is in units of $I_{\rm{sat}}$}
\STATE $\delta[n, t=0]=0$ \COMMENT{light initially resonant, $\delta$ in units of $\Gamma/2$}
\STATE $H_f=0$ \COMMENT{Radiant fluence seen by camera after passing through cloud}
\FOR[loop over time steps]{$t=0$ to $t_f$}
 \FOR[loop over bins, N is total bin number]{$n=1$ to $N$}
 \STATE $A=\sigma_0\rho[n] dz$ \COMMENT{$dz$ is the size of spatial step}
 \STATE $B=v_{\rm{r}} dt/(\hbar c \rho[n])$  \COMMENT{$dt$ is the size of the time step}
\STATE $I[n,t]=I[n-1,t] - A I[n-1,t]/(1+\delta[n,t-1]^2+I[n-1,t])$  \COMMENT{Eq. (\ref{eq3})}
\STATE $\delta[n,t]=\delta[n,t-1]+B\left(I[n-1,t]-I[n,t]\right)$  \COMMENT{Eq. (\ref{eq4})}
\ENDFOR
\STATE $H_f =H_f+ I[N,t]dt$ \COMMENT{collecting total fluence seen by the camera}
\ENDFOR
\STATE $OD^{\rm{sim1}}=-\ln{(H_f/I_0t_f)}$
\end{algorithmic}
\end{algorithm}

We call the optical depth obtained in this way $OD^{\rm{sim1}}$, to distinguish if from the simulated optical depth via the method described in the next section. 

The validity of this model can be checked by considering limits where the equations are analytically solvable. For short imaging times, the recoil-induced detuning should not contribute to the optical depth, and therefore Eq. \ref{eq2} should become exact. This is seen in Fig. \ref{fig:IsatLimits}a, where the imaging pulse is only 3 $\mu \rm{s}$ long and the simulated optical depth (blue dots) agrees with that given by Eq. \ref{eq2} for all intensity regimes.  

Even at longer imaging times, the problem can be analytically solved for limits of both high and low intensity compared to the saturation intensity.  At intensities $I\gg I_{\mathrm{sat}}$, even far detuned atoms will scatter light at their maximum, and we can assume $\delta^2/\Gamma^2 \ll I/I_{\mathrm{sat}}$, reducing back to Eq. \ref{eq2}. At extremely low intensities, atoms will scatter very little light and the detuning $\delta^2/\Gamma^2 \ll 1$, again reducing back to Eq. \ref{eq2}. As seen in Fig. \ref{fig:IsatLimits} b,c the simulation agrees with the analytic Eq. \ref{eq2} in the limit of both high and low intensities. But, as the imaging time increases, the disagreement due to recoil induced detuning grows.
%
%\par We checked the validity of our simulation in the limits where the problem is analytically solvable. In the limit where the probe intensity is much weaker than the saturation intensity, $\tilde{I}_0\ll 1$, the atoms' velocities are hardly changed, and Eq.(\ref{eq3}) reduces to
%\begin{eqnarray}
%\frac{d\tilde{I}(z)}{dz}&=-\rho\sigma_0\tilde{I}(z), \mbox{ from which we recover the analytic form }\\
%\sigma_0 n^{\rm{(0)}} &= -\ln\tilde{I}_0/\tilde{I}_f. \label{eq6}
%\end{eqnarray}
%In the limit where the probe intensity is much larger than the saturation intensity, $\tilde{I}_0\gg \tilde{\delta}$, even far detuned atoms will scatter light at their maximum rate. The time dependence of the detuning can thus be neglected, and Eq. (\ref{eq3}) becomes
%\begin{eqnarray}
%\frac{d\tilde{I}(z)}{dz}&=-\rho\sigma_0, \mbox{ which integrates to }\\
%\sigma_0 n &= \tilde{I}_0 - \tilde{I}_f. \label{eq8}
%\end{eqnarray}
%We recognize the right hand sides of Eq. (\ref{eq6}) and Eq. (\ref{eq8}) as the two terms in Eq. (\ref{eq2}). Thus, as shown in  Fig. \ref{fig:IsatLimits}, $OD^{\rm{sim1}}$  coincides with the optical depth as predicted by Eq. (\ref{eq2}) in both the small and large probe intensity limits.
\begin{figure}
	\includegraphics{"Chapter2 Figures/figure3".pdf}
\caption{Optical depth as a function of probe intensity as predicted by the simulation (blue symbols) and by Eq. (\ref{eq2}) (green curves), for three different imaging times. As expected, the predictions agree in both the high and low intensity limits, and differ for probe intensities comparable to the saturation intensity and longer imaging times. }
\label{fig:IsatLimits}
\end{figure}

The simulation allows us to extract both the intensity and the detuning as a function of both time and position. We can use this information to infer the velocity and therefore the displacement of the atoms during the imaging pulse, and check if our assumption that the atoms stay in their original 'bins' during the image pulse is valid. Figure \ref{fig:simTests}a shows the position, deduced by integrating the recoil-induced velocity, as a function of time of the first, middle, and last spatial 'bin' for a probe intensity slightly above saturation, $I = 1.2 I_{\mathrm{sat}}$. As seen in the figure, not only do the atoms move beyond their 'bins', but also at long imaging times the first atoms (which have absorbed the most light) overtake the last ones. Therefore, the atomic cloud does not maintain shape during the imaging pulse, and our initial assumption is invalid.  

\begin{figure}
	\subfigure[]{\includegraphics{"Chapter2 Figures/figure4".pdf}}
	\subfigure[]{\includegraphics{"Chapter2 Figures/figure5".pdf}}
\caption{(a) Position of atoms as a function of imaging time for atoms in the first (solid green), middle (dashed red), and last (dotted blue) bins of the simulated density distribution for an initial cloud 50 \um{} in extent. The probe intensity used in this calculation was $1.2\, I_{\rm{sat}}$, and the column density was $\sigma_0 n=1.6$. (b) The velocity of a single composite atom as a function of probe intensity for various imaging times. Simulation data (dots) and numerical  solutions of Eq. (\ref{eq10}) (lines) are in agreement.}
\label{fig:simTests}
\end{figure}	

%
%We used the results of this simulation to check the self-consistency of the  stationary atom assumption, i.e. the distance traveled by the atoms (as deduced from integrating the acquired recoil velocity over the imaging time) is less than the bin size. As can be seen from Fig. \ref{fig:simTests}(a), not only do the atoms travel more than the bin size, but they travel far beyond the initial extent of the cloud. Moreover, owing to the higher initial scatter rate, the back of the cloud overtakes the front for long imaging times. Thus, the atomic distribution as a function of position changes dramatically during the imaging pulse, and the stationary assumption is invalid.


\section{Traveling atom model}

To model the recoil-induced detuning effect during the imaging pulse taking into account the potentially significant spatial displacement of the atoms, we performed a second version of our simulation. In this version, we clumped some number of atoms $N_{\rm{ca}}$ into a single composite atom, and then tracked the detuning, velocity and position of each composite atom as a function of imaging time. Tacking individual atoms would be computationally inaccessible for reasonable cloud sizes. The algorithm used is given by Alg. \ref{algorithm2}.

%
%To account for the changing atomic distribution during the imaging pulse, we numerically simulated the classical kinetics of atoms subject to the recoil driven optical forces. To simulate large ensembles in a reasonable time, we modeled composite atoms, each describing the aggregate behavior of $N_{\rm{ca}}$ atoms. The amended algorithm is shown in Alg. (\ref{algorithm2}).
\begin{algorithm}
\caption{Travelling atom model}
\label{algorithm2}
\begin{algorithmic}
\STATE $z[n]=z_0$, $\delta[n]=0$ \COMMENT{initialize position and detuning for each composite atom, labeled by index $n$}
\STATE $O[i]=n$ \COMMENT{make a list of composite atom indexes, ordered by position}
\STATE $I[n=0,t]=I_0$ \COMMENT{ $t$ is the time index, $I$ is in units of $I_{\rm{sat}}$}
\STATE $H_f=0$ \COMMENT{Radiant fluence seen by camera after passing through cloud}
\FOR[loop over time steps]{$t=0$ to $t_f$}
 \FOR[loop over superatoms]{$i=1$ to $N$}
\STATE $n=O[i]$ \COMMENT{apply probe intensity to composite atoms in order of appearance}
 \STATE $A=\sigma_0 N_{sa} dz$ \COMMENT{dz is length over which atoms were grouped into single composite atom}
 \STATE $B=v_{\rm{r}} dt/(\hbar c  N_{sa})$  \COMMENT{dt is the time step}
\STATE $I[n,t]=I[n-1,t] - A I[n-1,t]/(1+\delta[n]^2+I[n-1,t])$  \COMMENT{Eq. (\ref{eq3})}
\STATE $\delta[n]\mathrel{+}=B\left(I[n-1,t]-I[n,t]\right)$  \COMMENT{Eq. (\ref{eq4}), detuning in units of $\Gamma/2$}
\STATE $z[n]\mathrel{+}=dt\Gamma\delta/2k$ \COMMENT{$k$ is the wavenumber, $\Gamma\delta/2k$ is the velocity at $\delta$ detuning}
\ENDFOR
\STATE $O[i]$=sort($n$, key =$z[n]$) \COMMENT{sort composite atom indexes by current position}
\STATE $H_f H_f+ I[N,t]dt$ \COMMENT{collecting total fluence seen by the camera}
\ENDFOR
\STATE $OD^{\rm{sim2}}=-\ln{(H_f/I_0t_f)}$
\end{algorithmic}
\end{algorithm}

To check the validity of this version of the simulation, we check the velocity of a composite atom as a function of time in an analytically solvable limit. In this case, we take the limit of a single composite atom, such that the intensity seen by the composite atom becomes time independent. This simplifies  Eqs. \ref{eq3} and \ref{eq4} to only carry time dependence in the detuning term, and we can then plug Eq. \ref{eq3} into Eq. \ref{eq4} and differentiate both sides with respect to time to obtain
%
%\par To validate our code, we again checked the velocity predicted in this model against known limits. One such limit is that of a single composite atom. In this case, there is no attenuation, and the intensity seen by the composite atom is constant at $\tilde{I}_0$. Only the detuning  evolves in time, and Eqs. (\ref{eq3}) and (\ref{eq4}) give
\begin{equation}
\frac{d\delta(t)}{dt}= \frac{\Gamma k_{\rm{r}} v_{\rm{r}}}{2} \frac{I/I_{\rm{sat}}}{1+4\delta^2/\Gamma^2+I/I_{\rm{sat}}}.
\label{eq10}
\end{equation}
Equation (\ref{eq10}) can be solved numerically, and is in agreement with our simulation, as seen in Fig. \ref{fig:simTests}(b).


We then used this version of the simulation to look at the motion of composite atoms as a function of imaging time in phase space (ie, velocity and position). Some examples of this motion can be seen in  Fig. \ref{fig:phaseSpace}. As seen in the figure, the atomic cloud is significantly distorted during the imaging pulse and the atoms perform some crazy acrobatics. 
%We used this model to study the time evolution of the cloud shape during imaging and visualized the phase space evolution of superatoms, shown in Fig. \ref{fig:phaseSpace}. The cloud is strongly distorted during imaging.
\begin{figure}
	\subfigure[]{\includegraphics[scale=0.9]{"Chapter2 Figures/figure6a".pdf}}
	\subfigure[]{\includegraphics[scale=0.9]{"Chapter2 Figures/figure6b".pdf}}
\caption{Phase space evolution of an atomic cloud exposed to probe light with intensity $\tilde{I}_0=1.2$. We defined $\Delta v=v -\left< v(t) \right>$  and $\Delta z=z-\left< z(t) \right>$, subtracting out the center of mass position and velocity of the cloud. The column density $\sigma_0 n$ is 1.6, and the initial cloud is a Gaussian with a width of 10 $\mu$m in (a) and 1 $\mu$m in (b). The center of mass velocities $\left< v\right>$ are (0,  3.41, 5.26, 6.52, 7.50, 8.32) m/s sequentially, and are the same for both initial cloud widths. }
\label{fig:phaseSpace}
\end{figure}


It remains to check how the atoms' acrobatics affect the resulting optical depth, ie the attenuation of the probe beam. To do this, we compare the optical depths generated by our stationary atom model,  $OD^{\rm{sim1}}$ , and by our travelling atom model, $OD^{\rm{sim2}}$. The results of this comparison are seen in Fig. \ref{fig:compareModelsAndIsat}(a). As seen from the figure, the optical depths predicted by the two versions of the simulation are negligibly small -  $\left|OD^{\rm{sim1}}-OD^{\rm{sim2}}\right|/OD^{\rm{sim1}} \le 0.005$. We also checked the effect of having different initial distributions of atoms in space by varying the initial function $\rho(z)$ and keeping the total atom number constat. We found the effect of this to be negligible as well. Therefore, to infer atomic column densities from observed optical depths, it is sufficient to use the satationary atom model.



\begin{figure}
	\subfigure[]{\includegraphics{"Chapter2 Figures/figure7".pdf}}
	\subfigure[]{\includegraphics{"Chapter2 Figures/figure8".pdf}}
\caption{(a) Top. Optical depth as a function of probe intensity for an imaging time $t=100$ \us. $OD^{\rm{(1)}}$ and $OD^{\rm{(2)}}$ are optical depths predicted from a given column density by Eq. (\ref{eq2}) and (\ref{eq:perturb}) respectively.  The two versions of simulated optical depth, $OD^{\rm{sim1}}$ (green curve) and $OD^{\rm{sim2}}$ (green dots) are plotted. Bottom. The fractional difference between two versions of the simulated $OD$, $\left|OD^{\rm{sim1}}-OD^{\rm{sim2}}\right|/OD^{\rm{sim1}}$. (b) The optical depth as a function of probe intensity for three imaging times: $t=40$ \us{} (cyan),  $t=75$ \us{} (magenta),  $t=100$ \us{} (red). The dots represent experimental data and the lines represent the best fit of simulated data. The optimal fit parameters pictured are a $\sigma_0 n$ of 1.627(5) and saturation intensity of 29(7) counts/\us{}.}
\label{fig:compareModelsAndIsat}
\end{figure}


\section{Calibration of saturation intensity}

Saturation intensity is an intrinsic propety of the atom, so the idea of calibrating it may be confusing. However, there are several experimental parameters that may influence exactly what value of $I_{\rm{sat}}$ is appropriate to use in Eq. \ref{eq3} and \ref{eq4}, such as losses in the imaging system and polarization of the probe beam. In addition, we have no direct experimental access to the total radiant fluence (time integral of intensity) seen by the camera. Instead, the light hitting the charge-coupled device (CCD) camera triggers some number of photoelectrons to be registered. The proportionality between the number of photons hitting the camera and the number of photoelectrons it triggers is called the quantum efficiency $q_e$ of the camera. The number of these photelectrons, after some electronic gain and noise introduced during the readout process, is then read out as a number of 'counts' registered on each pixel. The camera-dependent factors influencing how the number of counts depends on the number of incoming photons can be convolved with the experimental factors of probe polarization and optical loss into a single calibration of the effective saturation intensity in units of  'counts' output by the camera per unit time. 

To calibrate this effective $I_{\rm{sat}}$ in camera counts per unit time, we absorption imaged our cloud of \K{} atoms for a range of probe intensities for three different values of imaging time: 40 \us{}, 100 \us{}, and 200 \us{}. We select a small region in the center of the cloud, where we can assume the atomic column density $\sigma_0 n$, and the initial probe intensity $I_0$ to be roughly constant. We then average the values of $I_0$ and $I_f$ over this region and plot the final intensity $I_f$ as a function of $I_0$. We then used the optical depth predicted by our simulation $OD^{\rm{sim}}$ and used that to simultaniously fit the three curves with $I_{\rm{sat}}$ and $\sigma_0 n$ as fit parameters, as shown in Fig. \ref{fig:compareModelsAndIsat}(b). As can be seen from the figure, this procedure not only allows us to read off $I_{\rm{sat}}$ in units of camera counts per \us{}, but also shows that our simulation accurately reproduces the differences in $OD$ dependenc on imaging time.  
%
%Our absorption images were taking using a charge-coupled device (CCD) camera.  Each camera pixel  converted the photons it was exposed to, with some efficiency, into photoelectrons, and digitally returned an integer, called `counts', that was proportional to the radiant fluence.  However, the proportionality constant depends on many factors, such as the quantum efficiency of the camera, the electronic gain during the readout process, losses in the imaging system and the polarization of the probe light.
%
%We determined this proportionality constant through direct measurement. In the limit where the system is adequately described by $\sigma_0 n=-\ln(\tilde{I}_f/\tilde{I}_0)$, only the ratio of the initial and final intensities matter, and this proportionality constant is irrelevant. In all other regimes, however, the ratio of the initial and final intensities to the saturation intensity also comes into play, making the proportionality constant significant. One way to approach this calibration is to determine the saturation intensity in units of `counts' per unit time.
%
%
%To calibrate the saturation intensity in camera counts per unit time, we took absorption images of \K{} clouds at three different imaging times (40 \us{}, 100 \us{}, and 200 \us{}) with varying probe intensities. In a small region at the center of the cloud the atomic density was approximately uniform, and we averaged the initial and final intensities of each pixel in that region. Thus, for each image we obtained $\tilde{I}_0$ and $\tilde{I}_f$, in counts per microsecond. We then simultaneously fit our simulated optical depth  $OD^{\rm{sim}}$ to this full data set, with the atomic density $\sigma_0 n$ and  $I_{\rm{sat}}$ in counts per microsecond as free parameters. As seen in Fig. \ref{fig:compareModelsAndIsat}(b), the model produced a good fit to the experimental data, and provided a calibration of the saturation intensity for our experiment.


\section{SNR optimization}

Using the simulations described in the previous sections, we can create a lookup table of atomic column densities as a function of initial and final probe intensities $I_0$ and $I_f$ for any given imaging time. This lookup table can be then used to interpret experimental data and obtain the atom number in regimes where the recoil-induced detuning is significant. This procedure can also be used to propagate photon shot noise into uncertainty in measured atom number. 

We consider Poisson distributed photon shot noise, converting into shot noise on photoelectrons triggered inside the CCD. The standard deviation will then be proportional to $q_{\rm{e}}\sqrt{N_p}$, where $q_{\rm{e}}$ is the quantum efficiency of the camera and $N_p$ is the photon number. This uncertainty can be then propagated via the lookup table into uncertainty on the measured atomic column density $\delta_{\sigma_0 n}$. The signal-to-noise ratio (SNR) can then be expressed as $\sigma_0 n/\delta_{\sigma_0 n}$.

We study the SNR as a function of imaging time and initial probe intensity for a few different atomic column densities. Some representative data is shown in  Fig. \ref{fig:SNR}. As seen in Fig. \ref{fig:SNR}(a), for a wide range of atomic column densities, extending the imaging time beyond 40 \us{} no longer yields significant improvements in SNR. There is, however, a factor of 1.5 improvement between using an imaging time of 10 \us{}, where the simple model given by Eq. \ref{eq2} is appropriate, and 40 \us{}. Therefore, there are signifiant gains that can be made by going to longer imaging times and making use of the simulated lookup table.  


This simulation allowed us to interpret experimental data. For a given imaging time, we created a look-up table of predicted optical depth as a function of probe intensity and atomic column density. We then found the observed optical depth on this table, with the given probe intensity, and inferred the atomic density. The uncertainty in the measured intensities can be propagated through this procedure, and we established optimal imaging parameters to maximize the SNR of this detection scheme. Figure \ref{fig:SNR}(b) illustrates that the optimal initial probe intensity is different for different atom numbers. For low atom numbers, $\sigma_0 n\approx0.1$, a probe intensity of $I_0\approx0.6 I_{\rm{sat}}$ is best.


%The only source of measurement uncertainty we considered was the Poisson noise on the detected arriving photons (i.e., photoelectrons) with standard deviation proportional to $q_{\rm{e}}\sqrt{N_p}$, where $q_{\rm{e}}$ is the quantum efficiency of the camera and $N_p$ is the photon number. We then propagated this uncertainty through our correction scheme to obtain the uncertainty in our deduced value of $\sigma_0 n$. We define the SNR as $\sigma_0 n/\delta_{\sigma_0 n}$, where $ \delta_{\sigma_0 n}$ is the propagated measurement uncertainty.
%
%
%As seen in Fig. \ref{fig:SNR}(a), after about 40 \us{} extending the imaging time no longer yields appreciable improvement in SNR. Imaging for 40 \us{} as opposed to 10 \us{} where the uncorrected model is appropriate, improves the SNR by a factor of  1.5. We performed the experiments described in the second section at 40 \us{} imaging time. Figure \ref{fig:SNR}(b) shows that the optimal probe intensity varies with the atomic column density. For low atom numbers, $\sigma_0 n\approx0.1$, a probe intensity of $\tilde{I}_0\approx0.6$ is best. However, in our experiment the probe intensity had a Gaussian profile and was not uniform over the whole image.  The typical probe intensities used in our experiments varied over the $2\tilde{I}_0=0.1-0.7$  range.


\begin{figure}
	\subfigure[]{\includegraphics{"Chapter2 Figures/figure9a".pdf}}
	\subfigure[]{\includegraphics{"Chapter2 Figures/figure9b".pdf}}
\caption{SNR for three different column densities after correcting for recoil induced detuning. (a) SNR as a function of imaging time for a probe intensity of $I_0=5.0 I_{\rm sat}$ and (b) SNR as a function of probe intensity for an imaging time of 50 \us{}.}
\label{fig:SNR}
\end{figure}

%
%\begin{figure}
%	\includegraphics{"Chapter2 Figures/figure8".pdf}
%\caption{The optical depth as a function of probe intensity for three imaging times: $t=40$\us{} (blue),  $t=75$\us{} (green),  $t=100$\us{} (red). The dots represent experimental data and the lines represent the best fit of simulated data. The optimal fit parameters pictured are a $\sigma_0 n$ of 1.62 and saturation intensity of 29 counts/\us{}. }
%\label{fig:isatCalib}
%\end{figure}
