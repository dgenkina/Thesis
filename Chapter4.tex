%Chapter 4

\renewcommand{\thechapter}{4}


\chapter{Synthetic dimensional lattice}
	In this chapter we explain the idea of an effective 2-D lattice with one real dimension created by lattice cites of a 1-D optical lattice and one 'synthetic' dimension created by treating the internal spin states of the atom as sites along a transverse axis. We do this by treating each dimension individually first. In the section \ref{chap:4sec:1DOL}, we describe 1-D optical lattices. We exlain the far off resonant interaction that leads to the AC Stark shift, and use it to write down the 1-D optical lattice Hamiltonian. We describe the often used tight binding approximation of the lattice Hamiltonian. We describe how the lattice is calibrated in our lab, and introduce Bloch oscillations. 


	In section \ref{chap:4sec:rf}, we describe the hyperfine structure of Rubidium, used as sites in the 'synthetic' dimension. We then describe two methods for coupling them to introduce tunneling in the synthetic dimensions: rf and Raman transitions. We describe the calibration procedures, and show ground states of the system with each of the couplings applied. In the last section, we combine the two dimensions and write out the full synthetic dimensions Hamiltonian. We plot the band structure, and describe the emergence of the effective magnetic field. We show the calibration procedure, and show the ground states of the system, featuring both bulk and edge satates with narrowing due to the magnetic length. We then show skipping orbits created by loading a superposition of states on the edge of the system, analagous to edge magnetoplasmons. The experiments in this section were detailed in the publication \cite{Stuhl2015}.


\section{One dimensional optical lattices}\label{chap:4sec:1DOL}

\subsection{Far off-resonant atom-light interaction}
	As described in section \ref{chap:2sec:atomLight}, on timescales where spontaneous emission can be neglected, two-level atoms exposed to laser radiation undergo coherent Rabi oscillations between the two levels. Starting with $c_g$ and $c_e$ as the time-dependent coefficients multiplying the eigenstate wavefuntions of the ground and excited state respectively, and assuming the atom starts in the ground state $c_g(t=0)=1$, we make the traditional transformation to the rotating frame:
\begin{align}
c'_g(t) & = c_g(t)\\
c'_e(t) & = c_e(t) e^{-i\delta t}, 
\end{align}
where $\delta$ is the detuning of laser light from resonance. In this frame, we can write the atom-light Hamiltonian in the $\begin{pmatrix} c'_g \\ c'_e \end{pmatrix}$ basis as:
\begin{equation}
H = \hbar \begin{pmatrix} -\delta/2 & \Omega/2 \\ \Omega/2 & \delta/2 \end{pmatrix},
\end{equation}
where $\Omega$ is the coupling strength, also known as the Rabi frequency. In the limit of no coupling, $\Omega = 0$, in the rotating frame the eigenenegies are $E_{\pm} = \pm \hbar \delta/2$. For non-zero coupling, finding the eigenvalues of H gives $E_{\pm} = \pm \hbar \sqrt{\delta^2 + \Omega^2}/2$. Therefore, the bare (without light) eigenenrgies are shifted in the presence of the light. 

	For a far detuned laser beam, one expects that no absorption of the light will actually take place, and the atom will remain entirely in the ground state. Indeed, solving the Shroedinger equation with the above Hamiltonian
\begin{equation}
i\hbar \frac{d}{dt} \begin{pmatrix} c'_g \\ c'_e \end{pmatrix} = H \begin{pmatrix} c'_g \\ c'_e \end{pmatrix}
\end{equation}
we obtain the oscillating excited state population 
\begin{equation}
c'_e(t) = -i \frac{\Omega}{\sqrt{\Omega^2 + \delta^2}}{\rm sin}\left(\frac{\sqrt{\Omega^2 + \delta^2}t}{2}\right),
\end{equation}
where the amplitude of the oscillation approches zero in the limit $\Omega \ll \delta$. Thus, the only effect of the light in this regime is to shift the eigenenergies of the ground and excited states. Expanding the energies in the small parameter $\Omega/\delta$, we obtain the shifted energies $E_{\pm} = \pm \hbar \sqrt{\delta^2 + \Omega^2}/2 \approx \pm (\delta/2 + \Omega^2/4\delta)$. The shift from bare energy levels is thus 
\begin{equation}
\Delta E_{\pm} = \pm \Omega^2/4\delta.
\end{equation}
This laser intensity dependent energy shift is called the AC Stark shift, and is the basis of most laser created potentials for cold atoms. 
	
	For the ground state, and a red detuned laser beam (where the laser frequency is lower than the resonant frequency), this creates energy minima in locations of maximal laser intensity. For the lattice described in this chapter, as well as for the trapping of our atoms in the final stages of cooling, we use high power (up to 10 W) lasers with wavelength $\lambda_L = 1064 $ nm. 

\subsection{Lattice Hamiltonian}

	Our 1-D optical lattice is created by retro-reflecting the  $\lambda_L = 1064 $ nm laser, creating a standing wave of light. Via the AC Stark shift, this creates a periodic potential for the atoms of the form
\begin{equation}
V = V_0 {\rm sin}^2 (k_L x),
\end{equation} 
where $k_L = 2\pi/\lambda_L$ is the wavenumber associated with the lattice recoil momentum. The time-independent Hamiltonian, for some eigenenergy $E_n$, will be given by
\begin{equation}
-\frac{\hbar^2}{2 m}\frac{d^2}{dx^2}\Psi_n(x)+V_0 {\rm sin}^2 (k_L x) \Psi_n(x) = E_n \Psi_n(x).
\end{equation}
Since the potential is spatially periodic, we can invoke Bloch's theorem \cite{Ashcroft}:
\begin{equation}
\Psi_{n,q} =  e^{iqx}u_{n,q}(x), 
\end{equation}
where $q$ is the crystal momentum restricted to $\pm\hbar k_L$, and $u_{n,q}(x)$ is the spatially varying part of the wavefunction. 
Plugging this in to the Hamiltonian, we obtain
\begin{equation}
-\frac{\hbar^2}{2m}\left(-q^2 + 2iq\frac{d}{dx} + \frac{d^2}{dx^2}\right) u_{n,q}(x) + V_0{\rm sin}^2(k_L x) u_{n,q}(x) = E_n u_{n,q}(x).
\end{equation}
Expanding $u_{n,q}(x)$  in Fourier components commensurate with the lattice period of $2 k_L$ as $u_{n,q}(x) = \sum_{j=-\inf}^{\inf} a_j e^{i 2 k_L j x}$, we obtain
\begin{equation}
\sum_j \left(\frac{\hbar^2}{2m}(q + 2 k_L)^2 a_j + V_0 {\rm sin}^2(k_L x) a_j\right) e^{i 2 k_L j x} = E_n\sum_j a_j  e^{i 2 k_L j x}.
\end{equation}
Re-writing ${\rm sin}^2(k_L x) = (e^{-2ik_L x} + e^{2 i k_L x} - 2)/4$, multiplying both sides by $e^{i 2 k_L j' x}$ and invoking $\sum c_j e^{i k (j-j')} = \delta_{jj'}$, where $\delta_{j j'}$ is the Kroniker delta and $c_j$ are appropriately normalized coefficients, we get for any value of the index $j$
\begin{equation}
\frac{\hbar^2}{2m}(q + 2 k_L j)^2 a_j - \frac{V_0}{4}(a_{j+1}+a_{j-1}) = E_n a_j.
\end{equation}
This can be expressed in matrix form, in the basis of momentum orders
\begin{equation}
 \begin{pmatrix} \ddots &  & & & & & & \\ 
 & \frac{\hbar^2}{2m}(q + 4 k_L j)^2 & \frac{V_0}{4} & 0 & 0 & 0 &  \\
 & \frac{V_0}{4} &\frac{\hbar^2}{2m}(q + 2 k_L j)^2 & \frac{V_0}{4} & 0 & 0 &  \\
& 0 & \frac{V_0}{4} & \frac{\hbar^2}{2m}q^2 & \frac{V_0}{4} & 0 &  \\
 & 0 & 0 & \frac{V_0}{4} & \frac{\hbar^2}{2m}(q - 2 k_L j)^2 &\frac{V_0}{4} &  \\
 &  & 0 & 0 & \frac{V_0}{4} & \frac{\hbar^2}{2m}(q + 4 k_L j)^2 &  \\
& & & & & & &  \ddots \end{pmatrix} \\
\end{equation}
This matrix can be diagonalized for every value of the crystal momentum $q$, with some examples shown in Figure [ADD BAND STRUCTURE FIGURE]
In any numerical simulation, the number of momentum orders that can be included is finite. We determine the value of the parameter $n = {\rm max}(|j|)$ as the lowest $n$ at which the eigenvalues stop changing to machine precision from $n-1$. The code for finding and plotting the eigenvalues and eigenvectors of the lattice hamiltonian is included in Appendix [MAKE APPENDIX WIHT CODE].
\subsection{Tight binding approximation}


\subsection{Pulsed lattice}


\subsection{Adiabatically loaded lattice}

\section{Raman and rf coupling}\label{chap:4sec:rf}


\subsection{Hyperfine structure}


\subsection{Raman and rf coupling Hamiltonians}


\subsection{Adiabatically loaded Raman and rf dressed states}


\section{Synthetic dimensions}\label{chap:4sec:syn_dim}


\subsection{Lattice Hamiltonian with Raman coupling}


\subsection{Emergence of effective magnetic field}


\subsection{Eigenstates of the synthetic 2-D lattice}


\subsection{Observation of skipping orbits}