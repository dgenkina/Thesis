%Chapter 4

\renewcommand{\thechapter}{4}


\chapter{Synthetic dimensional lattice}
	In this chapter we explain the idea of an effective 2-D lattice with one real dimension created by lattice cites of a 1-D optical lattice and one 'synthetic' dimension created by treating the internal spin states of the atom as sites along a transverse axis. We do this by treating each dimension individually first. In the section \ref{chap:4sec:1DOL}, we describe 1-D optical lattices. We exlain the far off resonant interaction that leads to the AC Stark shift, and use it to write down the 1-D optical lattice Hamiltonian. We describe the often used tight binding approximation of the lattice Hamiltonian. We describe how the lattice is calibrated in our lab, and introduce Bloch oscillations. 


	In section \ref{chap:4sec:rf}, we describe the hyperfine structure of Rubidium, used as sites in the 'synthetic' dimension. We then describe two methods for coupling them to introduce tunneling in the synthetic dimensions: rf and Raman transitions. We describe the calibration procedures, and show ground states of the system with each of the couplings applied. In the last section, we combine the two dimensions and write out the full synthetic dimensions Hamiltonian. We plot the band structure, and describe the emergence of the effective magnetic field. We show the calibration procedure, and show the ground states of the system, featuring both bulk and edge satates with narrowing due to the magnetic length. We then show skipping orbits created by loading a superposition of states on the edge of the system, analagous to edge magnetoplasmons. The experiments in this section were detailed in the publication \cite{Stuhl2015}.


\section{One dimensional optical lattices}\label{chap:4sec:1DOL}
	The first, 'real' dimension in the synthetic dimensional lattice is discretized into sites by an optical lattice. Optical lattices are a staple of cold atomic physics, and are commonly used to emulate the crystal structure present in metals and other condensed matter systems. In this section, we describe the orgin of the periodic potential, derive the Hamiltonian and show how the lattice depth is detected and calibrated in the lab. 

\subsection{Far off-resonant atom-light interaction}
	As described in section \ref{chap:2sec:atomLight}, on timescales where spontaneous emission can be neglected, two-level atoms exposed to laser radiation undergo coherent Rabi oscillations between the two levels. Starting with $c_g$ and $c_e$ as the time-dependent coefficients multiplying the eigenstate wavefuntions of the ground and excited state respectively, and assuming the atom starts in the ground state $c_g(t=0)=1$, we make the traditional transformation to the rotating frame:
\begin{align}
c'_g(t) & = c_g(t)\\
c'_e(t) & = c_e(t) e^{-i\delta t}, 
\end{align}
where $\delta$ is the detuning of laser light from resonance. In this frame, we can write the atom-light Hamiltonian in the $\begin{pmatrix} c'_g \\ c'_e \end{pmatrix}$ basis as:
\begin{equation}
H = \hbar \begin{pmatrix} -\delta/2 & \Omega/2 \\ \Omega/2 & \delta/2 \end{pmatrix},
\end{equation}
where $\Omega$ is the coupling strength, also known as the Rabi frequency. In the limit of no coupling, $\Omega = 0$, in the rotating frame the eigenenegies are $E_{\pm} = \pm \hbar \delta/2$. For non-zero coupling, finding the eigenvalues of H gives $E_{\pm} = \pm \hbar \sqrt{\delta^2 + \Omega^2}/2$. Therefore, the bare (without light) eigenenrgies are shifted in the presence of the light. 

	For a far detuned laser beam, one expects that no absorption of the light will actually take place, and the atom will remain entirely in the ground state. Indeed, solving the Shroedinger equation with the above Hamiltonian
\begin{equation}
i\hbar \frac{d}{dt} \begin{pmatrix} c'_g \\ c'_e \end{pmatrix} = H \begin{pmatrix} c'_g \\ c'_e \end{pmatrix}
\end{equation}
we obtain the oscillating excited state population 
\begin{equation}
c'_e(t) = -i \frac{\Omega}{\sqrt{\Omega^2 + \delta^2}}{\rm sin}\left(\frac{\sqrt{\Omega^2 + \delta^2}t}{2}\right),
\end{equation}
where the amplitude of the oscillation approches zero in the limit $\Omega \ll \delta$. Thus, the only effect of the light in this regime is to shift the eigenenergies of the ground and excited states. Expanding the energies in the small parameter $\Omega/\delta$, we obtain the shifted energies $E_{\pm} = \pm \hbar \sqrt{\delta^2 + \Omega^2}/2 \approx \pm (\delta/2 + \Omega^2/4\delta)$. The shift from bare energy levels is thus 
\begin{equation}
\Delta E_{\pm} = \pm \Omega^2/4\delta.
\end{equation}
This laser intensity dependent energy shift is called the AC Stark shift, and is the basis of most laser created potentials for cold atoms. 
	
	For the ground state, and a red detuned laser beam (where the laser frequency is lower than the resonant frequency), this creates energy minima in locations of maximal laser intensity. For the lattice described in this chapter, as well as for the trapping of our atoms in the final stages of cooling, we use high power (up to 10 W) lasers with wavelength $\lambda_L = 1064 $ nm. 

\subsection{Lattice Hamiltonian}

	Our 1-D optical lattice is created by retro-reflecting the  $\lambda_L = 1064 $ nm laser, creating a standing wave of light. Via the AC Stark shift, this creates a periodic potential for the atoms of the form
\begin{equation}
V = V_0 {\rm sin}^2 (k_L x),
\end{equation} 
where $k_L = 2\pi/\lambda_L$ is the wavenumber associated with the lattice recoil momentum. The time-independent Hamiltonian, for some eigenenergy $E_n$, will be given by
\begin{equation}
-\frac{\hbar^2}{2 m}\frac{d^2}{dx^2}\Psi_n(x)+V_0 {\rm sin}^2 (k_L x) \Psi_n(x) = E_n \Psi_n(x).
\end{equation}
Since the potential is spatially periodic, we can invoke Bloch's theorem \cite{Ashcroft}:
\begin{equation}
\Psi_{n,q} =  e^{iqx}u_{n,q}(x), 
\end{equation}
where $q$ is the crystal momentum restricted to $\pm\hbar k_L$, and $u_{n,q}(x)$ is the spatially varying part of the wavefunction. 
Plugging this in to the Hamiltonian, we obtain
\begin{equation}
-\frac{\hbar^2}{2m}\left(-q^2 + 2iq\frac{d}{dx} + \frac{d^2}{dx^2}\right) u_{n,q}(x) + V_0{\rm sin}^2(k_L x) u_{n,q}(x) = E_n u_{n,q}(x).
\end{equation}
Expanding $u_{n,q}(x)$  in Fourier components commensurate with the lattice period of $2 k_L$ as $u_{n,q}(x) = \sum_{j=-\inf}^{\inf} a_j e^{i 2 k_L j x}$, we obtain
\begin{equation}
\sum_j \left(\frac{\hbar^2}{2m}(q + 2 k_L)^2 a_j + V_0 {\rm sin}^2(k_L x) a_j\right) e^{i 2 k_L j x} = E_n\sum_j a_j  e^{i 2 k_L j x}.
\end{equation}
Re-writing ${\rm sin}^2(k_L x) = (e^{-2ik_L x} + e^{2 i k_L x} - 2)/4$, multiplying both sides by $e^{i 2 k_L j' x}$ and invoking $\sum c_j e^{i k (j-j')} = \delta_{jj'}$, where $\delta_{j j'}$ is the Kroniker delta and $c_j$ are appropriately normalized coefficients, we get for any value of the index $j$
\begin{equation}
\frac{\hbar^2}{2m}(q + 2 k_L j)^2 a_j - \frac{V_0}{4}(a_{j+1}+a_{j-1}) = E_n a_j.
\end{equation}
This can be expressed in matrix form
\begin{equation}
H_L =
 \begin{pmatrix} \ddots &  & & & & & & \\ 
 & \frac{\hbar^2}{2m}(q + 4 k_L )^2 & \frac{V_0}{4} & 0 & 0 & 0 &  \\
 & \frac{V_0}{4} &\frac{\hbar^2}{2m}(q + 2 k_L )^2 & \frac{V_0}{4} & 0 & 0 &  \\
& 0 & \frac{V_0}{4} & \frac{\hbar^2}{2m}q^2 & \frac{V_0}{4} & 0 &  \\
 & 0 & 0 & \frac{V_0}{4} & \frac{\hbar^2}{2m}(q - 2 k_L)^2 &\frac{V_0}{4} &  \\
 &  & 0 & 0 & \frac{V_0}{4} & \frac{\hbar^2}{2m}(q - 4 k_L)^2 &  \\
& & & & & & &  \ddots \end{pmatrix} \\,
\end{equation}
in the basis of momentum orders $\ket{k}=e^{i k x}$ given by:
\begin{equation}
 \begin{pmatrix} \vdots \\
\ket{q+4 k_L}\\
\ket{q+2 k_L}\\
\ket{q}\\
\ket{q -2 k_L}\\
\ket{q - 4 k_L}\\
\vdots
\end{pmatrix} \\.
\end{equation}

This matrix can be diagonalized for every value of the crystal momentum $q$, with some examples shown in Figure [ADD BAND STRUCTURE FIGURE]. It is convenient to define the lattice recoil energy $E_L = \hbar^2 k_L^2/2m$. Then, we can re-write the Hamiltonian with $V_0$ in units of $E_L$ and momenta $q$ in units of $k_L$ as 
\begin{equation}
H_L/E_L =
 \begin{pmatrix} \ddots &  & & & & & & \\ 
 & (q + 4)^2 & \frac{V_0}{4} & 0 & 0 & 0 &  \\
 & \frac{V_0}{4} &(q + 2)^2 & \frac{V_0}{4} & 0 & 0 &  \\
& 0 & \frac{V_0}{4} & q^2 & \frac{V_0}{4} & 0 &  \\
 & 0 & 0 & \frac{V_0}{4} & (q - 2)^2 &\frac{V_0}{4} &  \\
 &  & 0 & 0 & \frac{V_0}{4} & (q  - 4)^2 &  \\
& & & & & & &  \ddots \end{pmatrix} \\.
\end{equation}

In any numerical simulation, the number of momentum orders that can be included is finite. We determine the value of the parameter $n = {\rm max}(|j|)$ as the lowest $n$ at which the eigenvalues stop changing to machine precision from $n-1$. The code for finding and plotting the eigenvalues and eigenvectors of the lattice hamiltonian is included in Appendix [MAKE APPENDIX WITH CODE?].

\subsection{Tight binding approximation}

In the limit of large lattice depths, $V_0 > \approx 5 E_L$, the lattice Hamiltonian is well approximated by the tight-binding model. In the tight binding model, the basis is assumed to be a set of orthogonal functions, called Wannier functions, localized to each lattice site $\ket{j}$.  The approximation lies in assuming only nearest neighbor tunnelings between the sites, forming the tight-binding Hamiltonian
\begin{equation}
H_{\rm tb} = - t \ket{j}\bra{j+1} + {\rm H.c.},
\end{equation}
where $t$ is the tunneling amplitude between nearest neighbor sites and H.c. stands for Hermitian conjugate. We have neglected the diagonal kinetic energy term, as it will be equal for every Wannier function $\ket{j}$ and thus represents a constant energy offset. All the information about the lattice depth is therefore reflected in the tunneling amplitude $t$. 

The tight binding Hamiltonian can also be expressed in the momentum basis by Fourier transforming the basis functions:
\begin{equation}
\ket{j} = \frac{1}{\sqrt{N}}\sum_{k_j} e^{-i k_j j}\ket{k_j},
\end{equation}
giving the Hamiltonian
\begin{equation}
H_{\rm tb} = - \frac{1}{N}\sum_{k_1}\sum{k_2} t e^{-i j k_1} e^{i k_2 (j+1)} \ket{k_1}\bra{k_2} + {\rm H.c} = 2 t {\rm cos}(k)\ket{k}\bra{k}.
\end{equation}  
From this we can directly read off the band structure of the tight binding Hamiltonian. First, we notice that we only obtain one band - to approximate higher bands with the tight binding approximation we would need to construct a different set of Wannier functions and a different tunneling strength. Second, we see that the lowest band is simply a cosine - therefore we have solved for the band structure without even defining what the basis Wannier functions are! Third, the amplitude of the cosine function is given by the tunneling strength $t$. This gives us a good clue as to how to determine the appropriate tunneling given a lattice depth $V_0$ - [FIND EXPRESSION FOR AMPLITUDE OF LOWEST BAND COSINE AT LARGE V0] simply find a $t$ that matches the amplitude of the lowest band. 

The precise form of the Wannier functions depends on both the depth of the lattice and the band being reproduced. It is not necessary for us to find their full expression, as the band structure can be calculated without them. The most commonly used definition, however, is
\begin{equation}
\ket{j} = \int_{\rm BZ} e^{i q j a}\Psi_q(x) {\rm d} q,
\end{equation}
where the integral is over the Brillouin zone, from $-k_L$ to $k_L$, $a$ is the lattice spacing $\lambda_L/2$, and $\Psi_q$ is the Bloch wavefunction at crystal momentum $q$. 

\subsection{Pulsing vs adiabatic loading of the lattice}

The lattice depth parameter $V_0/4$, for a range of values, can be well calibrated experimentally by pulsing on the lattice. Here, the word pulsing indicates that the lattice is turned on fully non-adiabatically, if not instantaneously, such that the original bare momentum state is projected onto the lattice eigenbasis. If the atoms start out stationary in the trap, the bare state in the momentum basis is simply
\begin{equation}
\ket{\Psi_0} = \begin{pmatrix} \vdots \\
0\\
0\\
1\\
0\\
0\\
\vdots
\end{pmatrix} \\.
\end{equation}

Since the lattice eigenbasis is distinct from the bare one, this will necessarily excite the atoms into a superposition of lattice eigenstates, each evolving with a different phase according to the eigenenergy while the lattice is on. Then, when the lattice is snapped back off, the wavefunction is projected back into the bare basis, and the varying phase accumulation results in a beating of the different momentum orders. This can be calculated simply by using the time evolution operator
\begin{equation}
\ket{\Psi(t)} = e^{-i H_L t/\hbar}\ket{\Psi_0}.
\label{eqn:timeEvolveLat}
\end{equation}
By pulsing on the lattice for variable amounts of time $t$, we can obtain fractional populations in the different momentum states. Time-of-flight imaging captures the momentum distribution of the cloud, and the different entries of $\Psi(t)$ in the momentum basis will thus appear as different clouds on the absorption image [INCLUDE IMAGES FROM PULSING DATA]. The fractional population in these clouds corresponds to a measurement of $|a_j|^2$.  Typically for our values of the lattice depth $V_0 < 10 E_L$, it is sufficient to simply count three central momentum orders, $k = q, q \pm 2 k_L$. Then, we can fit Eq. \ref{eqn:timeEvolveLat} to the data with fitting parameter $V_0$, thus deducing the lattuce depth. Some examples of these pulsing experiments are presented in figure [MAKE FIGURE]:
%Flea3,PIXIS_20Aug2017_0034-0063
%low lattice power:
%Flea3,PIXIS_20Aug2017_0124-0153

In contrast to pulsing, adiabatic loading turns the lattice on slowly, such that the atomic wavefunction starting in the bare ground state can continuously adjust to remain in the ground state of the current Hamiltonian, without projecting onto any of the higher bands. The adiabatic timescale depends on the spacing between the ground and next excited band (or if starting in a different eigenstate, the nearest eigenstate). If the energy difference between the ground and first excited state is $\Delta E$, the timescale on which the lattice is turned on must fullfill $t \gg h/\Delta E$.  

[FIND PICTURES OF ADIABATICALLY LOADED LATTICE STATES]



\section{Raman and rf coupling}\label{chap:4sec:rf}

	The second, 'synthetic' dimension in the effectively 2-D lattice is formed by the internal hyperfine states of the atoms, forming sites along a second dimension. To induce tunneling along the synthetic sites, analogous to lattice hopping between neighboring sites, we must engeneer some coupling between them. There are two ways we induce this tunneling - with rf coupling, tuned to be directly resonant with the energy difference between the hyperfine levels, and with two-photon Raman coupling, tuned such that the energy difference between the two photons matches the hyperfine splitting. In this section we describe the hypefine structure of \Rb{}, derive the Hamiltonians for both rf and Raman coupled states, and show how the Raman and rf coupled states are measured and calibrated in the lab.


\subsection{Hyperfine structure}


\subsection{Raman and rf coupling Hamiltonians}


\subsection{Adiabatically loaded Raman and rf dressed states}


\section{Synthetic dimensions}\label{chap:4sec:syn_dim}


\subsection{Lattice Hamiltonian with Raman coupling}


\subsection{Emergence of effective magnetic field}


\subsection{Eigenstates of the synthetic 2-D lattice}


\subsection{Observation of skipping orbits}