\renewcommand{\thechapter}{4}

\chapter{Direct Imaging of Scattering Near a Feshbach Resonance}\label{chap:SwaveScattering}

In this chapter, we describe our experiment directly imaging \swave{} scattering halos of \K{} atoms in the vicinity of a Fesbach resonance between the $\ket{F=9/2,m_F=-9/2}$ and $\ket{F=9/2,m_F=-7/2}$ internal states. We used this data to extract the location of the magnetic fields resonance of 20.206(15) \mT{} and a width of 1.0(5) \mT{}, similar to the accepted values of 20.210(7) \mT{} and 0.78(6) \mT{} \cite{Regal04}. The data presented in this chapter was previously reported in \cite{Genkina2015}. 

We first introduced Feshbach resonances in section \ref{sec:Feshbach}. Although Feshbach resonances are extremely useful for studying and manipulating Fermi gases, their resonant magnetic field values are difficult to predict analytically and are commonly computed via numerical models based on experimental input parameters \cite{Tiesinga93, Lysebo09, Gao11} or determined experimentally \cite{Inouye98, Cornish00}. There have been a variety of experimental techniques used to characterize Feshbach resonances, including measuring atom loss due to three-body inelastic scattering, measurement of re-thermalization timescales, and anisotropic expansion of the cloud upon release from a confining potential, all of which infer the elastic scattering cross section from collective behavior of the cloud \cite{Regal03,OHara02,Monroe93}. 

Here, we present an alternative technique, where we directly image the enhancement in elastic scattering due to the resonance. We collided pairs of ultra-cold Fermi gases and directly imaged the resulting \swave{} scattered atoms as a function of magnetic field strength. This allowed us to observe the enhancement in scattering without relying on proxy effects. We measured the fraction of atoms scattered during the collision, and from this fraction deduced the resonant magnetic field  and the width of the resonance.

In our dilute DFGs, even with the resonant enhancement of the scattering cross section, only a small fraction of the atoms scattered as the clouds passed through each other. This made direct detection of scattered atoms difficult due to detection uncertainty that disproportionately affected regions of low atomic density. To optimize the signal-to-noise ratio (SNR) for low atom numbers, we absorption imaged with fairly long, high-intensity pulses --- a non-standard regime, where the atoms acquired a velocity during imaging and the resulting Doppler-shift was non-negligible.  Simulation of the absorption imaging process was necessary for an accurate interpretation of these images, as described in Chapter \ref{AbsorptionImagingChapter}. Using the simulation-corrected images, we extracted the fraction of atoms scattered in our collision experiment.

\section{Experimental procedure}
We prepared our degenerate \K{} clouds as described in section \ref{sec:DFGsequence}. After this preparation, we used adiabatic rapid passabe (ARP) to transfer the degenerate cloud of \K{} atoms in the $\ket{F=9/2,m_F=9/2}$ state into the $\ket{F=9/2,m_F=-9/2}$ state by using a 3.3 MHz rf field and sweeping the bias magnetic field from -0.518 mT to -0.601 mT in 150 ms.
%We used a Zeeman slower to slow both species before capturing in a magneto-optical trap (MOT). After 7 s seconds of MOT loading \K{} followed by 1.5 s of loading both \K{} and \Rb{}, we cooled both species in optical molasses for 2 ms. We optically pumped both species into their maximally stretched magnetically trappable states, $\ket{F=9/2, m_F=9/2}$ for \K{} and $\ket{F=2,m_F=2}$ for \Rb{}. Both species were then loaded into a quadrupole magnetic trap with a $\approx$ 7.68 mT/cm gradient along \ez, and cooled evaporatively via forced RF evaporation, sweeping the RF frequency from 18 MHz to 2 MHz in 10 s. The magnetic trap was plugged by a $\lambda =$ 532 nm beam, tightly focused to $\approx$ 30 \um{} and $\approx$ 5 W in power, providing a repulsive potential around the zero field point to prevent Majorana losses. Since the \K{} atoms were spin polarized and therefore only interacted by the strongly suppressed \pwave{} interactions, they re-thermalized only due to sympathetic cooling with \Rb{} atoms.
%
%We then loaded the atoms into a crossed optical dipole trap, provided by a 1064 nm fiber laser, and continued evaporative cooling by slowly ramping down the dipole trap to trap frequencies of $(\omega_x,\omega_y,\omega_z)/2\pi =(39, 42, 124)$ Hz in the three spatial directions, while also turning off the quadrupole field. We then used adiabatic rapid passage (ARP) to transfer the \Rb{} atoms from the $\ket{F=2, m_F=2}$ state to the  $\ket{F=1, m_F=-1}$ absolute ground state via 6.8556 GHz microwave coupling (20.02 MHz from the zero field resonance) followed by a magnetic field sweep from -0.469 mT to -0.486 mT in 50 ms. This state was chosen to minimize spin changing collisions with \K{} atoms during any further evaporation \cite{BestThesis}.  We then briefly applied an on-resonant probe laser, ejecting any remaining \Rb{} atoms in the $F=2$ manifold from the trap. 


Following the state transfer, we had two versions of the protocol \--- one for approaching the Feshbach resonance from higher fields and one for approaching it from lower fields. For approaching the resonance from lower fields, we proceeded by ramping the bias magnetic field to 19.05 mT, turning on a 42.42 MHz RF field, and then sinusoidally modulating the bias field at 125 Hz for 0.5 s, with a 0.14 mT amplitude, decohering the \K{} state into an equal mixture of $\ket{F=9/2,m_F=-9/2}$ and $\ket{F=9/2,m_F=-7/2}$. For approaching the resonance from higher fields, the same was done at a bias field of 21.71 mT and an RF frequency of 112.3 MHz. The depolarization allowed the \K{} atoms to interact and re-thermalize, allowing us to further evaporate in the dipole trap \cite{DeMarco99}. Since \Rb{} is heavier than \K{}, we were able to evaporate the \K{} atoms past the point where \Rb{} atoms were no longer suspended against gravity and had been completely removed.  These hyperfine states of \K{} were then used to study their Feshbach resonance.


After evaporation, we ramped the bias field in a two-step fashion to the desired value $B$ near the Feshbach resonance. We approached the field using our quad coils in Helmholtz configuration (0.19 mT/A, see sec. \ref{sec:magneticCoils}) to bring the magnetic field to a setpoint 0.59 \mT{} away from $B$,  $B-0.59$ \mT{} when approaching from below and $B+0.59$  \mT{} from above. We held the atoms at this field for 100 $\rm{ms}$ to allow the eddy currents induced by the large quad coils to settle, and then used our lower inductance biasZ coils (0.017 mT/A, see sec. \ref{sec:magneticCoils}) to quickly change the field the remaining 0.59 \mT{}. This allowed us to study the resonance from both sides without the added losses associated with going through the resonance \cite{Chin10}.

Once at the intended bias field, we split the cloud into two spatially overlapping components with opposing momenta  and observed scattering as they moved through each other and separated. These counterpropagating components were created using an  8$E_{\rm{L}}$ deep near resonant ($\lambda_{\rm{L}}$=766.704 nm) 1-d retro-reflected optical lattice (see sec. \ref{sec:laserBeams}), where $E_{\rm{L}}=\hbar^2 k_{\rm{L}}^2/2m_{\rm{K}}$ is the lattice recoil energy and $\hbar k_{\rm{L}}=2\pi \hbar/ \lambda$ is the recoil momentum. We rapidly pulsed this lattice on and off with a double-pulse protocol \cite{Wu05, Edwards10}. The pulse sequence was optimized to transfer most of the atoms into the $\pm 2 \hbar k_{\rm{L}}$ momentum states. Since the initial Fermi gas had a wide momentum spread (in contrast to a BEC, which has a very narrow momentum spread), and the lattice pulsing is a momentum dependent process  \cite{Wu05}, not all the atoms were transferred into the target momentum states. We experimentally optimized our pulse times to minimize the atoms remaining in the zero momentum state. The optimized pulse times were 23 \us{} for the first pulse, 13 \us{} off interval, and 12 \us{} for the second pulse \cite{Edwards10}.

We then released the atoms from the trap and allowed 1 ms for the two opposite momentum states within the cloud to pass through each other, scattering on the way. For the data taken coming from below the Feshbach resonance, we then simply ramped down the field and imaged the atoms. For the data taken coming from above the Feshbach resonance, we ramped the field back up, retreating through the resonance if it had been crossed and thereby dissociating any molecules that were created, and then quickly ramped the field back down and imaged the atoms. We used a 40 \us{} imaging pulse with $I_0/I_{\rm{sat}}\approx 0.6$ at the center of the probe laser. The total time-of-flight was $t_{TOF}=6.8$ $\rm{ms}$.

The magnetic fields produced by the combination of our quad and biasZ coils in the regime of interest were independently calibrated by rf-spectroscopy. We prepared \K{} atoms in the $\ket{F=9/2, m_F=-9/2}$ state and illuminated them with and rf-field with some frequency $\nu_{rf}$. We then ramped our high-inductance coils to variable set points, followed by an adiabatic 250\us{} ramp of 2.84 mT in the lower inductance coils. We then used Stern-Gerlach and observed the fractional population in the $\ket{F=9/2, m_F=-9/2}$  and $\ket{F=9/2, m_F=-7/2}$ states as a function of the high-inductance coil current. We fit the fractional population curve to a Gaussian, and considered the center of the fit to be on-resonant, with an uncertainty given by the Gaussian width. We used the Breit-Rabi formula [ADD CITATION TO  THIS ONCE ITS PUT INTO THE HYPERFINE SECTION OF ATOM LIGHT CHAPTER] to determine the resonant field value at $\nu_{rf}$. We did this for 5 different rf frequencies, and acquired a field calibration with an uncertainty of 0.3 mT, which was included in the listed uncertainty on the center field of the Feshbach resonance.


\section{Data analysis}

We first processed each image by comparing the obsereved $OD$s to simulations taking into account the recoil induced detuning as described in Chapter \ref{AbsorptionImagingChapter}. An example of images before and after processing are shown in Fig. \ref{fig:SampleCorrection}.  To improve the signal and mitigate our shot to shot number fluctuations, we took 15 nominally identical images for each data point.
\begin{figure}
	\subfigure[]{\includegraphics{"Chapter3 Figures/figure10a".pdf}}
	\subfigure[]{\includegraphics{"Chapter3 Figures/figure10b".pdf}}
\caption[An sample absorption image after 6.8 ms TOF]{An example of our absorption image after 6.8 ms TOF. The 1-D lattice imparts momentum along \ex{}. The two large clouds on the left and right are the atoms in the $\pm 2 k_{\rm{L}}$ momentum orders that passed through each other unscattered. The smaller cloud in the center is the atoms that remained in the lowest band of the lattice after pulsing, and thus obtained no momentum. The thin spread of atoms around these clouds is the atoms that underwent scattering.   This image was taken coming from below the Feshbach resonance at 20.07  \mT{}. (a) Raw optical depth, (b) atomic column density obtained by comparing to simulated $OD$s, $\sigma_0 n^{\rm{sim}}$ }
\label{fig:SampleCorrection}
\end{figure}

We counted the fraction of atoms that experienced a single scattering event for each of the fifteen images at a given bias magnetic field. Single scattering events are easily identified, as two atoms that scatter elastically keep the same amplitude of momentum, but depart along an arbitrary direction. Therefore, an atom traveling at $2 \hbar k_{\rm{L}}$ to the right that collides elastically with an atom traveling at $-2 \hbar k_{\rm{L}}$ to the left will depart with equal and opposite momenta $2 \hbar k_{\rm{L}}$ at an arbitrary angle, and in a time-of-flight image such atoms will lie in a spherical shell, producing the scattering halo pictured in Fig. \ref{fig:halo}(a).
\begin{figure}
	\subfigure[]{\includegraphics[scale=0.22]{"Chapter3 Figures/Picture12".pdf}}
	\subfigure[]{\includegraphics[scale=0.9]{"Chapter3 Figures/figure12b".pdf}}
\caption[Experimental setup and inverse Abel transform]{(a) Our experimental setup. After time-of-flight, the two clouds traveling along $\pm \hat{e}_x$ directions have separated and the atoms that underwent a single scattering event were evenly distributed in a scattering halo around the unscattered clouds. The 1-D lattice defined the axis of cylindrical symmetry. (b) Inverse Abel transformed image. The atoms within the Fermi momentum $k_F$ of each unscattered cloud center are in the unscattered region and counted towards the total unscattered number. The atoms outside the radius $ k_{\rm{L}}-k_{\rm{F}}$ but inside $k_{\rm{L}}+k_{\rm{F}}$ while also being outside the unscattered region are counted towards the number of single scattered atoms.   }
\label{fig:halo}
\end{figure}

Absorption images captured the integrated column density along \ez{}, a projected 2D atomic distribution. To extract the radial dependence of the 3D distribution from the 2D image, we performed a standard inverse Abel transform. The inverse Abel transform assumes cylindrical symmetry, which was present in our case, with the axis of symmetry along \ex{}, defined by the lattice. We neglect the initial asymmetry of the trap, as during time-of-flight the atoms travel far beyond the initial extent of the cloud  $(r_x,r_y,r_z)\approx$ (45,48,15) \um{}, while the cloud width after TOF is $\approx$ 82 \um in each direction. We thus obtained the atomic distribution $\rho(r,\theta)$ as a function of $r$, the radial distance from the scattering center, and $\theta$, the angle between $r$ and symmetry axis \ex{}, integrated over $\phi$, the azimuthal angle around the $x$ axis.

We then extracted the number of scattered atoms $N_{\rm{scat}}$ as a fraction of the total atom number $N_{\rm{tot}}$ for each image, as shown in Fig. \ref{fig:halo}(b). The unscattered atom number was the number of atoms in the two unscattered clouds. The number of atoms that underwent a single scattering event was the number of atoms outside the Fermi radius of the unscattered clouds, but inside the arc created by rotating the Fermi momentum $k_{\rm{F}}$ around the original center of the cloud (red arcs in Fig. \ref{fig:halo}(b)). For both the scattered and unscattered numbers, we accounted for atoms that fell outside the field of view of our camera by multiplying the counted atom number by a factor of the total area as defined by the radii divided by the visible area on the camera. The atoms in the center region were not counted as they were originally in the zero momentum state and could not contribute to the scattering halo under study.

We fit the fraction of scattered atoms versus the total atom number for each of the 15 images taken at the same bias magnetic field to a line constrained to be zero at zero. The slope of this fit was taken to be the value of $N_{\rm{scat}}/N_{\rm{tot}}^2$ at that bias magnetic field, and the variance of the fit gave the uncertainty on that data point. This uncertainty reflected our shot to shot number fluctuations, which produced variable atomic densities and thus influence the scattered fraction. 

We then deduced the resonant field value $B_0$ and width of the resonance  $\Delta$, the parameters in Eq. (\ref{eqn:feshbach}).  Since we were in the low energy regime (the atomic momentum was much smaller than the momentum set by the van der Waals length $k_{\rm{L}}+k_{\rm{F}}\ll1/l_{\rm{vdW}}$, and we were well below the p-wave threshold temperature \cite{DeMarco99}), the scattering cross-section was given by $\sigma=4\pi a^2$.

The scattering cross-section $\sigma$ gives the probability $P_{\rm{scat}}=\sigma N/A$ that a single particle will scatter when incident on a cloud of atoms with a surface density of $N/A$, where $A$ is the cross-sectional area of the cloud and $N$ is the number of atoms in the cloud. In our case, each half of the initial cloud, with atoms number $N_{\rm{tot}}/2$, is incident on the other half. Thus, the number of expected scattering events is $N_{\rm{scat}}= (N_{\rm{tot}}/2) \sigma  (N_{\rm{tot}}/2)=\sigma N_{\rm{tot}}^2/4A$. Assuming $A$ is constant for all our data, we can define a fit parameter $b_0=4\pi a_{\rm{bg}}^2/4A$, where $a_{\rm{bg}}$ is the background scattering length. We can thus adapt Eq. (\ref{eqn:feshbach}) to obtain the fit function
\begin{equation}
\frac{N_{\rm{scat}}}{N_{\rm{tot}}^2}=b_0\left(1-\frac{\Delta}{B-B_0}\right)^2 + C.
\label{eqn:fitFeshbach}
\end{equation}
We found that our imaging noise skewed towards the positive, giving rise to a small background offset. We accounted for this in our fit by including a constant offset parameter $C$.


\section{Results}
Our final data is presented in Fig. \ref{fig:fittedFractions}. The red curve depicts a best fit of the model given in Eq. (\ref{eqn:fitFeshbach}). The fit parameters we extracted were $\Delta = 1(5)$  \mT{}, $B_0 = 20.206(15)$  \mT{}, $b_0 = 5(3)\times 10^{-3}$ and $C=8(1)\times 10^{-4}$. To obtain the fit, we used data taken by approaching the resonance from above for points above where we expected the resonance to be and data taken approaching the resonance from below for points below. We also excluded from the fit data points very near the resonance, as there the assumption $\sigma\rho\ll1$, where $\rho$ is the atom number per unit area, is no longer valid and the problem must be treated hydrodynamically.

The accepted values for the $^{40}K$ s-wave Feshbach resonance for the  $\ket{9/2,-9/2}$ and $\ket{9/2,-7/2}$ states are $B_0=20.210(7)$  \mT{} and $\Delta=0.78(6)$  \mT{} \cite{Regal04}, which is in good agreement with our findings. Some potential sources of systematic uncertainty that we did not account for include scattering with atoms that did not receive a momentum kick from the lattice pulsing or the impact of multiple scattering events.
\begin{figure}
	\includegraphics{"Chapter3 Figures/figure11".pdf}
\caption[Normalized scattered population vs bias field $B$]{Normalized scattered population plotted versus bias field $B$. Green dots represent data taken coming from below the resonance, and blue dots represent the data taken coming from above the resonance. The red curve depicts the best fit, where data coming from above the resonance was used above the resonance and data coming from below the resonance was used below the resonance to create the fit; the unused data points are indicated by hollow dots. The regime where the scattering length is likely large enough for the atoms to behave hydrodynamically is shaded in gray, and data points in that area were also excluded from the fit. Resonant field value $B_0$ as found in this work and our systematic uncertainty in the bias magnetic field $\delta B_0$ are indicated.    }
\label{fig:fittedFractions}
\end{figure}
%\section{Conclusion}
%We studied the effects of recoil-induced detuning effects on absorption images and found an optimal imaging time of $\approx40$ \us{} for \K{} atoms for noise minimization after corrections. We use these results to observe s-wave scattering halos of the Fermi gas around the $\approx 20.2$ \mT{} Feshbach resonance and directly verified the resonance location and width. Our analysis can be used in any absorption imaging application where SNR optimization is critical.

