%Acknowledgments

\renewcommand{\baselinestretch}{2}
\small\normalsize
\hbox{\ }
 
\vspace{-.65in}

\begin{center}
\large{Acknowledgments} 
\end{center} 

\vspace{1ex}

This thesis describes work that was carried out by many people, in a way that the multitude of authors on each of the papers only begins to convey. However, as this is the acknowledgement section of {\it my} thesis, I will endeavor to thank all the people that guided and supported me during my admittedly long tenure as a graduate student. 

I would first and foremost like to thank my adviser, Dr. Ian Spielman, for taking a chance on a lab-averse recovering theorist. Ian's mentorship style was always patient and supportive, but without compromising on standards. Ian would never ask \lq{why isn't this done?}\rq{}, but instead ask \lq{how are things going this fine morning?}\rq{} and offer insight and advice regarding whatever I happened to be working on. He was a role model to aspire to, both in his unique combination of intuition and rigor in physics and his unparalleled coffee consumption.

The first \lq{trial}\rq{} semester of my lab work was spent in the RbLi lab at UMD. During that time, I am grateful to have learned from Dan Campbell, who voluntarily took on teaching me about the BEC production process one step every day, and Ryan Price, whose magical skills with all equipment and dark humor got me through that time. When I first joined the NIST lab, I had the great fortune of learning from a team of outstanding postdocs---the \lq{old guard}\rq{}---consisting of Lindsey LeBlanc, Ross Williams, and Matthew Beeler. I thank them for an excellent introduction to not only laboratory techniques and the physics under study but also the art of working as a team. 

An unparalleled debt of gratitude goes to Lauren Aycock, who shared the lab with me from day one and taught me most of what I know being an experimentalist. She endured my clumsiness and obstinance and continued teaching and supporting me throughout my time at NIST. Beyond the lab, she helped teach me how to drive and pushed me to do my first half-marathon. I cannot imagine what my graduate career would have been like without her. I would also like to thank post-docs Ben Stuhl and Hsin-I Lu for the role they played in my physics training. Ben's enthusiasm for electronic design was infectious and it was an honor to learn from him, be it about about feedback loops or version control. Hsin-I's unparalleled productivity in her own work didn't stop her from taking lots of time to help me in mine. Her wisdom for knowing what the root of a problem is and how to prioritize tasks was invaluable, and she remains a role model for me today.

I am happy to have had the opportunity to work with several visiting German students: Marcell Gall, Max Schemmer, and Martin Link. Marcell excelled in balancing hard, productive work in the lab with fun in his free time, and was generally a joy to be around.  Max jumped into the research with a running start, and it was exciting to see what he accomplished in a short amount of time. Martin (Marcell III) forced me to give up my belief that all crossfitters are annoying. It was a pleasure to work with him as well as to exercise and hang out with him. 

My later years in the lab were spent with post-doc Mingwu Lu and fellow grad student Alina Escalera. Mingwu has taught me so much about physics, electronics, western philosophy and wine. I am grateful for having the opportunity to learn from him and get to know him. It has been a pleasure to watch Alina grow in the lab, and to have her understanding support during the rougher parts of paper and thesis writing. Her tenacity and drive are to be admired. I have had few opportunities to spend time with new lab recruits, post-doc Amilson Fritsch and graduate student Graham Reid, but even from our limited interactions I am confident the lab will be left in good hands. 

During the writing of this thesis, I have had amazing help and support from my roommates, fellow students and friends. I would like to thank Ana Valdes-Curiel, Paco Salces-Carcoba, Chris Billington, Kristen Voigt and Jon Hoffman for bravely reading through entire chapters of my thesis when I couldn't bare to even look at it. I am extremely grateful to have been part of the amazing living community that is the international house of physicists (IHOP)---it has felt more like a large family than a housing arrangement. I would like to thank both the current IHOPers, Ana, Paco, Chris and Daniel, and notable past IHOPers, Dimitry Trypogeorgos and Rory Speirs. A special thank you to Ana and Paco for unpromptedly feeding me during the final thesis writing days and to them and Chris for continued moral support. Extra special thank you to Kristen, my oldest graduate school friend, for teaching me how to drive, housing me in my times of need, providing me with thesis writing space and support, feeding me uncountable times, and telling me its okay to take a nap. 

I would also like to thank Jenna Beckwith for her help at the beginning of my graduate journey, Taylor Cole for her insightful support in the last year that has allowed me to work on my thesis almost without panic,  and Dr. Diane Stabler for her constant presence and support throughout my time here and for offering deceptively simple advice. I would also like to thank the good people at Eli Lily and Company without whose pioneering research my work would not have been possible. 

Above all I would like to thank my family. If it wasn't for my parents' upbringing, graduate school would never have been a possibility for me. I am grateful to them for teaching me all that they have, supporting me throughout my time as a graduate student, but prioritizing my happiness above graduate success. I would like to thank them, my sister Alya, and my brother Pten, for everything. 