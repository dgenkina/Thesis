%Abstract Page

\hbox{\ }

\renewcommand{\baselinestretch}{1}
\small \normalsize

\begin{center}
\large{{ABSTRACT}}

\vspace{3em}

\end{center}
\hspace{-.15in}
\begin{tabular}{ll}
Title of dissertation:    & {\large  MEASURING TOPOLOGY OF BECS}\\
&                     {\large IN A SYNTHETIC DIMENSIONAL LATTICE} \\
\ \\
&                          {\large  Dina Genkina} \\
&                           {\large Doctor of Philosophy, 2019} \\
\ \\
Dissertation directed by: & {\large  Professor Ian Spielman} \\
&  				{\large	 Joint Quantum Institute,} \\
&				{\large National Institute of Standards and Technology,}\\
&				{\large and} \\
&				{\large Department of Physics, University of Maryland}
\end{tabular}

\vspace{3em}

\renewcommand{\baselinestretch}{2}
\large \normalsize

We describe several experiments performed on a two species apparatus capable of producing Bose-Einstein condensates (BECs) of \Rb{} and degenerate Fermi gases (DFGs) of \K{}. 

We first describe computational results for observed optical depths with absorption imaging, in a regime where imaging times are long enough that recoil-induced detuning introduces significant corrections. We report that the obseved optical depth depends negligibly on the cloud shape. We also find that the signal-to-noise(SNR) ratio for low atom numbers can be significantly improved by entering this regime and applying the appropriate corrections. We take advantage of this SNR improvement in our subsequent experiment colliding two clouds of \K{} for different values of background magnetic field in the vicinity of a Feshbach resonance. We directly imaged the fraction of scattered atoms, which was low and difficult to detect. We used this method to measure the resonance location to be $B_0 = 202.06(15)$ Gauss with width $\Delta = 10.(5)$ Gauss, in good agreement with accepted values.

Next, we describe experiments creating an elongated effectively 2D lattice for a BEC of \Rb{} with non-trivial topological structure using the technique of synthetic dimensions.  We set up the lattice by applying a 1D optical lattice to the atoms along one direction, and treating the internal spin states of the atoms as lattice sites in the other direction. This synthetic direction is therefore very short, creating a strip geometry. We then induce tunneling along the synthetic direction via Raman coupling, adding a phase term to the tunneling coefficient. This creates an effective magnetic flux through each lattice plaquette, in the Hofstadter regime, where the flux is of order the flux quantum $h/e$. We detect the resulting eigenstate structure, and observe chiral currents when atom are loaded into the central synthetic site. We further launch analogues of edge magnetoplasmons and image the resulting skipping orbits along each edge of the strip.

We then applied a force along the real dimension of the 2D lattice and directly imaged the resulting motion in the transverse, synthetic, direction. We performed these measurements with 3 and 5-site width lattices along the synthetic direction. We used these measurements to identify the value of the Chern number, the topological invariant in 2D, by leveraging the Diophantine equation derived by Thouless, Kohomoto, Nightingale, and den Nijs. We measure Chern numbers with typical uncertainty of 5\%, and show that although band topology is only properly defined in infinite systems, its signatures are striking even in extremely narrow systems.
