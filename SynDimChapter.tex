\renewcommand{\thechapter}{6}

\chapter{Synthetic Magnetic Fields in Synthetic Dimensions}\label{chap:SynDim}


In condensed matter, 2-D systems in high fields have proved to be of great technological use and scientific interest. The integer quantum Hall effect (IQHE)\cite{Klitzing1980}, with its quantized Hall resistance, has given rise to an ultra-precise standard for resistivity. It was also one of the first examples of topology playing an important role in physics---the precise quantization of the Hall conductance is guaranteed by the non-trivial topology of the system\cite{Thouless1982}. This quantizes the magnetic flux into flux quanta of $\Phi_0=h/e$, where $e$ is the electron charge, and leads to a new \lq{plateau}\rq{} in the resistivity when an additional quantum of flux is threaded through the system. 

In the IQHE system, the underlying lattice structure of metal is effectively washed out---the magnetic flux per individual lattice plaquette is negligible. However, new physics arises when the magnetic flux per plaquette is increased to some non-negligible fraction of the flux quantum, giving rise to the Hofstadter butterfly\cite{Hofstadter1976}. These regimes are hard to reach experimentally, since the typical plaquette size in crystalline material is of order a square Angstrom, and the magnetic field necessary to thread a magnetic flux of $\Phi_0$ through such a narrow area is of order $\approx10^4$ Tesla, not access able with current technology. 

Several platforms have however reached the Hofstadter regime by engineering systems with large effective plaquette size, in engineered materials\cite{Geisler2004,Hunt2013}, and in atomic\cite{Jaksch2003,Aidelsburger2013,Miyake2013,Jotzu2014,Aidelsburger2014,Mancini2015} and optical\cite{Hafezi2013} settings. Here, we use the approach of synthetic dimensions \cite{Celi2014} in atomic BEC of \Rb{} to reach the Hofstadter regime. We demonstrate the non-trivial topology of the system created, and use it to image skipping orbits at the edge of the 2-D system---a hallmark of 2-D electron systems in a semi-classical treatment. 

We first describe the experimental setup of the effective 2D lattice with a large effective magnetic flux through it created with synthetic dimensions in sec. \ref{sec:SynDimSetup}. We write down the Hamiltonian of the system, calculate the band structure and discuss its basic features in sec. \ref{sec:SynDimHamiltonian}. We then describe the results of our experiments. First, we describe the measured eigenstates of the system in sec. \ref{sec:SynDimEigenstates}. Second, we detail the measurement of chiral edge motion when atoms are loaded into the central site of the lattice along the synthetic direction. Finally, we describe our observation of skipping orbits along the edges of the system, when atoms are loaded into the edge sites along the synthetic direction. This measurement represents the first direct observation of the phenomenon of skipping orbits. The work described in this chapter was published in\cite{Stuhl2015}.

\section{Synthetic dimensions setup}\label{sec:SynDimSetup}

Any internal degree of freedom can be thought of as a synthetic dimension---the different internal states can be treated as sites along this synthetic direction. As long as there is some sense of distance along this direction, i.e. some of the internal states are \lq{nearest neighbors}\rq{} while others are not, it is sensible to consider it is a dimension. In our case, an effective 2-D lattice is formed by sites formed by a 1-D optical lattice along a \lq{real}\rq{} direction, here  \ex{}, and the atom's spin states forming sites along a \lq{synthetic}\rq direction,  $\ess{}$. 

The experimental setup for this system is schematically represented in Fig. \ref{fig:synDimSchematic}a. The BEC is subject to a 1-D optical lattice, formed by a retro-reflected beam of $\lambda_L=1064$ nm along  \ex{}. A bias magnetic field $B_0$ along  \ez{} separates the different spin states. The spin states can be thought of a sites along a synthetic dimension even without any coupling field. However, only once a coupling field is present do they acquire a sense of distance. We couple them via rf or Raman coupling, which only couples adjacent spin states. The Raman beams illuminating the atoms are along the same  \ex{} direction as the 1-D optical lattice. The rf field has components both along the  \ex{} and \ey{}. 


\begin{figure}
	\includegraphics{"SynDim Figures/Fig1_Schematic".pdf}
\caption[Setup of effective 2-D lattice]{Setup of effective 2-D lattice. (a) Beam geometry. The BEC is subject to a bias magnetic field $B_0$ in the \ez{} direction. The 1-D lattice beam and Raman beams are both along the \ex{} direction, and the rf field can be applied with projections onto both \ex{} and \ey{}. (b) Schematic of the effective 2-D lattice. Sites along \ex{} are formed by the 1-D optical lattice and labeled by site number $j$. Sites along the synthetic direction  $\ess{}$ are formed by the spin states: 3 sites for atoms in the $F=1$ manifold and 5 sites for atoms in $F=2$. These sites are labeled by $m$. Raman transitions induce a phase shift, which can be gauge transformed into a tunneling phase along the  \ex{} direction. This leads to a net phase when hopping around a single lattice plaquette of $\phi_{AB}$.  }
\label{fig:synDimSchematic}
\end{figure}


Figure \ref{fig:synDimSchematic}b sketches out the effective 2-D lattice created. Here, we have labeled the lattice sites along the \lq{real}\rq direction  \ex{} by site index $j$. In the tight binding approximation, we can describe a lattice hopping between adjacent sites with tunneling amplitude $t_x$. Similarly, the sites along the \lq{synthetic}\rq dimension are labeled by site index $m$ (identical to spin projection quantum number $m_F$), and the rf or Raman coupling here plays the role of a tunneling amplitude $t_s$. In the case of rf coupling, there is no momentum kick associated with spin exchange, and both $t_x$ and $t_s$ are real. 

In the case of Raman coupling, however, there is a momentum kick of $2k_R$ associated with every spin transfer, and therefore a phase factor of ${\rm exp}(2i k_R x)$ with every spin \lq{tunneling}\rq{} event. Since position $x$ is set by the 1-D lattice, $x_j = j \lambda_L/2 = j \pi/k_L$, and the space dependent phase factor is ${\rm exp}(2\pi i k_R/k_L j)$. An absolute phase change in the wavefunction is not meaningful. However, a phase acquired when going around a plaquette and coming back to the same place is meaningful, as one could imagine one atom staying at the same site and the other going around a plaquette and coming back to detect the phase difference. In this setup, the phases acquired while going around a single plaquette are, starting at some lattice site $\ket{j,m}$, are: $0$ (for tunneling right to $\ket{j+1,m}$), $2\pi i k_R/k_L (j+1)$ (for tunneling up to $\ket{j+1,m+1}$, $0$ (for tunneling left to $\ket{j,m+1}$) and $-2\pi i k_R/k_L j$ (for tunneling back down to $\ket{j,m}$). The total phase acquired is thus $\phi_{\rm AB} = 2\pi k_R/k_L$, independent of the starting lattice site. Since the absolute phase does not matter and only the value of $\phi_{\rm AB}$, we can perform a phase transformation that shifts the tunneling phase onto the spatial direction, defining $t_x = |t_x|{\rm exp}(i\phi_{\rm AB}m)$ and $t_s=|t_s|$, as labeled in Figure \ref{fig:synDimSchematic}b. 

To see how this phase implies an effective magnetic field, we draw an analogy to the Aharonov-Bohm effect\cite{Aharonov1959, Aharonov1992} from quantum mechanics. This effect considers an infinite solenoid with an electric current running through it. The magnetic field $B$ in this setup exists only inside the solenoid, while the magnetic vector potential persists outside the solenoid.  However, if two electrons are sent on a trajectory around the solenoid, even though they never pass through any magnetic field, they nevertheless acquire a relative phase that can be detected by interfering them with each other. This relative phase is given by $\phi_{\rm AB} =2 \pi \Phi/\Phi_0$, where $\Phi = B*A$ is the magnetic flux through the solenoid ($A$ is the area inside the solenoid, pierced by the magnetic field) and $\Phi_0=h/e$ is the flux quantum, with $e$ the electron charge. Since in our system, the atoms acquire a phase when they perform a closed loop around a single lattice plaquette. Therefore, they behave as though there was an infinite solenoid piercing each plaquette with a magnetic field going through it, and the flux per plaquette in units of the flux quantum is $\Phi/\Phi_0=\phi_{\rm AB}/2\pi =  k_R/k_L$. For the case of rf coupling, the phase acquired at every transition is $0$ and the flux $\Phi/\Phi_0=0$. In this way, we are able to engineer large fluxes per individual plaquette simply by adjusting the ratio $k_R/k_L$, allowing us to sidestep the need for experimentally inaccessible field strengths and reach the Hofstadter regime. 

\section{Hamiltonian of the effective 2-D system}

\subsection{Hamiltonian}\label{sec:SynDimHamiltonian}
The full Hamiltonian of this system, without making the tight binding approximation, can be written down by combining the lattice Hamiltonian (eqn. \ref{eqn:LatHam}) and the rf (eqn. \ref{eqn:rfHam}) or Raman Hamiltonian (eqn. \ref{eqn:RamanHam}). To do this, we write a new basis that encompasses both the momentum and the spin degrees of freedom. For the lattice Hamiltonian, we used the momentum basis
\begin{equation}
 \begin{pmatrix} \vdots \\
\ket{q+4 k_L}\\
\ket{q+2 k_L}\\
\ket{q}\\
\ket{q -2 k_L}\\
\ket{q - 4 k_L}\\
\vdots
\end{pmatrix} \\.
\end{equation}
For the Raman Hamiltonian in the $F=1$ manifold, we used the spin and momentum basis
\begin{equation}
\begin{pmatrix}
\ket{k_x-2k_R,-1}\\
 \ket{k_x,0}\\
\ket{k_x+2k_R,1}
\end{pmatrix}\\.
\end{equation}

In a lattice, the momentum $k_x$ becomes crystal momentum $q$. For every state in the lattice basis, we now expand to three states, one for each spin state, with the appropriate momentum shifts, giving
\begin{equation}
 \begin{pmatrix} \vdots \\
\ket{q+2k_L-2k_R,-1}\\
 \ket{q+2k_L,0}\\
\ket{q+2k_L+2k_R,1}\\
\ket{q-2k_R,-1}\\
 \ket{q,0}\\
\ket{q+2k_R,1}\\
\ket{q-2k_L-2k_R,-1}\\
 \ket{q-2k_L,0}\\
\ket{q-2k_L+2k_R,1}\\
\vdots
\end{pmatrix} \\.
\end{equation}

In this basis, we combine the lattice and Raman Hamiltonians (omitting the kinetic energy in the other two directions) in an infinite block matrix form as 
\begin{equation}
H =
 \begin{pmatrix} \ddots &  & & & \\ 
 &{\bf H_R}(2k_L)  & {\bf \frac{V_0}{4}} &{\bf 0} &  \\
 &  {\bf \frac{V_0}{4}} &{\bf H_R}(0) & {\bf \frac{V_0}{4}}&  \\
 & {\bf 0} &  {\bf \frac{V_0}{4}} & {\bf H_R}(-2k_L)  &  \\
 & & & &  \ddots \end{pmatrix} \\,
\label{eqn:SynDimHam}
\end{equation}
where ${\bf H_R}(x)$ is the Raman Hamiltonian with a momentum shift of $x$:
 \begin{equation}
{\bf H_R}(n 2k_L) = 
 \begin{pmatrix} \frac{\hbar^2(q + n 2k_L -2k_R)^2}{2m}+\hbar\delta & \hbar\Omega/2  &  0  \\ 
\hbar\Omega/2 & \frac{\hbar^2 (q+n 2k_L)^2}{2m} - \hbar\epsilon &  \hbar\Omega/2\\
 0 & \hbar\Omega/2 &  \frac{\hbar^2(q+n 2k_L+2k_R)^2}{2m} -\hbar\delta  \\
 \end{pmatrix} \\,
\end{equation}
the matrix ${\bf \frac{V_0}{4}}$ is a 3x3 diagonal matrix lattice coupling strength $\frac{V_0}{4}$ on the diagonal, and $\bf 0$ is a 3x3 matrix of zeros. This extends in both directions with ${\bf H_R}(2nk_L)$ on the diagonal blocks and ${\bf \frac{V_0}{4}}$ as the first off-diagonal blocks and $\bf 0$ everywhere else. 

This Hamiltonian is easily extended to higher $F$ values by replacing the Raman blocks ${\bf H_R}(x)$ with the corresponding Raman coupling Hamiltonian from eqn. \ref{eqn:RamanAllF}, and extending the diagonal matrix  ${\bf \frac{V_0}{4}}$ and the zero matrix $\bf 0$ to be ($2F+1$)x($2F+1$).

For computational convenience, we convert to lattice recoil units, $E_L=\hbar^2 k_L^2/2m$, $k_L=2\pi/\lambda_L$. Then the diagonal blocks become
 \begin{equation}
{\bf H_R}(n)/E_L = 
 \begin{pmatrix} (q +2 n -2\phi_{\rm AB}/2\pi)^2+\hbar\delta & \hbar\Omega/2  &  0  \\ 
\hbar\Omega/2 & (q+2 n)^2 - \hbar\epsilon &  \hbar\Omega/2\\
 0 & \hbar\Omega/2 & (q + 2 n + 2 \phi_{\rm AB}/2\pi)^2 -\hbar\delta  \\
 \end{pmatrix} \\,
\end{equation}
where $\hbar\delta$, $\hbar\Omega$ and $\hbar\epsilon$ are now written in units of $E_L$, $q$ is written in units of $k_L$ and we have used the fact that $\phi_{\rm AB}/2\pi = k_R/k_L$. The off-diagonal blocks ${\bf \frac{V_0}{4}}$ will be the same 3x3 diagonal matrices, with $\frac{V_0}{4}$ in units of $E_L$. 

This Hamiltonian can be written for general values of $F$ in the presence of Raman coupling and a 1-D optical lattice as
\begin{equation*}
H = \sum _{m=-F,n=-\infty}^{F,\infty}H_{0}+H_{\rm{R}}+H_{\rm{L}},
\label{eqn:SynDimHamiltonian}
\end{equation*}
where the diagonal term
\begin{eqnarray*}
H_{0} =&\\
&\left(\hbar^2\left(q-2m\Phi/\Phi_0-2n\right)^2 k_L^2/2m+\hbar\delta m+\hbar\epsilon m^2 \right)\\
&\ket{q+n2k_L,m}\bra{q+n2k_L,m}
\end{eqnarray*}
includes the kinetic energy as well as the two-photon Raman detuning from resonance $\delta$ and the quadratic Zeeman shift $\epsilon$. The second term represents the Raman coupling  with coupling strength $\hbar\Omega$, with anisotropic tunneling arising from the spin-dependent prefactor (Clebsch-Gordan coefficient):
\begin{eqnarray*}
H_{\rm{R}} =&\\
&\hbar\Omega\sqrt{F(F+1)-m(m+1)}/2\sqrt{2}\ket{q+n2k_L,m}\bra{q+n2k_L,m+1}\\
&+\rm{H.c.}
\end{eqnarray*}
Here,  H.c. stands for Hermitian conjugate. The third term represents lattice coupling to higher order lattice states, with lattice depth $V_0$: 
\begin{equation*}
H_{\rm{L}} = V_0/4\ket{q+n2k_L,m}\bra{q+(n+1)2k_L,m} +\rm{H.c.}
\end{equation*}

\subsection{Band structure}\label{sec:SynDimBandStructure}
\begin{figure}
	\includegraphics{"SynDim Figures/SynDimBandStructure".pdf}
\caption[Band structure of the synthetic dimensions Hamiltonian]{Band structure of the synthetic dimensions Hamiltonian, eqn. \ref{eqn:SynDimHam}. For all panels, the detuning $\hbar\delta=0$ and the quadratic shift $\hbar\epsilon=0.02 E_L$. (a) $F=1$, $\hbar\Omega=0.0$. The color represents $\langle m\rangle$, magnetization along $\ess{}$. (b) $F=1$, $\hbar\Omega=0.5$. (c) $F=2$, $\hbar\Omega=0.0$. (d) $F=2$, $\hbar\Omega=0.5$. }
\label{fig:SynDimBandStruct}
\end{figure}

The band structure of this Hamiltonian is presented in Figure \ref{fig:SynDimBandStruct}. Here, we have restricted ourselves to the lowest lattice band. We can do this because the energy splitting between the lowest and second lowest lattice band is of order $4 E_L$ (see Figure \ref{fig:latticeBandStructure}), while the width of the lowest band, given by the amplitude of the approximate sinusoid, is of order $0.3 E_L$ for lattice depths around $5.0 E_L$, relevant to our experiment. As long as the Raman coupling stays small compared to the lattice band spacing, the higher lattice bands are energetically separated enough that they can be ignored. 

Therefore, we can think of the Raman coupling analogously to the free space Raman coupling (see section \ref{sec:Raman}), except instead of free space parabolas each spin state gets a lowest lattice band sinusoid.   Figure \ref{fig:SynDimBandStruct}a shows this in the limit of no Raman coupling, $\Omega=0$, but with the 1D lattice on at $V_0=4.0 E_L$. The quadratic Zeeman shift is $\hbar\epsilon=0.02 E_L$ and the detuning $\delta=0$. The $m_F=-1$ sinusoid is shifted $2k_R$, similarly to section \ref{sec:Raman}, but since the sinusoid is periodic with $2k_L$, it folds into the first Brillouin zone of the lattice, such that the nearest minimum to $q=0$ is at $q = 2k_R-2k_L = (2\phi_{\rm AB}/2\pi - 2)k_L$. The edges of the Brillouin zone are marked by horizontal lines. The color indicates magnetization $\langle m\rangle=\sum_{m_F}m_F n_{m_F}$, where $n_{m_F}$ is the fractional population in the $m_F$ state. In synthetic dimensions language, $\langle m \rangle$ is the expectation value of position along $\ess{}$.

In Figure \ref{fig:SynDimBandStruct}b, we have restricted ourselves to the first Brillouin zone and turned the Raman coupling to $\hbar\Omega = 0.5 E_L$. This results in avoided crossings in Figure \ref{fig:SynDimBandStruct}a, and the lowest band now has a spin dependence on crystal momentum. Figure \ref{fig:SynDimBandStruct}c-d shows the same progression for the $F=2$ manifold. Figure \ref{fig:SynDimBandStruct}c is taken in the limit of $\hbar\Omega=0$. All of the 5 spin states get \lq{folded}\rq{} back into the first Brillouin zone due to the lattice periodicity of the bands. The different overall energies of the sinusoids are due to the quadratic Zeeman shift $\hbar\epsilon=0.02 E_L$. The lattice depth is again $V_0=5.0 E_L$ and detuning $\hbar\delta=0$. In Figure \ref{fig:SynDimBandStruct}d we have restricted ourselves to the first Brillouin zone and turned on the Raman coupling to $\hbar\Omega=0.5 E_L$.  Note that the inverted hyperfine structure in Fig. \ref{fig:SynDimBandStruct}c (meaning that the quadratic shift pushed the $m_F=0$ state up rather than down in energy compared to the others), combined with the Raman coupling serves to make the lowest band in the $F=2$ manifold close to flat. This makes the band more similar to a quantum Hall Landau level, and also shows promise for potential simulation of fractional quantum Hall physics, which require bands to be very flat. 


\subsection{Calibration}\label{sec:SynDimCalibration}

To calibrate the lattice depth $V_0$ in the synthetic dimensions system, we can simply calibrate the lattice depth without Raman or rf coupling as described in Section \ref{sec:LatticeCalib}. However, we are operating at very low Raman coupling strengths, $\hbar\Omega\approx0.5 E_L$. This is necessary because in the synthetic dimensional system the Raman coupling plays the role of tunneling, which has to be small, $t_s\approx0.1 E_L$, to approximate the tight binding limit. At these low Raman couplings, simple pulsing as described in Section \ref{sec:RamanCalib} is not useful for calibration, as the contrast of the Rabi oscillations would be too low to resolve. Therefore, we calibrate the Raman coupling and detuning with the full synthetic dimensions system. The \lq{folding in}\rq{} effect of the lattice (meaning, the folding in of the sinusoids into the first Brillouin zone, or Umklapp processes) makes the higher Raman bands much closer energetically than without the lattice, leading to larger contrast and allowing for accurate calibration.  

\begin{figure}
	\includegraphics{"SynDim Figures/SynDimPulsing".pdf}
\caption[Calibration of synthetic dimensions lattice]{Calibration of synthetic dimensions lattice. (a) Ramping procedure. The blue line represents the 1-D lattice depth as a function of time and the red line represents Raman coupling as a function of time. Both are held on for a variable amount of time $t$, producing Rabi oscillations. (b) Example of fractional populations in different $m$ states as a function of time $t$ in the $F=1$ manifold. Dots indicate data and lines indicate the best fit to theory, with parameters $\hbar\Omega=0.56\pm0.01 E_L$ and $\hbar\delta = 0.029 \pm 0.002 E_L$. (c) Example time-of-flight image in the $F=1$ manifold. A Stern-Gerlach gradient pulse separates different $m$ states along the horizontal axis, while the lattice and Raman beams give momentum along the vertical axis. (d)  Example of fractional populations in different $m$ states as a function of time $t$ in the $F=2$ manifold. Dots indicate data and lines indicate the best fit to theory, with parameters $\hbar\Omega=0.61\pm0.002 E_L$ and $\hbar\delta = 0.002 \pm 0.001 E_L$. (e) Example time-of-flight image in the $F=2$ manifold. A Stern-Gerlach gradient pulse separates different $m$ states along the horizontal axis, while the lattice and Raman beams give momentum along the vertical axis.  }
\label{fig:SynDimPulsing}
\end{figure}

To do this, we must first adiabatically load the lowest 1-D lattice band. To do that, we must ramp on the lattice potential on a time scale slow compared to the band spacing, $\approx 4 E_L$. This gives $t\approx h/4E_L=0.12$ ms. Figure \ref{fig:SynDimPulsing}a shows the full ramping scheme. We ramp the lattice on in $\approx 20$ ms. Then, we must pulse on the Raman coupling on a time scale fast compared to the spin sub-band level spacing to produce Rabi oscillations, but still adiabatic with respect to the lattice spacing to avoid exciting to the higher lattice band. We ramp the Raman beams on in $300$ $\mu$s. Then, the system is held on for a variable amount of time before all light is snapped off and the atoms are allowed to expand in time-of-flight. For the case of $F=2$ atoms, the transfer to the $F=2$ manifold is done in the 1-D lattice before the Raman beams are ramped on to minimize the time spend in the $F=2$ manifold. 

Figure \ref{fig:SynDimPulsing}c,d shows sample time-of-flight images during the calibration procedure for $F=1$ and $F=2$ respectively. The vertical axis is \ex{}, aligned with the lattice and Raman beams. Since the atoms have expanded in time-of-flight, this axis corresponds to the momentum $k_x$. The horizontal axis of the image is the axis along which a Stern-Gerlach magnetic field gradient, separating the different spin states, is applied. Therefore, this axis is the position $m$ along the synthetic dimension $\ess{}$. In the effectively 2-D synthetic dimensions lattice language, this is a \lq{hybrid}\rq{} imaging technique, imaging momentum along one lattice direction and position along the other.  

Figure \ref{fig:SynDimPulsing}c labels some notable momentum orders. The central order is at $k_x=0$, where the atoms start before the experiment. Two higher lattice orders, at $k_x=\pm2k_L$, are populated for the same spin $m=0$. $k_x=\pm2k_R$ is labeled, but not visibly populated, to indicate where the orders would appear of only Raman coupling was present with a higher coupling strength. Due to the 'folding in' effect of the lattice, the brightest orders of the $m=\pm 1$ states appear at $k_x=\pm (2k_L-2k_R)$. The $F=2$ states follow the same pattern, not labeled in Figure \ref{fig:SynDimPulsing}e as there are too many orders.  

For each value of the time $t$ we sum up the total optical depth in all of the orders of each spin state to obtain fractional populations for each spin state as a function of time. An example scan in the $F=1$ manifold is shown in Figure \ref{fig:SynDimPulsing} b. The colored dots represent the data for different spin states, and the lines represent the best fit to theory. Here, the significant detuning makes populations in the $m=\pm 1$ states unequal. An example scan in the $F=2$ manifold is shown in Figure \ref{fig:SynDimPulsing}e. Here, the detuning is small and states with opposite spin oscillate in approximate unison. This technique allows us to calibrate our experimental parameters.

\subsection{Tight binding approximation}

The synthetic dimensions Hamiltonian can be approximated in the tight binding limit as:
\begin{equation}
H=-\sum_{j,m} |t_x| e^{i\phi_{\rm AB}m}\ket{j+1,m}\bra{j,m}+t_s(m)|j,m+1\rangle\langle j,m| + A_m|j,m\rangle\langle j,m| + h.c.,
\label{eqn:TBham}
\end{equation}
where $j$ and $m$ label sites along \ex{} and $\ess{}$ respectively, as shown in Figure \ref{fig:synDimSchematic}b. $t_s=|t_s|$ and $t_x=|t_x|{\rm exp}(i\phi_{AB}m)$ are the associated tunnelings. $A_m$ captures the spin dependent diagonal elements, detuning $\hbar\delta$ and quadratic shift $\hbar\epsilon$. Here, we have implicitly restricted ourselves to the lowest 1-D lattice band, and assumed that tight binding, i.e. confinement at discrete lattice sites, is a good approximation (see \ref{sec:tightBinding}).  $t_s$ is not a spin dependent quantity for $F=1$ atoms, but is for $F=2$, where differences in Clebsch-Gordan coefficients create non-uniform tunneling. In the limit of zero detuning and neglecting the quadratic shift as well as the $t_s$ dependence on spin, this becomes the traditional Harper-Hofstadter Hamiltonian
\begin{equation}
H=-\sum_{j,m} t_x e^{i\phi_{\rm AB}m}\ket{j+1,m}\bra{j,m}+t_s|j,m+1\rangle\langle j,m| +  h.c.
\label{eqn:TBhamFlat}
\end{equation} 

We can transform this Hamiltonian into momentum space along \ex{} by plugging the Fourier transform formula 
\begin{equation}
|j,m\rangle = \frac{1}{\sqrt{N}}\sum_{k_j}e^{-ik_j j}|k_j,m\rangle,
\end{equation}
with $N$ the number of sites $j$, into the above Hamiltonian to obtain 
\begin{equation}
H=-\frac{1}{N}\sum_{k_j,m}t_s|k_j,m+1\rangle\langle k_j,m| + h.c. +2|t_x|{\rm cos}(k_j-\phi_{\rm AB})|k_j,m\rangle\langle k_j,m| + A_m|k_j,m\rangle\langle k_j,m|
\label{eqn:TBhamKspace}
\end{equation}

To approximate the full Hamiltonian, eqn. \ref{eqn:SynDimHam}, by the tight binding Hamiltonian, eqn. \ref{eqn:TBham}, we must find appropriate values for $t_s$ and $t_x$. We find $|t_x|$ by treating the 1-D lattice independently, and matching the tight binding band to the lowest full lattice band. For most of the experiments described in the chapter, the lattice depth was $V_0=6E_L$, corresponding to $|t_x|\approx0.01 E_L$. To find the appropriate value of $t_s$, we fit the full synthetic dimensions band structure to the tight binding band structure eqn. \ref{eqn:TBhamKspace} with $t_s$ as a free parameter. 

\begin{figure}
	\includegraphics{"SynDim Figures/tbFit".pdf}
\caption[Band structure of the tight binding versus full Hamiltonian]{Band structure of the tight binding versus full Hamiltonian. $V_0=6.0 E_L$, giving $|t_x|=0.1 E_L$, $\hbar\delta=0$, $\hbar\epsilon=0.02 E_L$, $\hbar\Omega=0.5 E_L$. (a) $F=1$, fitted value $t_s=0.154 E_L$. (b) $F=2$, fitted value $t_s=0.284$. }
\label{fig:tbFit}
\end{figure}

Figure \ref{fig:tbFit} shows the overlayed band structure of the full Hamiltonian, eqn. \ref{eqn:SynDimHam}, and the best fit tight binding band structure, eqn. \ref{eqn:TBhamKspace}. To fit, we minimize the square difference between the energies in the lowest two bands, relevant to our experiment. 

\section{Eigenstates of the synthetic 2-D lattice}

After calibrating the synthetic dimensional lattice via pulsing, we can study the eigenstates of the lowest band of the system by adiabatically loading, i.e. ramping both the lattice and Raman or rf coupling on on a time scale slow compared to the magnetic band spacing. Along the synthetic direction, in the $F=1$ manifold, there are no $m=\pm2$ sites. This can be thought of as hard wall boundary conditions at the $m=\pm2$ sites, confining the atoms in the allowed $m=0,\pm1$ sites. Therefore, we can consider the position eigenstates along the synthetic direction in relation to eigenstates of a square-well potential. 

\begin{figure}
	\includegraphics{"SynDim Figures/Fig2_EdgeStates".pdf}
\caption[Eigenstates of the synthetic dimensions lattice]{Eigenstates of the synthetic dimensions lattice. Left column: time-of-flight images, with position along $\ess{}$ on the vertical axis and momentum along $\ess{}$ on the horizontal. Right column: fractional populations in each site $m$.  (a,b) rf coupling, resulting in $\phi_{\rm AB}=0$. (c,f) Raman coupling, resulting in $\phi_{\rm AB}>0$, adiabatically loaded from the $m_F=1$ state.  (d,g) Raman coupling, resulting in $\phi_{\rm AB}>0$, adiabatically loaded from the $m_F=0$ state. (c,f) Raman coupling, resulting in $\phi_{\rm AB}>0$, adiabatically loaded from the $m_F=-1$ state.}
\label{fig:TOFeigenstates}
\end{figure}

Figure \ref{fig:TOFeigenstates}a shows a time-of-flight image of an adiabatically loaded synthetic dimensions lattice eigenstate with rf coupling along the synthetic direction. The vertical axis is single site resolved spin states $m$. The horizontal axis is momentum along the \ex{} direction. Note that for each site $m$ the distribution of momenta $k_x$ is symmetric. Figure \ref{fig:TOFeigenstates}b shows the fractional population in each site $m$, summed over all momenta $k_x$.  In the case of rf coupling, $\phi_{\rm AB}=0$ and the effective magnetic flux $\Phi_{\rm AB}/\Phi_0=0$.  Therefore, the fractional population along the spin direction looks simply like a discretized ground state probability distribution of the square well potential.

Figure \ref{fig:TOFeigenstates}c-h shows analogous data with Raman coupling along the synthetic direction. Figure \ref{fig:TOFeigenstates}d,g are the time-of-flight image and corresponding fractional populations of atoms adiabatically loaded from the $m_F=0$ spin state, corresponding to the central minimum ($q=0$) of the lowest band in Figure \ref{fig:SynDimBandStruct}b. There are two key differences between this case and the rf case in Figure \ref{fig:TOFeigenstates}a-b. First, the momenta of the different spin states are no longer symmetric, as explained in sec. \ref{sec:SynDimCalibration}. Second, the fractional populations in Figure \ref{fig:TOFeigenstates}g are no longer simply the discretized ground state probability distribution of the square well potential: they are a narrowed version of it, more strongly concentrated in the $m=0$ site. 

This can be understood by analogy with a 2D electron system in a perpendicular magnetic field, confined in one dimension with hard walls. Along the confined direction, the wavefunction is localized to the scale of the magnetic length ${\it l}_B=\sqrt{\hbar/qB}$, with the center position at $k_x {\it l}_B^2$ in the bulk state, where $\hbar k_x$ is the electron’s canonical momentum. In our system, the magnetic length ${\it l}_B=\sqrt{a^2 \Phi_0/2\pi\Phi_{\rm AB}}$, or in units of the lattice period $a$, ${\it l}^*_B=\sqrt{3/2\pi}$; this explains the narrowing of the bulk state in Figure \ref{fig:TOFeigenstates}g.

In the 2D electron system, at large $|k_x|$, the electron becomes localized near the edges, lifting the degeneracy of the otherwise macroscopically degenerate Landau levels. In our case, stable edge states appeared as the additional minima in Figure \ref{fig:SynDimBandStruct}b, at $q\approx\pm0.66 k_L$. We loaded these edge states by starting in the $m_F=\pm1$ states before adiabatically turning on the synthetic dimensions lattice to obtain the eignestates displayed in Figure \ref{fig:TOFeigenstates}c,f and Figure \ref{fig:TOFeigenstates}e,h respectively. These edge states predominantly occupy the edge sites in the synthetic direction, and are strongly confined there due to the narrow magnetic length. These localized edge states are the analog to the current-carrying edge states in Fermionic integer quantum Hall effect systems\cite{Hugel2014}.

\section{Chiral edge currents}

The same pulsing procedure that was used for calibration (sec. \ref{sec:SynDimCalibration}) can also be interpreted by analogy with the 2-D electron system. Figure \ref{fig:edgeCurrents}a shows schematically what happens when atoms are loaded from the $m=0$ site into the lattice and tunneling along the synthetic dimension is pulsed on. Atoms begin analogues of cyclotron orbits, tunneling out into the edge $m=\pm1$ sites and tunneling back to the  bulk $m=0$ state. The fractional populations in the three $m$ sites as a function of time are shown in Figure \ref{fig:edgeCurrents}b. 

We performed this experiment for three different magnetic flux values: with rf coupling giving $\Phi_{\rm AB}/\Phi_0=0$, with Raman coupling giving $\Phi_{\rm AB}/\Phi_0\approx4/3$ and with inverted Raman coupling giving $\Phi_{\rm AB}/\Phi_0\approx-4/3$.
The inverted Raman coupling was accomplished by switching the roles of the two Raman beams (see Figure \ref{fig:synDimSchematic}a): the right going beam frequency was changed to $2\pi (\omega + \Delta\omega)$ and the left going beam frequencdy was change to $2\pi\omega$, resulting in the opposite recoil momentum for the same spin flip, flipping the direction of the effective magnetic field. 

We  define the current $I_{m=\pm1}=n_m \langle v_m \rangle$, where $n_m$ is the fractional population in site $m$ and $\langle v_m \rangle$ is the expectation value of velocity along \ex{} for atoms in sites $m$, as depicted in Figure \ref{fig:edgeCurrents}a. The velocity is derived from the momentum measured in time-of-flight images. The chiral current of the system is then defined as $\mathcal{I}=I_1-I_{-1}$. We calculate this chiral current for data in  Figure \ref{fig:edgeCurrents}b, with $\Phi_{\rm AB}/\Phi_0\approx4/3$, displayed in red dots in Figure \ref{fig:edgeCurrents}c. Atoms in the edge sites $m=\pm1$ exhibit chiral motion, therefore the resulting chiral current is directly proportional to the fractional population in those sites and oscillates as a function of time in concert with the oscillation in Figure \ref{fig:edgeCurrents}b. in Figure \ref{fig:edgeCurrents}c also includes data for the $\Phi_{\rm AB}/\Phi_0\approx-4/3$ (solid black dots indicate data and solid gray lines are from theory) and $\Phi_{\rm AB}/\Phi_0=0$ (empty black dots). As seen in the figure, reversing the direction of the effective magnetic flux reverses the direction of the chiral currrent, and turning off the magnetic flux results in no net chiral current. The chiral current $\mathcal{I}$ is normalized here by the tunneling velocity $2 t_x/\hbar k_L$. 

\begin{figure}
	\includegraphics{"SynDim Figures/Fig3_EdgeCurrent".pdf}
\caption[Measuring chiral currents in synthetic dimensions]{Measuring chiral currents in synthetic dimensions. (a) Schematic of the formation of chiral currents when the system is loaded into the bulk $m=0$ site and tunneling along $\ess{}$ is turned on suddenly. They start to tunnel both towards the $m=1$ sites (pink arrows), moving towards the right along \ex{}, and the $m=-1$ sites (blue arrows), moving towards the left along \ex{}. They then return to $m=0$, completing cyclotron orbits. (b) Fractional population in each spin state ($m=0$, $+1$, and $-1$ in green, pink, and blue respectively) as a function of time for a system with $\phi_{\rm AB}>0$. Dots represent data and lines represent theory calculated from the full Hamiltonian, eqn. \ref{eqn:SynDimHamiltonian}, with parameters $\hbar\Omega=0.5 E_L$, $V_0=6 E_L$, $\hbar\delta=0.001 E_L$, and $\hbar\epsilon=0.05 E_L$. (c) Chiral current $\mathcal{I}$ as a function of time for $\phi_{\rm AB}>0$ (red) $\phi_{\rm AB}=0$ (empty black dots) and $\phi_{\rm AB}<0$ (solid black). (d) Chiral current $\mathcal{I}$ as a function of $\langle|m|\rangle$ for the three values of $\phi_{\rm AB}$. Solid lines calculated from theory, with the same parameters as in (b) for $\phi_{\rm AB}\neq0$, and with parameters $\hbar\Omega=033 E_L$, $V_0=6 E_L$, $\hbar\delta=-0.01 E_L$, and $\hbar\epsilon=0.05 E_L$ for $\phi_{\rm AB}=0$.  (e) Peak chiral current $\mathcal{I}_{\rm max}$ as a function of tunneling asymmetry $t_s/t_x$, as measured experimentally (red dots) and calculated from the full Hamiltonina (red line). Inset: slope of best fit lines of current $\mathcal{I}$ as a function of $\langle|m|\rangle$ (as in (d)) as a function of tunneling asymmetry $t_s/t_x$: nearly independent. }
\label{fig:edgeCurrents}
\end{figure}

As the chiral current $\mathcal{I}$ is proportional to the edge state population, we plot it as a function of the expectation value of the absolute value of m, $\langle|m|\rangle$, in Figure \ref{fig:edgeCurrents}d. As expected, the chiral current is linear, with slope $\mathcal{S}$. $\mathcal{S}$ is positive for $\Phi_{\rm AB}/\Phi_0\approx4/3$, negative for $\Phi_{\rm AB}/\Phi_0\approx-4/3$, and zero for $\Phi_{\rm AB}/\Phi_0=0$. We then study the dependence of the chiral current on the strength of tunneling along the synthetic dimension, in units of the real axis tunneling $t_s/t_x$. We refer to this as the tunneling anisotropy: the asymmetry between the two dimensions. As shown in the inset to Figure \ref{fig:edgeCurrents}e, the slope $\mathcal{S}$ of the chiral current as a function of $\langle|m|\rangle$ is practically independent of the tunneling anisotropy. The small deviation from a flat line is explained by the deviation of our system from the tight binding model. However, the maximal chiral current attained during the pulsing experiment, $\mathcal{I}_{\rm max}$, depends strongly on the tunneling anisotropy (see Figure \ref{fig:edgeCurrents}e). This is because the maximum fractional population in the edge states $\langle|m|\rangle$ increases with increased $t_s$. The increase is approximately linear at first, and then saturates at large $t_s/t_x$ when the fractional population in the edge states $m=\pm1$ approaches $1$.


\section{Observation of skipping orbits}

Semi-classically, electrons in a 2-D material pierced by a magnetic field can be described in terms of cyclotron orbits in the bulk, as described in the previous section, and skipping orbits on the edge. Skipping orbits arise from electrons on the edge beginning cyclotron orbits, but hitting the edge of the system and being reflected and beginning the next cyclotron orbit. Due to the chirality of the cyclotron orbits, this results in the skipping orbits traveling in one direction along the top edge and in the opposite direction along the bottom edge.   

We observed an analogue of these skipping orbits in our system. We performed the same experiment, pulsing on tunneling along the synthetic dimension, but this time initializing the system on the edge, as shown schematically in Figure \ref{fig:skippingOrbits}a. To populate these states, we initially applied a detuning $\hbar\delta=\pm0.087 E_L$, tilting the potential along the synthetic direction as shown in Figure \ref{fig:skippingOrbits}b. This made the initial state, $m=-1$ in the figure, a potential minimum. We then pulsed on the tunneling and observed the resulting dynamics. 

\begin{figure}
	\includegraphics{"SynDim Figures/Fig4_SkippingOrbit".pdf}
\caption[Imaging skipping orbits]{Imaging skipping orbits. (a) Schematic of pulsing experiment when atoms are initialized on the $m=-1$ edge. The atoms move towards the $m=0$ site, while moving to the right along \ex{} (blue arrows). They then continue their semi-cyclotron orbits back to $m=-1$, from where they cannot finish the circle, forming skipping orbits.  (b) Schematic of the tilted box potential applied along the synthetic direction. (c) Expectation value of position along $\ess{}$, $\langle m \rangle$, as a function of pulse time for atoms initialized in the $m=+1$ (red) and $m=-1$ (blue) states. Dots represent data and lines are from theory calculated from the full Hamiltonian, eqn. \ref{eqn:SynDimHamiltonian}, with parameters $\hbar\Omega=0.41 E_L$, $V_0=5.2 E_L$, $\hbar\delta=\pm0.087 E_L$, and $\hbar\epsilon=0.13 E_L$. (d) Expectation value of the group velocity along \ex{}, $\langle v_x\rangle$, for the same data as in (c). (e) Expectation value of displacement along \ex{},  $\langle \delta j \rangle$ in units of lattice spacing, for the same data as in (c) and (d). The displacement was obtained by integrating $\langle v_x/a \rangle$, where $a$ is the period of the optical lattice. Atoms initialized in $m=-1$ performed skipping orbits to the left, while atoms starting in $m=-1$ traveled to the right.  }
\label{fig:skippingOrbits}
\end{figure}

Figure \ref{fig:skippingOrbits}c shows the expectation value of position along $\ess{}$ as a function of time during the pulsing experiment. This expectation value is obtained by calculating the fractional population $n_m$ on each site $m$ and summing $\langle m\rangle = \sum_m m n_m$. The red dots were obtained from an experiment  where the atoms were initialized in the $m=1$ site. The blue dots were obtained by starting in the $m=-1$ site. The expected position oscillated with time, as expected for Rabi oscillations. The same data was then used to extract the expected group velocity along \ex{}, $\langle v \rangle = \sum_m n_m \langle v_m \rangle$ as a function of time. This is shown in Figure \ref{fig:skippingOrbits}d. The group velocity oscillated with the expected position $\langle m \rangle$, and was positive for experiments starting in $m=-1$ and negative for experiments starting in $m=1$. 

We obtain the expected displacement in units of the lattice spacing $a$, $\langle \delta j \rangle$ along \ex{} as a function of time by directly integrating the expected group velocity. The resulting displacement is shown in Figure \ref{fig:skippingOrbits}e. As seen in the figure, for experiments initialized in $m=1$, the atoms began cyclotron orbits, but reflected off the edge and performed skipping orbits towards the left. Likewise, atoms initialized in $m=-1$ performed skipping orbits along the opposite edge and in the opposite direction. This experiment presents the first direct observation of skipping orbit motion. 
