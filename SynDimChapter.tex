\renewcommand{\thechapter}{6}

\chapter{Synthetic Magnetic Fields in Synthetic Dimensions}


In condensed matter, 2-D systems in high fields have proved to be of great technological use and scientific interest. The integer quantum Hall effect (IQHE)\cite{Klitzing1980}, with it's quantized Hall resistance, had given rise to an ultra-precise standard for resistivity. It was also one of the first examples of topology playing an important role in physics---the precise quantization of the Hall conductance is guaranteed by the non-trivial topology of the system\cite{Thouless1982}. This quantizes the magnetic flux into flux quanta of $\Phi_0=2\pi\hbar/e$, where $e$ is the electron charge, and leads to a new 'plateau' in the resistivity when an additional quantum of flux is threaded through the system. 

In the IQHE system, the underlying lattice structure of metal is effectively washed out---the magnetic flux per individual lattice plaquette is negligible. However, new physics arises when the magnetic flux per plaquette is increased to some non-negligible fractionj of the flux quantum, giving rise to the Hofstadter butterfly\cite{Hofstadter1976}. These regimes are har to reach experiementally, since the typical plaquette size in crystalline material is of order a square angstrum, and the magnetic field necessary to thread create a magnetic flux of $\Phi_0$ through such a narrow area is of order $\approx10^4$ Tesla, not accessable with current technology. 

Several platforms have, however, reached the Hofstadter regime by engineering systems with large effective plaquette size, in engineered materials\cite{Geisler2004,Hunt2013}, and in atomic\cite{Jaksch2003,Aidelsburger2013,Miyake2013,Jotzu2014,Aidelsburger2014,Mancini2015} and optical\cite{Hafezi2013} settings. Here, we use the approach of synthetic dimensions \cite{Celi2014} to reach the Hofstadter regime. We demonstrate the non-trivial topology of the system created, and use it to image skipping orbits at the edge of the 2-D system---a hallmark of 2-D electron systems in a semiclassical treatment. 

The work described in this chapter was published in\cite{Stuhl2015}.

%	In materials the quantum Hall effects represent an extreme quantum limit, where a system's behavior defies any description with classical physics.  In the modern parlance, the integer quantum Hall effect (IQHE) for two-dimensional (2-D) electronic systems in magnetic fields\cite{Klitzing1980} was the first topological insulator\cite{Hasan2010}: a bulk insulator with dispersing edge states --- always present in finite-sized topological systems--- which give rise to the IQHE's signature quantized Hall resistance\cite{Thouless1982}.  
%	
%	In classical systems the magnetic field acts purely through the Lorentz force, while in quantum systems a particle with charge $q$ in a uniform field $B$ additionally acquires an Aharonov-Bohm phase $\phi_{\rm AB}/2\pi = \mathcal{A} B / \Phi_0$ after its path encircles an area $\mathcal{A}$ normal to $B$.  (Here $\Phi_0=2\pi\hbar/q$ is the flux quanta and $2\pi\hbar$ is Planck's constant.)   As proposed in Ref.~\cite{Celi2014}, we engineered an elongated 2-D square lattice formed from the sites of an optical lattice along the long direction and the internal atomic spin states along the narrow direction: a synthetic dimension.  We directly controlled the acquired phases as atoms traversed the lattice, giving a tunneling phase $\phi_{\rm AB}/2\pi \approx4/3$ around each plaquette.  These phases take the place of the Aharonov-Bohm phases produced by true magnetic fields and suffice to fully define the effective magnetic field.  Aharonov-Bohm phases of order unity in the Harper-Hofstadter Hamiltonian ---currently realized in engineered materials\cite{Geisler2004,Hunt2013}, or in atomic\cite{Jaksch2003,Aidelsburger2013,Miyake2013,Jotzu2014,Aidelsburger2014,Mancini2015} and optical\cite{Hafezi2013} settings--- fragment the low-field Landau levels into the fractal energy bands of the Hofstadter butterfly\cite{Hofstadter1976}.   Such Hofstadter bands are generally associated with a non-zero topological index: the Chern number\cite{Thouless1982}.
%	
%	Topologically nontrivial bulk properties are reflected by the presence of edge channels, composed of edge states, with quantized conductance.  In fermionic systems, the number of edge channels is fixed by the aggregate topological index of the filled bands\cite{Thouless1982,Hasan2010,Beugeling2012}; this ultimately gives rise to phenomena such as the IQHE for electrons.    Conceptually the constituent edge states can be viewed as skipping orbits\cite{Buttiker1988PRB,Hasan2010,Montambaux2011}: in the presence of a strong magnetic field, nascent cyclotron orbits near the boundary reflect from the hard wall before completing an orbit, leading to skipping trajectories following the system's boundary.  In contrast, localized bulk states correspond to closed cyclotron orbits.
%	
%	By applying large effective fields to atomic Bose-Einstein condensates (BECs), we directly imaged individual, deterministically prepared, bulk and edge eigenstates.  In IQHE systems these states would govern the conductivity, but as individual eigenstates they exhibit no time-dependance.  The corresponding dynamical entities are edge magnietoplasmons, consisting of superpositions of edge eigenstates in different Landau levels\cite{Kern1991,Ashoori1992}, or here magnetic bands.  We launched these excitations and recorded their full motion for the first time, observing both a chiral drift along the system's edge and the underlying skipping motion.

\section{Synthetic dimensions setup}

Any internal degree of freedom can be thought of as a synthetic dimension---the different internal states can be treated as sites along this synthetic direction. As long as there is some sense of distance along this direction, i.e. some of the internal states are 'nearest neighbors' while others are not, this is a meaningful treatment. In our case an effective 2-D lattice is formed by sites formed by a 1-D optical lattice along a 'real' direction, here  ${\bf e}_x$, and the atom's spin states forming sites along a 'synthetic' direction,  ${\bf e}_s$. 

The experimental setup for this system is schematically represented in \ref{fig:synDimSchematic}a. The BEC is subject to a 1-D optical lattice, formed by a retro-reflected beam of $\lambda_L=1064 nm$ along  ${\bf e}_x$. A bias magnetic field $B_0$ along  ${\bf e}_z$ separates the different spin states. The spin states can be thought of a sites along a synthetic dimension even without any coupling field. However, only once a coupling field is present do they acquire a sense of distance. We couple them via rf or Raman coupling, which only couples adjacent spin states. The Raman beams illuminating the atoms are along the same  ${\bf e}_x$ direction as the 1-D optical lattice. The rf field has components both along the  ${\bf e}_x$ and  ${\bf e}_y$. 


\begin{figure}
	\includegraphics{"SynDim Figures/Fig1_Schematic".pdf}
\caption{Setup of effective 2-D lattice. (a) Beam geometry. The BEC is subject to a bias magnetic field $B_0$ in the ${\bf e}_z$ direction. The 1-D lattice beam and Raman beams are both along the ${\bf e}_x$ direction, and the rf field can be applied with projections onto both the  ${\bf e}_x$ and  ${\bf e}_y$. (b) Schematic of the effective 2-D lattice. Sites along  ${\bf e}_x$ are formed by the 1-D optical lattice and labelled by site number $j$. Sites along the synthetic direction  ${\bf e}_s$ are formed by the spin states: 3 sites for atoms in the $F=1$ manifold and 5 sites for atoms in $F=2$. These sites are labelled by $m$. Raman transitions induce a phase shift, which can be gauge transformed into a tunneling phase along the  ${\bf e}_x$ direction. This leads to a net phase when hopping around a single lattice plaquette of $\phi_{AB}$.  }
\label{fig:synDimSchematic}
\end{figure}


Figure \ref{fig:synDimSchematic}b sketches out the effective 2-D lattice created. Here, we have labelled the lattice sites along the 'real' direction  ${\bf e}_x$ by site index $j$. In the tight binding approximation, we can describe a lattice hopping between adjacent sites with tunneling amplitude $t_x$. Similarly, the sites along the 'synthetic' dimension are labelled by site index $m$ (identical to spin projection quantum number $m_F$), and the rf or Raman coupling here plays the role of a tunneling amplitude $t_s$. In the case of rf coupling, there is no momentum kick associated with spin exchange, and both $t_x$ and $t_s$ are real. 

In the case of Raman coupling, however, there is a momentum kick of $2k_R$ associated with every spin transfer, and therefore a phase factor of ${\rm exp}(2i k_R x)$ with every spin 'tunneling' event. Since position $x$ is set by the 1-D lattice, $x_j = j \lambda_L/2 = j \pi/k_L$, and the space dependent phase factor is ${\rm exp}(2\pi i k_R/k_L j)$. An absolute phase change in the wavefunction is not meaningful. However, a phase aquired when going around a plaquette and coming back to the same palce is meaningful, as one could imagine one atom staying at the same site and the other going around a plaquette and coming back to detect the phase difference. In this setup, the phases acquired while going around a single plaquetter are, starting at some lattice site $\ket{j,m}$, are: $0$ (for tunneling right to $\ket{j+1,m}$), $2\pi i k_R/k_L (j+1)$ (for tunneling up to $\ket{j+1,m+1}$, $0$ (for tunneling left to $\ket{j,m+1}$) and $-2\pi i k_R/k_L j$ (for tunneling back down to $\ket{j,m}$). The total phase aquired is thus $\phi_{\rm AB} = 2\pi k_R/k_L$, independent of the starting lattice site. Since the absolute phase does not matter and only the value of t $\phi_{\rm AB}$, we can perform a phase transformation that shifts the tunneling phase onto the spatial direction, defining $t_x = |t_x|{\rm exp}(i\phi_{\rm AB}m)$ and $t_s=|t_s|$, as labelled in Figure \ref{fig:synDimSchematic}b. 

To see how this phase implies an effective magnetic field, we draw an analogy to the Aharonov-Bohm effect\cite{Aharonov1959, Aharonov1992} from quantum mechanics. In this effect, consider an infinite solenoid with an electric current running through it. The magnetic field $B$ in this setup exists only inside the solenoid, while the magnetic vector potential persists outside the solenoid.  However, if two electrons are sent on a trajectory around the solenoid, even though they never pass through any magnetic field, they neverthless acquire a relative phase that can be detected by interfering them with each other. This relative phase is given by $\phi_{\rm AB} =2 \pi \Phi/\Phi_0$, where $\Phi = B*A$ is the magnetic flux through the solenoid ($A$ is the are pierced by the magnetic field) and $\Phi_0=h/e$ is the flux quantum, with $e$ the electron charge. Since in our system, the atoms acquire a phase when they perform a closed loop around a single lattice plaquette. Therefore, they behave as though there was an infinite solenoid piercing each plaquette with a magnetic field going through it, and the flux per plaquette in units of the flux quantum is $\Phi/\Phi_0=\phi_{\rm AB}/2\pi =  k_R/k_L$. For the case of rf coupling, the phase acquired at every transition is $0$ and the fluxss $\Phi/\Phi_0=0$.

\section{Hamiltonian of the effective 2-D system}

\subsection{Hamiltonian}
The full Hamiltonian of this system, without making the tight binding approximation, can be written down by combining the lattice Hamiltonian (eqn. \ref{eqn:LatHam}) and the rf (eqn. \ref{eqn:rfHam}) or Raman Hamiltonian (eqn. \ref{eqn:RamanHam}). To do this, we write a new basis that encompasses both the momentum and the spin degrees of freedom. For the lattice Hamiltonian, we used the momentum basis
\begin{equation}
 \begin{pmatrix} \vdots \\
\ket{q+4 k_L}\\
\ket{q+2 k_L}\\
\ket{q}\\
\ket{q -2 k_L}\\
\ket{q - 4 k_L}\\
\vdots
\end{pmatrix} \\.
\end{equation}
For the Raman Hamiltonian in the $F=1$ manifold, we used the spin and momentum basis
\begin{equation}
\begin{pmatrix}
\ket{k_x-2k_R,-1}\\
 \ket{k_x,0}\\
\ket{k_x+2k_R,1}
\end{pmatrix}\\.
\end{equation}

In a lattice, the momentum $k_x$ becomes crystal momentum $q$. For every state in the lattice basis, we now expand to three states, one for each spin state, with the appropriate momentum shifts. We obtain
\begin{equation}
 \begin{pmatrix} \vdots \\
\ket{q+2k_L-2k_R,-1}\\
 \ket{q+2k_L,0}\\
\ket{q+2k_L+2k_R,1}\\
\ket{q-2k_R,-1}\\
 \ket{q,0}\\
\ket{q+2k_R,1}\\
\ket{q-2k_L-2k_R,-1}\\
 \ket{q-2k_L,0}\\
\ket{q-2k_L+2k_R,1}\\
\vdots
\end{pmatrix} \\.
\end{equation}

In this basis, we combine the lattice and Raman Hamiltonians (ommiting the kinetic energy in the other two directions) in an infinite block matrix form as 
\begin{equation}
H =
 \begin{pmatrix} \ddots &  & & & \\ 
 &{\bf H_R}(2k_L)  & {\bf \frac{V_0}{4}} &{\bf 0} &  \\
 &  {\bf \frac{V_0}{4}} &{\bf H_R}(0) & {\bf \frac{V_0}{4}}&  \\
 & {\bf 0} &  {\bf \frac{V_0}{4}} & {\bf H_R}(-2k_L)  &  \\
 & & & &  \ddots \end{pmatrix} \\,
\label{eqn:SynDimHam}
\end{equation}
where ${\bf H_R}(x)$ is the Raman Hamiltonian with a momentum shift of $x$:
 \begin{equation}
{\bf H_R}(n 2k_L) = 
 \begin{pmatrix} \frac{\hbar^2(q + n 2k_L -2k_R)^2}{2m}+\hbar\delta & \hbar\Omega/2  &  0  \\ 
\hbar\Omega/2 & \frac{\hbar^2 (q+n 2k_L)^2}{2m} - \hbar\epsilon &  \hbar\Omega/2\\
 0 & \hbar\Omega/2 &  \frac{\hbar^2(q+n 2k_L+2k_R)^2}{2m} -\hbar\delta  \\
 \end{pmatrix} \\,
\end{equation}
the matrix ${\bf \frac{V_0}{4}}$ is a 3x3 diagonal matrix lattice coupling strength $\frac{V_0}{4}$ on the diagonal, and $\bf 0$ is a 3x3 matrix of zeros. This extends in both directions with ${\bf H_R}(2nk_L)$ on the diagonal blocks and ${\bf \frac{V_0}{4}}$ as the first off-diagonal blocks and $\bf 0$ everywhere else. 

This Hamiltonian is easily extended to higher $F$ values by replacing the Raman blocks ${\bf H_R}(x)$ with the corresponding Raman coupling Hamiltonian from eqn. \ref{eqn:RamanAllF}, and extending the diagonal matrix  ${\bf \frac{V_0}{4}}$ and the zero matrix $\bf 0$ to be ($2F+1$)x($2F+1$).

For computational convenience, we convert to lattice recoil units, $E_L=\hbar^2 k_L^2/2m$, $k_L=2\pi/\lambda_L$. Then the diagonal blocks become
 \begin{equation}
{\bf H_R}(n)/E_L = 
 \begin{pmatrix} (q + n -\phi_{\rm AB}/2\pi)^2+\hbar\delta & \hbar\Omega/2  &  0  \\ 
\hbar\Omega/2 & (q+n)^2 - \hbar\epsilon &  \hbar\Omega/2\\
 0 & \hbar\Omega/2 & (q + n +\phi_{\rm AB}/2\pi)^2 -\hbar\delta  \\
 \end{pmatrix} \\,
\end{equation}
where $\hbar\delta$, $\hbar\Omega$ and $\hbar\epsilon$ are now written in units of $E_L$, $q$ is written in units of $k_L$ and we have used the fact that $\phi_{\rm AB}/2\pi = k_R/k_L$. The off-diagonal blocks ${\bf \frac{V_0}{4}}$ will be the same 3x3 diagonal matices, with $\frac{V_0}{4}$ in units of $E_L$. 

\subsection{Band structure}
\begin{figure}
	\includegraphics{"SynDim Figures/SynDimBandStructure".pdf}
\caption{Band structure of the synthetic dimensions Hamiltonian, eqn. \ref{eqn:SynDimHam}. For all panels, the detuning $\hbar\delta=0$ and the quadratic shift $\hbar\epsilon=0.02 E_L$. (a) $F=1$, $\hbar\Omega=0.0$. (b) $F=1$, $\hbar\Omega=0.5$. (c) $F=2$, $\hbar\Omega=0.0$. (d) $F=2$, $\hbar\Omega=0.5$. }
\label{fig:SynDimBandStruct}
\end{figure}

The band structure of this Hamiltonian is presented in Figure \ref{fig:SynDimBandStruct}. Here, we have restricted ourselves to the lowest lattice band. We can do this because the energy splitting between the lowest and second lowest lattice band is of order $4 E_L$ (see Figure \ref{fig:latticeBandStructure}), while the width of the lowest band, given by the amplitude of the approximate sinusoid, is of order $0.3 E_L$ for our range of lattice depths, around $5.0 E_L$. As long as the Raman coupling stays small compared to the lattice band spacing, the higher lattice bands are energetically separated enough that they can be ignored. 

Therefore, we can think of the Raman coupling analogously to the free space Raman coupling (see section \ref{sec:Raman}), except instead of free space parabolas each spin state gets a lowest lattice band sinusoid.   Figure \ref{fig:SynDimBandStruct}a shows this in the limit of no Raman coupling, $\Omega=0$, but with the lattice on at $V_0=4.0 E_L$. The quadratic Zeeman shift is $\hbar\epsilon=0.02 E_L$ and the detuning $\delta=0$. The $m_F=-1$ sinusoid is shifted $2k_R$, similarly to section \ref{sec:Raman}, but since the sinusoid is periodic with $2k_L$, it folds into the first Brillouin zone of the lattice, such that the nearest minimum to $q=0$ is at $q = 2k_R-2k_L = (2\phi_{\rm AB}/2\pi - 2)k_L$. The edges of the Brillouin zone are marked by horizontal lines. The color indicates magnetization $\langle m\rangle=\sum_{m_F}m_F*n_{m_F}$, where $n_{m_F}$ is the fractional population in the $m_F$ state. In synthetic dimensions language, $\langle m \rangle$ is the expectation value of position along ${\bf e}_s$.

In Figure \ref{fig:SynDimBandStruct}b, we have restricted ourselves to the first Brillouin zone and turned on the Raman coupling to $\hbar\Omega = 0.5 E_L$. The crossings of the bands in Figure \ref{fig:SynDimBandStruct}a become avoided crossings, and the lowest band now has a spin dependence on crystal momentum. Figure \ref{fig:SynDimBandStruct}c-d shows the same progression for the $F=2$ manifold. Figure \ref{fig:SynDimBandStruct}c is taken in the limit of $\hbar\Omega=0.0$. All of the 5 spin states get 'folded' back into the first Brillouin zone due to the lattice periodicity of the bands. The different heighs of the sinusoids are due to the quadratic Zeeman shift $\hbar\epsilon=0.02 E_L$. The lattice depth is again $V_0=5.0 E_L$ and detuning $\hbar\delta=0$. In Figure \ref{fig:SynDimBandStruct}d we have restricted ourselves to the first Brillouin zone and turned on the Raman coupling to $\hbar\Omega=0.5 E_L$.  Note that the inverted hyperfine structure in \ref{fig:SynDimBandStruct}c (meaning that the quadratic shift pushed the $m_F=0$ state up rather than down in energy compared to the others), combined with the Raman coupling serves to make the lowest band in the $F=2$ manifold close to flat. 


\subsection{Calibration}

To calibrate the lattice depth $V_0$ in the synthetic dimensions system, we can simply calibrate the lattice depth without Raman or rf coupling as described in Section \ref{sec:LatticeCalib}. However, we are operating at very low Raman coupling strengths, $\hbar\Omega\approx0.5 E_L$. This is necessary because in the synthetic dimensional system the Raman coupling plays the role of tunneling, which has to be small to approximate the tight binding limit. At these low Raman couplings, simple pulsing as described in Section \ref{sec:RamanCalib}, as the contrast of the Rabi oscillations would be too low to resolve. Therefore, we calibrate the Raman coupling and detuning with the full synthetic dimensions system, where the 'folding in' effect of the lattice makes the higher Raman bands much closer energetically than without the lattice, leading to larger constrast and allowing for accurate calibration.  

\begin{figure}
	\includegraphics{"SynDim Figures/SynDimPulsing".pdf}
\caption{Calibration of synthetic dimesnions lattice. (a) Ramping procedure. The blue line represents the 1-D lattice depth as a function of time and the red line represents Raman coupling as a function of time. Both are held on for a variable amount of time $t$, producing Rabi oscillations. (b) Example of fractional populations in different $m$ states as a function of time $t$ in the $F=1$ manifold. Dots indicate data and lines indicate the best fit to theory, with parameters $\hbar\Omega=0.56\pm0.01 E_L$ and $\hbar\delta = 0.029 \pm 0.002 E_L$. (c) Example time-of-flight image in the $F=1$ manifold. A Stern-Gerlach gradient pulse separates different $m$ states along the horizontal axis, while the lattice and Raman beams give momentum along the vertical axis. (d)  Example of fractional populations in different $m$ states as a function of time $t$ in the $F=2$ manifold. Dots indicate data and lines indicate the best fit to theory, with parameters $\hbar\Omega=0.61\pm0.002 E_L$ and $\hbar\delta = 0.002 \pm 0.001 E_L$. (e) Example time-of-flight image in the $F=2$ manifold. A Stern-Gerlach gradient pulse separates different $m$ states along the horizontal axis, while the lattice and Raman beams give momentum along the vertical axis.  }
\label{fig:SynDimPulsing}
\end{figure}

To do this, we must first adiabatically load the lowest 1-D lattice band. To do that, we must ramp on the lattice potential on a time scale slow compared to the band spacing, $\approx 4 E_L$. This gives $t\approx h/4E_L=0.12$ ms. Figure \ref{fig:SynDimPulsing}a shows the full ramping scheme. We ramp the lattice on in $\approx 20$ ms. Then, we must pulse on the Raman coupling on a time scale fast compared to the spin sub-band level spacing to produce Rabi oscillations, but still adiabatic with respect to the lattice spacing to avoid exciting to the higher lattice band. We ramp the Raman beams on in $300$ ms. Then, the system is held on for a variable amount of time before all light is snapped off and the atoms are allowed to expand in time-of-flight. For the case of $F=2$ atoms, the transfer to the $F=2$ manifold is done in the 1-D lattice before the Raman beams are ramped on to minimize the time spend in the $F=2$ manifold. 

Figure \ref{fig:SynDimPulsing}c,d shows sample time-of-flight images during the calibration procedure for $F=1$ and $F=2$ respectively. The vertical axis is ${\bf e}_x$, algned with the lattice and Raman beams. Since the atoms have expanded in time-of-flight, this axis corresponds to the momentum $k_x$. The horizontal axis of the image is the axis along which a Stern-Gerlach magnetic field gradient, separating the different spin states, is applied. Therefore, this axis is the position $m$ along the synthetic dimension ${\bf e}_s$. In the effectively 2-D synthetic dimensions lattice language, this is a 'hybrid' imaging techniqe---imaging momentum along one lattice direction and position along the other.  

Figure \ref{fig:SynDimPulsing}c labels some notable momentum orders. The central order is at $k_x=0$, where the atoms start before the expeirment. Two higher lattice orders, at $k_x=\pm2k_L$, are populated for the same spin $m=0$. $k_x=\pm2k_R$ is labelled, but not visibly populated, to indicate where the orders would appear of only Raman coupling was present with a higher coupling strength. Due to the 'folding in' effect of the lattice, the brightest orders of the $m=\pm 1$ states appear at $k_x=\pm (2k_L-2k_R)$. The $F=2$ states follow the same patern, not labelled in Figure \ref{fig:SynDimPulsing}e as there are too many orders.  

For each value of the time $t$ we sum up the total optical depth in all of the orders of each spin state to obtain fractional populations for each spin state as a function of time. An example scan in the $F=1$ manifold is shown in Figure \ref{fig:SynDimPulsing} b. The colored dots represent the data for different spin states, and the lines represent the best fit to theory. Here, the significant detuning makes populations in the $m=\pm 1$ states unequal. An example scan in the $F=2$ manifold is shown in Figure \ref{fig:SynDimPulsing}e. Here, the detuning is small and states with opposite spin oscillate in approximate unison.

\subsection{Tight binding approximation}

The synthetic dimensions Hamiltonian can be approximated in the tight binding limit as:
\begin{equation}
H=-\sum_{j,m} t_x e^{i\phi_{\rm AB}m}\ket{j+1,m}\bra{j,m}+t_s(m)|j,m+1\rangle\langle j,m| + A_m|j,m\rangle\langle j,m| + h.c.,
\end{equation}
where $j$ and $m$ label sites along ${\bf e}_x$ and ${\bf e}_s$ respectively, as shown in Figure \ref{fig:synDimSchematic}b. $t_s=|t_s|$ and $t_x=|t_x|{\rm exp}(-i\phi_{AB}m$ are the associated tunnelings. $A_m$ captures the spin dependent diagonal elements, detuning $\hbar\delta$ and quadratic shift $\hbar\epsilon$. Here, we have implicitly restricted ourselves to the lowest 1-D lattice band, and assumed that tight binding, ie confinement at discrete lattice sites, is a good approximation (see \ref{sec:thightBinding}.  $t_s$ is not a spin dependent quantity for $F=1$ atoms, but is for $F=2$, where differences in Clebsch-Gordan coefficients create non-uniform tunneling. In the limit of zero detuning and neglecting the quadratic shift as well as the $t_s$ dependence on spin, this becomes the traditional Harper-Hofstadter Hamiltonian. 

We can transform this Hamiltonian into momentum space along ${\bf e}_x$ by plugging the Fourier transform formula 
\begin{equation}
|j,m\rangle = \frac{1}{\sqrt{N}}\sum_{k_j}e^{-ik_j j}|k_j,m\rangle
\end{equation}
into the above Hamiltonian to obtain 
\begin{equation}
H=-\frac{1}{N}\sum_{k_j,m}t_m|k_j,m+1\rangle\langle k_j,m| + h.c. +2t_x{\rm cos}(k_j-\phi_{\rm AB})|k_j,m\rangle\langle k_j,m| + A_m|k_j,m\rangle\langle k_j,m|
\label{eqn:TBhamKspace}
\end{equation}

To go from the full Hamiltonian, eqn. \ref{eqn:SynDimHam}, to the tight binding Hamiltonian we must find appropriate values for $t_s$ and $t_x$. We find $|t_x|$ by treating the 1-D lattice independently, and matching the tight binding band to the lowest full lattice band. For most of the experiments described in the chapter, the lattice depth was $V_0=6E_L$, corresponding to $|t_x|\approx0.01 E_L$. To find the appropriate value of $t_s$, we fit the full synthetic dimensions band structure to the tight binding band structure eqn. \ref{eqn:TBhamKspace} with $t_s$ as a free parameter. 

\begin{figure}
	\includegraphics{"SynDim Figures/tbFit".pdf}
\caption{Band structure of the tight binding versus full Hamiltonian. $V_0=6.0 E_L$, giving $t_x=0.1 E_L$, $\hbar\delta=0$, $\hbar\epsilon=0.02 E_L$, $\hbar\Omega=0.5 E_L$. (a) $F=1$, fitted value $t_s=0.154 E_L$. (b) $F=2$, fitted value $t_s=0.284$. }
\label{fig:tbFit}
\end{figure}

Figure \ref{fig:tbFit} shows the overlayed band structure of the full Hamiltonian, eqn. \ref{eqn:SynDimHam}, and the best fit tight binding band structure, eqn. \ref{eqn:TBhamKspace}. To fit, we minimize the square difference between the energies in the lowest two bands, relevant to our experiment. 

\section{Eigenstates of the synthetic 2-D lattice}

\begin{figure}
	\includegraphics{"SynDim Figures/Fig2_EdgeStates".pdf}
\caption{}
\label{fig:TOFeigenstates}
\end{figure}

par1: 'Edge' and 'bulk' states of electrons in a magnetic field

par2: Explain loading procedure for adiabatic states

par3: Explain figure, magnetic length

\section{Chiral edge currents}

\section{Observation of skipping orbits}

\begin{figure}
	\includegraphics{"SynDim Figures/Fig4_SkippingOrbit".pdf}
\caption{}
\label{fig:skippingOrbits}
\end{figure}

par1: Skipping orbits in 2-D electrons in a magnetic fields

par2: Experimental sequence for skipping orbits measurement

par3: How position is calculated from integrating velocity as a function of time. Explain figure